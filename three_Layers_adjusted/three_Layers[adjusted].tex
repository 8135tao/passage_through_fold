\documentclass[letterpaper,11pt]{article}

\usepackage{ucs}
\usepackage[utf8x]{inputenc}
\usepackage{graphicx}
\usepackage{amsfonts}
\usepackage{dsfont}
\usepackage{amssymb}
\usepackage{amsmath}
\usepackage{amsthm}
\usepackage[titletoc]{appendix}

\usepackage{enumerate}
\usepackage{stmaryrd}
\usepackage{fullpage}
\usepackage{ifthen}
\usepackage{subfigure}
\usepackage{epic}
\usepackage{authblk}
\usepackage{textcomp}
\usepackage[small]{caption}
\SetSymbolFont{stmry}{bold}{U}{stmry}{m}{n}

\usepackage{enumitem}

\usepackage[hypertexnames=false,colorlinks=true,linkcolor=blue,citecolor=blue]{hyperref}
\usepackage[numbers,comma,square,sort&compress]{natbib}
\usepackage[letterpaper,text={7in,9in},centering]{geometry}

\usepackage{bm}
\usepackage{color}
\usepackage{titlesec}
\setlength{\parindent}{0.0in}
\setlength{\parskip}{1.0ex plus0.2ex minus0.2ex}
\renewcommand{\baselinestretch}{1.1}
\graphicspath{{eps/}{pdf/}}
\setcaptionmargin{0.25in}
\def\captionfont{\itshape\small}
\def\captionlabelfont{\upshape\small}

\renewcommand{\labelenumi}{(\roman{enumi})}

\newcommand{\bqq}{\begin{equation}}
\newcommand{\eqq}{\end{equation}}
\newcommand{\bqs}{\begin{equation*}}
\newcommand{\eqs}{\end{equation*}}

\newcommand{\Ral}{\mathcal{R}}


\newcommand{\C}{\mathbb{C}}
\newcommand{\D}{\mathbb{D}}
\newcommand{\N}{\mathbb{N}}
\newcommand{\R}{\mathbb{R}} 
\newcommand{\Z}{\mathbb{Z}}

\newcommand{\rme}{\mathrm{e}}
\newcommand{\rmi}{\mathrm{i}}
\newcommand{\rmd}{\mathrm{d}}
\newcommand{\rmo}{{\scriptstyle\mathcal{O}}}
\newcommand{\rmO}{\mathcal{O}}
\newcommand{\eps}{\varepsilon}
\newcommand{\lar}{ \lesssim }


\newcommand{\Rho}{\bm{\rho}}
\newcommand{\bigma}{\bm{\sigma}}
\newcommand{\diag}{\operatorname{diag}}
\newcommand{\supp}{\operatorname{supp}}

\renewcommand{\qedsymbol}{$\blacksquare$}


\numberwithin{equation}{section}

\newenvironment{Hypothesis}[1]%
  {\begin{trivlist}\item[]{\bf Hypothesis #1 }\em}{\end{trivlist}}

\renewcommand{\arraystretch}{1.25}


% Define Theorem Styles ----------------------------------
\theoremstyle{plain}
\newtheorem{theorem}{Theorem}[section]
\newtheorem{proposition}[theorem]{Proposition}
\newtheorem{lemma}[theorem]{Lemma}
\newtheorem{corollary}[theorem]{Corollary}
\newtheorem{conjecture}[theorem]{Conjecture}
\newtheorem{main}[theorem]{Main Result}
\newtheorem{rmk}[theorem]{rmk}


\newcommand{\etal}{\textit{et al.}\ }

\newcommand{\greg}[1]{%
  {\color{blue}\textbf{Greg:} #1}%
 }
 
\newcommand{\arnd}[1]{%
  {\color{red}\textbf{Arnd:} #1}%
 }

\newenvironment{Proof}[1][.]%
 {\begin{trivlist}\item[]\textbf{Proof#1 }}%
 {\hspace*{\fill}$\rule{0.3\baselineskip}{0.35\baselineskip}$\end{trivlist}}

\renewcommand\labelitemi{$\bullet$}

\title{Passage through a fold without a phase space}
\author{author}
\date{2016}
\begin{document}
\begin{center}

{\fontsize{17}{17}\fontfamily{cmr}\fontseries{b}\selectfont{Alternatives to the blow-up method in singular perturbation problems}}\\[0.2in]
Arnd Scheel and Tianyu Tao\\
\textit{\footnotesize 
University of Minnesota, School of Mathematics,   206 Church St. S.E., Minneapolis, MN 55455, USA}
\date{\small \today} 
\vspace*{0.2in}
\end{center}

\begin{abstract}
\noindent 
We show how an functional-analytic based method could serve as an alternative to the method of blow-up used to deal with fold-points in singular-perturbation problems. 
\end{abstract}

\section{Introduction}
We consider singularly perturbed ordinary differential equaitons (ODEs) of the form

\begin{align}\label{intro_slow}
\begin{split}
\eps \dot{x} &=  f(x,y;\eps),\\
\dot{y } &=   g(x,y;\eps),   
\end{split}\hspace{0.2in} x\in \mathbb{R}^n, \hspace{0.2in} y \in \mathbb{R}^m, \hspace{0.2in} 0<\eps \ll 1,
\end{align}
where $f$ and $g$ are $C^k$ functions for $k>=3$.

The standard way of studying \eqref{intro_slow} is using methods from geometric singular perturbation theory. An breif overview of this theory invloves treating system \eqref{intro_slow}, which is refered to as the \textit{slow-system}, together with its  equivalent counterpart, the \textit{fast-system}:
\begin{align}\label{intro_fast}
\begin{split}
x' &=  f(x,y;\eps),\\
y' &=  \eps g(x,y;\eps),   
\end{split}
\end{align}

where if $\tau$ denotes the (slow) time varibale in system \eqref{intro_slow}, then $t = \tau/\eps$ is the (fast) time varible used in system \eqref{intro_fast}. The dynamics can then be studied by setting $\eps = 0$ in both systems to obtain the so-called \textit{reduced problem}
\begin{align}\label{intro_reduced}
\begin{split}
0 &=  f(x,y,0),\\
\dot{y} &=   g(x,y,0),   
\end{split}
\end{align}

and the \textit{layer problem}
\begin{align}\label{intro_layer}
\begin{split}
x' &=  f(x,y,0),\\
y' &=  0. 
\end{split}
\end{align}
The basic premise of the theory, which was first laid out by Fenichel, is that the dynamics of reduced problem \eqref{intro_reduced} happens on the \textit{critical manifold}
\[
S:=  \{ (x,y) \mid f(x,y;0) = 0 \},
\]
one then switches to a \textit{normally hyperbolic submanifold} of equlibra $S_0 \subset S$ of the layer problem \eqref{intro_layer}, which will persist to $S_\eps$ for $0<\eps \ll 1$. Then, the dynamics of \eqref{intro_slow} is an $\eps$-perturbation of the reduced problem \eqref{intro_reduced}. Moreover, there exist a stable and an unstable invariant foliation with base $S_\eps$ with the dynamics along each foliation being a small perturbation of the suitable restriction of the dynamics of \eqref{intro_layer}.

The above approach relies heavily on the notion of normal hyperbolicity, which may not be always satisfied in the problems to be studied. The most common case is the so-called \textit{fold point}, where the critical manifold $S$ loses its normal hyperbolicity near bifurcation points due to a zero eigenvalue in the Jacobian $\frac{\partial f }{\partial x}$.

To overcome these difficuties, Krupa and Szmolyan proposed the method of \textit{blow-up} to extend the reach of geometric singluar perurbation theory. Roughly speaking, it is a set of coordinate transfomations which desingularizes the vector field near the fold point so that infomation can be gained by using standard tools in dynamical system.


Their example was the following planar system 
\begin{align}\label{ori_eqn}
\begin{split}
u' &= \mu+u^2+ f(u,\mu; \eps),\\
\mu' &=  \eps g(u,\mu; \eps),
\end{split}
\end{align}
where $f(u,\mu;\eps) = \rmO(\eps, u\mu,\mu^2,u^3)$ and $g(u,\mu;\eps) = 1+\rmO(u,\mu,\eps)$. 

For $u,\mu$ sufficiently small, the critical manifold $S_0 = \{ (\mu, u) \mid \mu + u^2 +f (u,\mu ;0) = 0\}$ resembles the parabola $\mu = - u^2$. Away from the fold point $(0,0)$, there exist the attracing manifold $M_a$ with a section $S_a$ sketched in figure (????), from Fenichel's theory, $S_a$ perturbes to $S_{a,\eps}$ until it reaches the fold point, thus a natural question is to track how the slow manifold $S_{a,\eps}$ passes through the fold point.

Krupa and Szmoylyan starts by setting two sections $\Delta^{in} = \{(-\delta, u) \mid u\in J,\text{ $J$ is some small interval }\}$ and $\Delta^{out} = \{( \mu ,\sqrt{\delta})\mid \mu \in \mathbb{R})\}$ with $\delta>0$ small but fixed, which are also shown in figure (????). To track the passage through the fold amounts to study the transition map 
\[
\pi: \Delta^{in} \to \Delta^{out},
\]

Krupa and Szmoylyan then proceeds by defining the blow up transformation
\[
\mu = \bar{r}^2 \bar{\mu}, \quad u =\bar{r} \bar{u}  \quad, \eps = \bar{r}^3 \bar{\eps},
\]
which ``blows up'' the vector field of the extended $(\mu, u, \eps)$ system near $(0,0,0)$ into a vector field on the ball $B = S^2 \times [0,\eps_0^{1/3}] \ni (\bar{\mu}, \bar{u}, \bar{\eps}, \bar{r})$ for some $\eps_0>0$ small, and further directional blow-ups to obtain charts $K_1,K_2,K_3$ which was used to derscribe the flows in regions near different parts of the manifold $B$. After a careful and thorough analysis, they were able to prove the following results:
\begin{theorem}\label{ks_main}
For $\eps$ sufficiently small, the following statements are true:
\begin{enumerate}
\item The manifold $S_{a,\eps} $ passes through  $\Delta^{out}$ at a point $(h(\eps), \sqrt{\delta})$ where $h(\eps) = \rmO(\eps^{2/3})$.
\item The transition map $\pi$ is a contraction with rate $\rmO(e^{-c/\eps} )$, where $c$ is a positive constant.
\end{enumerate}
\end{theorem}


In this paper, we focus on the same problem \eqref{ori_eqn} with an aim to recreate the result of Theorem \ref{ks_main} via a different, functional analytically-based approach. Instead of proceeding with the geometrically-flavoured blow-up approach, we directly prove a tractory exists with the properties claimed in Theorem \ref{ks_main} by dividing the time of passage into apporpriate parts and setting up close ansatz in each region, closing the arguments with carefully reformulating the existence question as a fixed-point argument.

\paragraph{Outline}
The reminder of the paper is organized as follows. In Section \ref{sec_main} we give a precise statement of our result, as well as an overview of our set ups. In Sections \ref{sec_A}, \ref{sec_B} and \ref{sec_C} we construct the ansatz mentioned earlier and in Section \ref{sec_glue} we show how to piece together all the parts to arrive at our main results.


\paragraph{Notation}
We use $A \lar B$ to indicate that there is a constant $C$ such that $A \le C \cdot B$, independent of the properties of $A$ and $B$.



\section{Main Result}\label{sec_main}
We first make a change of variable to transform equation \eqref{ori_eqn} into a more covenient form in section \ref{euler_m}. Then we introduces the different ansatzs used in section \ref{ansatz_divide} and explain how we divide up the passage time into 3 different regions $A$, $B$, and $C$. Finally we give a statement of our main result in Section \ref{main_sum}. 

\subsection{Euler multiplier}\label{euler_m}
Let $\tau$ denote the independent time variable in \eqref{ori_eqn}, since for $u,\mu,\eps$ small, $g(u,\mu,\eps) = 1 + \rmO(u,\mu,\eps)>0$, we can define a new time $t = t(\tau)$ by $\frac{dt}{d\tau} = g$, consequently, equation \eqref{ori_eqn}  is transformed into
\begin{align}\label{euler_ori_eqn}
\begin{split}
\frac{d}{dt}u &= \mu+u^2+ \tilde{f}(u,\mu;\eps),\\
\frac{d}{dt}\mu &=  \eps ,
\end{split}
\end{align}
where now $\tilde{f}(u,\mu,\eps) = \rmO(\eps,  u\mu, \mu^2,\eps u, \eps \mu, \eps u^2, u^3)$.  The critical manifold 
\[
\tilde{S}_0 = \{ (\mu, u) \mid \mu + u^2 + \tilde{f}(u,\mu;\eps) = 0\},
\]
still has $(0,0)$ as the fold point, and our goal is to track how a trajectory of the flow of \eqref{euler_ori_eqn} pass through $(0,0)$. Following the set up of the sections $\Delta^{in}$ and $\Delta^{out}$ in Krupa-Szmolyan, we propose the following boundary conditions

\begin{equation}\label{ori_bc}
\mu(0) = -\delta, \hspace{0.2in} u(T) = \delta,
\end{equation}
where $T$ is also an unknow variable which marks the ``time of exit'' as the trajctory hits the section $\Delta^{out}$.

That is, we think the tracking of the passage of the fold as a shooting problem, if we can prove the existence of a solution $(\mu, u)$ to system \eqref{euler_ori_eqn} with the bondary condition \eqref{ori_bc}, and give the expansion for the components $\mu$, then we will prove the results in \eqref{ori_eqn}. Our contribution in this paper is that the method we used to prove the existence of such a solution is completely different than the blow-up approach.

In the rest of the paper, we shall drop the tilde to use $f(u,\mu,\eps)$ as the nonlinearity and $S_0$ to denote the critical manifold.
 
\subsection{Ansatz and dividing the time of passage}\label{ansatz_divide}


\iffalse
\subsection{Equation}
For demonstration purposes, we focus on the following equation first
\begin{align}\label{euler_ori_eqn}
\begin{split}
\frac{d}{dt}u &= \mu+u^2+ u^3,\\
\frac{d}{dt}\mu &=  \eps ,
\end{split}
\end{align}
with boundary condition \ref{ori_bc}.
\fi

To start, we want to get a rough idea of how long it takes from $t=0$ to $t=T$, as well as what forms of ansatz one should use. The primary motivation is the when equation \eqref{ori_eqn} reduces to the simple case with $f(u,\mu;\eps) = 0$, also known as the Riccati equation.

\subsubsection{The Riccati equation}
Consider the Riccati equation
\begin{equation}\label{ric}
\frac{d}{ds}u(s) = s+u(s)^2,
\end{equation}

We denote any solution to \eqref{ric} as $u_R$, it is known to have a unique solution (denoted by $\bar{u}_R$) with the following asymptotics:

\begin{equation} \label{ric_asy}
\bar{u}_R(s)=\begin{cases}
  (\Omega_0-s)^{-1}+\rmO(\Omega_0-s), \text{ as }s \to \Omega_0, \\
 -(-s)^{1/2} -\frac{1}{4}(-s)^{-1} + \rmO(|s|^{-3/2}), \text{ as }s \to -\infty.
\end{cases}
\end{equation}

Here the constant $\Omega_0 \approx 2.3381$ is the smallest positive zero of 
\[
J_{-1/3}(2z^{3/2}/3)+J_{1/3}(2z^{3/2}/3),
\]
where $J_{\pm 1/3}$ are Bessel functions of the first kind. (See [Krupa, Szmolyan])


More generally, we consider family of solutions  $u_R(s; u_0)$ of the Riccati equation  \eqref{ric} such that $u_R(0; u_0) = u_0$. That is, we take the initial condition $u_0$ as a parameter to the Riccati equation. For the special Riccati solution $\bar{u}_R$, we denote $\bar{u}_R(0) $ as $\bar{u}_0$. In fact, using simple phase plane analysis, we can extend the asymptotic results about the special Riccati solution $\bar{u}_R$, \eqref{ric_asy} to the $u_0$-parameter dependent family $u_R(s; u_0)$, the precise statement can be found in the appendix.

\subsubsection{Critical manifold}\label{c_mfld}
Another piece of the ansatz comes from part of  he critical manifold, we expect this because away from the fold, the critical manifold has an attacting branch $S_a$ which implies tracjectory has to stay close to it. 

Recall the critical manifold for system \eqref{crit_mfld} is the set of points $(\mu, u) \in \mathbb{R}^2$ which satisfies
\begin{equation} \label{crit_mfld}
\mu + u^2 + f(u,\mu; 0) =  0.
\end{equation}

By the implicit function theorem, whenever $\partial_u( u^2+f(u,\mu;0) ) \neq 0$, we can write $u=u(\mu),$ for $\mu$ small. For those $(\mu, u)$ such that $\partial_u( u^2+f(u,\mu;0) )<0$ (which corresponds to the attracting branch $S_a$), we define $u_s(t):= u(\mu(t))$ so that
\begin{equation}\label{singular}
0 = \mu(t) + u_s^2(t)+f(u_s(t),\mu; 0),
\end{equation}
for $t$ small.  

Due to the simple form of \eqref{euler_ori_eqn} and \eqref{ori_bc}, we have that $\mu(t)= \eps t-\delta$, the it can be calculated that $u_s(\mu)$ has the following asymptotics
\begin{align}\label{singularAsy}
\begin{split}
u(\mu) &= -\sqrt{\mu} + \rmO(\mu)\\
u_s(t) &= -\sqrt{\delta-\eps t} + \rmO(|\delta-\eps t|).
\end{split}
\end{align}


\subsubsection{Division of regions and the rescale of time}\label{t_sigma}

Having introduced the primary ansatz, the next step is to ``glue'' them together. Since these anstaz have quite different asymptotics on different time, it motivates us to divide the total time of of passage, from $t=0$ to $t=T$, and to choose the particular form of the ansatz, as follows:

Going backwards in time, we start with the exit time $t=T$ when the trajectory hits the section $\Delta^{out}$, we shall see that $T$ roughly has following asmpytotics
\begin{equation}
T = \eps^{-1}\delta + \rmO(\eps^{-1/3}),
\end{equation}
this $T$ also marks the rightmost boundary of the region $A$, where our ansatz takes the form
\[
u_A = u_*(t)  + w_r(t),
\]
here the function $u_* = u_*(t; u_0)$ is defined as:
\begin{equation}\label{urdef}
u_*(t; u_0) := \eps^{1/3}u_R(\eps^{1/3}(t-\eps^{-1}\delta); u_0),
\end{equation}

where $u_R=u_R(s; u_0)$ is the family of solutions to the Riccati equation which were shown to exist in proposition \ref{para_ric}, it solves the initial value problem
\begin{equation}\label{ureq}
\frac{d}{dt}u_*(t; u_0) = \mu(t) + u_*^2(t; u_0), \hspace{0.2in} u_*(\eps^{-1}\delta; u_0) = \eps^{\frac{1}{3}} u_0,
\end{equation}
with $\eps$ and $u_0$ as parameters.  The function $w_r$ is a correction term whose properties will be given in later sections. 

In region $A$, we think the main part of the ansatz, $u_*(t)$, which is merely a rescaled version of the Ricatti solution $u_R(s; u_0)$, follows the first half of the asymptotics in Proposition \ref{para_ric}, when $u_*(t)$ start to be controlled by the other part of the asymptotics, we will need to adaptively change our anstaz or function space in order to ``glue'' the solutions, an intuitive, but rather arbitrary place to switch regions is at $t = \eps^{-1} \delta$. This marks the start of region $B$, where our choice of ansatz is as follows:
 The ansatz takes the form 
 \[ 
 u_B(t) = \bar{u}_*(t)  +w_\ell(t).
\]

Where $\bar{u}_*$ is the function
\begin{equation}\label{uldef}
\bar{u}_*(t) = u_*(t;\bar{u}_0) = \eps^{1/3} u_R( \eps^{1/3}(t-\eps^{-1}\delta); \bar{u}_0)=\eps^{1/3}\bar{u}_R(\eps^{1/3}(t-\eps^{-1}\delta)),
\end{equation}
and $\bar{u}_*$ solves the equation
\begin{equation}\label{uleq}
\frac{d}{dt}\bar{u}_* (t) = \mu(t) + \bar{u}_*^2(t),
\end{equation}
so similarly $\bar{u}_*$ is a rescaled version of the special solution to the Riccati equation. Also similar to the situation in region $A$, $w_\ell$ is a correction term whose properties will be described later.

It might seem that there are little differences between the ansatz $u_A$ and $u_B$, however, as it will be demonstrated in the proof, the main  motivation for such a choice is to ensure the fixed point argument goes through with the choice of function spaces motivated by the different asymptotics of the ansatz.

The next piece of the anstaz will be used to connect the piece $u_B$, which roughly follows the special Ricatti solution $\bar{u}_R$, to the attracting branch $S_a$ of the critical manifold $S_0$, a natural ``gluing point'' is at where the error between $\bar{u}_R$ and $u_s(t)$, is small. Calculation shows that this is at a point $t=t^*$ where we roughly have
\begin{equation}
(\delta - \eps t^*) \approx -\eps^{1/2},
\end{equation} 

hence we choose $t^*$ as a natural transition point from region $B$ to the last region, region $C$, where it covers the rest of the passage time until at $t=0$. The corresponding ansatz will take the form
\[
u_C(t) = u_s(t) + w_s(t),
\]
the $w_s(t)$ is yet another correction term whose properties will be discussed later. 

In summary, region $A$, $B$ and $C$ are divided as follows:
\begin{align}\label{region_division_t}
\begin{split}
\text{Region A:} & \quad (\eps^{-1}\delta, T), \quad \text{ansatz in region A:} \quad u_A(t) = u_*(t)+w_r(t), \\
\text{Region B:} & \quad (t^*, \eps^{-1}\delta), \quad \text{ansatz in region B:} \quad u_B(t) = \bar{u}_*(t)+w_\ell(t),  \\
\text{Region C:} & \quad (0, t^*), \quad \text{ansatz in region C:} \quad u_A(t) = u_*(t)+w_s(t), .
\end{split}
\end{align}

Now we can breifly describe the strategy of our proof, we plug in the ansatz into Equation \eqref{euler_ori_eqn} to derive the equation for the ansatz $w_r, w_\ell $ and $w_s$, choose approprate function space with norms and set up the equations for ansatz as fixed point argument on these function spaces. A main technical part of our proof consists of appropriately rescale the time $t \in (0,T)$ so that we gain hyperbolicity in the sense that the linearized operator at the anstaz become Fredholm in the new time scale. This is the key observation in our approach, comparable to the blow-up approach which also gains hyperbolicity via a carefully chosen change of variable. Having solved the correction term $w_r, w_\ell$ and $w_s$, we can then collect information about their asymptotic expansion to confirm the corresponding solution has the properties we need.


Therefore, we need to describe how we transform time $t$ into time $\sigma$ which gains hyperbolicity. We do this using the following steps.
\begin{enumerate}
\item Step 1: Define $\psi$ as
\[
\psi = \eps^{1/3}(t - \eps^{-1}\delta),
\]

\item Step 2:
Fix $M>0$ large, define $\sigma$ as
\begin{align*}
\psi = \psi(\sigma; u_0) =\begin{cases}
-(-\frac{3}{2} \sigma)^{2/3} , \text{ for }\sigma \le -M, \\
\Omega_\infty(u_0) -e^{-\sigma}, \text{ for }\sigma \ge M,
\end{cases}
\end{align*}
here $\Omega_\infty$ is the blow-up time for $u_R$ found in proposition \ref{para_ric}.

\item Step 3: For $\sigma \in (-M, M)$, we define $\psi(\sigma)$ as the straight line connecting the two points $(M, \Omega_\infty-e^{-M})$ and $(-M, -(\frac{3}{2}M)^{2/3})$. As a result, if we define $\sigma_m=\sigma_m(u_0)$ as the value of $\sigma$ such that $\psi(\sigma_m; u_0) = 0$, then we have 
\[
\frac{|\sigma_m - M|}{M} = \left| \frac{(\frac{3M}{2})^{2/3}-(\Omega_\infty-e^{-M})}{(\frac{3M}{2})^{2/3}-(\Omega_\infty+e^{-M})} -1 \right|\le CM^{-2/3},
\] 
for some constant $C$ independent of $u_0$.

Therefore we can write
\begin{equation}\label{sigm_asy}
\sigma_m = M + M_r, \hspace{0.2in} |M_r| \le CM^{1/3}
\end{equation}
\end{enumerate}


We also denote $\varphi(\sigma) := \frac{d}{d\sigma}\psi(\sigma)$, which save us some space when writing the factor for the change of variable between $\sigma $ and $t$. 

Therefore, the original exit time $t=T$ will be transformed into $\sigma=\sigma_T$ under this change of variable, and similarly the boundary between region $A$ and region $B$ is  transformed from $t=\eps^{-1}\delta$ to $\sigma = \sigma_m$ and the boundary between region $B$ and region $C$ is  transformed from $t=t^*$ to $\sigma = \sigma^*$, and $t=0$ into $\sigma = \sigma_0$. Later sections will show that the $\sigma_0, \sigma^*, \sigma_m, \sigma_T$ satisfies:
\[
\sigma _0 \approx -\delta^{3/2}\eps^{-1}, \quad \sigma ^* \approx -\eps^{-1/4}, \quad \sigma_m = \rmO_\eps(1), \quad
\sigma_T \approx -\log(\eps^{1/3}\delta^{-1}) 
\]
In summary, the regions in time $\sigma$ are as follows:
\begin{align}\label{region_division_sig}
\begin{split}
\text{Region A:} & \quad (\sigma_m ,\sigma_T ),  \\
\text{Region B:} & \quad (\sigma^*, \sigma_m),  \\
\text{Region C:} & \quad (\sigma_0, \sigma^*),
\end{split}
\end{align}
and the following picture summrizes the relationship between the two time scales $t$ and $\sigma$, as well as the corresponding regions divided.


\iffalse
%\item Asymptotics for $u_R$ and $\varphi$.
\begin{equation*}
\varphi(\sigma) =\begin{cases}
 (-\frac{3}{2}\sigma)^{-1/3}, \text{ as }\sigma \to -\infty\\
e^{-\sigma} , \text{ as }\sigma \to \infty.
\end{cases}
\end{equation*}

\begin{equation*}
u_R(\psi(\sigma)) \to \begin{cases}
 -(-\frac{3}{2}\sigma)^{1/3}, \text{ as }\sigma \to -\infty\\
e^{\sigma} , \text{ as }\sigma \to \infty.
\end{cases}
\end{equation*}

\begin{equation*}
2u_R\varphi(\sigma) \to\begin{cases}
-2+ \rmO((-\sigma)^{-3/2}), \text{ as }\sigma \to -\infty\\
2+ \rmO(e^{-2\sigma}), \text{ as }\sigma \to \infty.
\end{cases}
\end{equation*}
%
\fi
\subsection{Main result - summary} \label{main_sum}

\begin{theorem}[Gluing of the solutions]
For each $\delta>0$ and $\eta>0$ small, there exist $\eps_0$ and $u_0$ such that for all $0<\eps<\eps_0$ and $|u_0 - \bar{u}_0|<\eta$, a solution of equation \ref{euler_ori_eqn} exists with $\mu(t) = \eps t - \delta$ and 
\begin{align*}
u(t; u_0)  =\begin{cases}
u_A(t;u_0) , \text{ for }t \in (\eps^{-1}\delta, T),\\
u_B(t), \text{ for } t \in (t^*,\eps^{-1}\delta), \\
u_C(t), \text{ for } t \in (0, t^*),
\end{cases}
\end{align*}
where $u_A(t; u_0)$, $u_B(t)$ and $u_C(t)$ are the solutions that were shown to exist in Theorem \ref{thm_r}, \ref{thm_l} and \ref{thm_s}.

Moreover, $u$ satisfies 
\[
u(T;u_0) = \delta, \hspace{0.2in} u(0;u_0)  = -\sqrt{\delta} +\delta + \rmO( \eps^{1-\alpha/3}\delta^{-1}).
\]
\end{theorem}

\pagebreak



\section{Region A}\label{sec_A}

Region A corresponds to the $t$-time interval $\{ t : t > \eps^{-1}\delta\}$.

\subsection{Ansatz in region A}

\begin{theorem}\label{thm_r}
For all $\delta, \alpha$ small enough, there exists $\eta,\eps_1,C$, such that for all $0<\eps <\eps_1$, and all $|u_0 - \bar{u}_0|<\eta$, there exist a time $T=T(\eps;u_0)$ and a solution to \eqref{ori_eqn} of the form
\[
u_A(t;u_0) = u_*(t; u_0) + w_r(t; u_0),
\]
exists on the time interval $t \in (\eps^{-1}\delta, T)$, such that $w_r$ and $T$ satisfies
\begin{enumerate}[label=\textnormal{(\arabic*)}]
\item \label{thm_r_1}$T=T(\eps;u_0) = \eps^{-1}\delta+\eps^{-1/3}\Omega_\infty(u_0)-\delta^{-1}+T_r$ with $|T_r|\le C\eps^{2/3}\delta^{-3}$,
\item \label{thm_r_2} $w_r(T; u_0) = 0$ and $u_*(T,u_0)=\delta$,
\item \label{thm_r_3} $|w_r(\eps^{-1}\delta; u_0)| \le C\eps^{(2-\alpha)/3}$,

\item \label{thm_r_4} $\sup_{t} |(T_\infty-t)^{2-\alpha} w_r(t; u_0)| \le C,$

\item \label{thm_r_5} The function $w_r(\eps^{-1}\delta; \cdot)$ is smooth, with Lipschitz constant  $|\text{Lip }w_r(\eps^{-1}\delta; \cdot) |\le C\eps^{2/3} . $

\end{enumerate}

\end{theorem}

We will prove this theorem in the following sections.
\subsection{The exit time \texorpdfstring{$T(u_0)$}{T(u_0)}}\label{exit_time}
The exit time $T$ is defined by the boundary condition 
\[
\delta = u_*(T; u_0) = \eps^{1/3}u_R(\eps^{1/3}(T-\eps^{-1}\delta); u_0),
\]
since the expansion for $u_R$ is given in \eqref{ric_exp} , if we define $s_T = \eps^{1/3}(T-\eps^{-1}\delta)$, then $s_T$ satisfies
\[
\frac{1}{\Omega_\infty-s_T} + (\Omega_\infty-s_T)r(\Omega_\infty-s_T) = \eps^{-1/3}\delta,
\]
from which we get the leading order expansion $\Omega_\infty-s_T = \rmO(\eps^{1/3}\delta^{-1})$. A fixed point argument gives
\[
\Omega_\infty - s_T = \eps^{1/3}\delta^{-1} + \rmO(\eps \delta^{-3}),
\]
hence the expansion for $T=T(\eps; u_0)$ is
\begin{equation}\label{T_exp}
T = T(\eps;u_0) = \eps^{-1}\delta + \eps^{-1/3}\Omega_\infty(u_0) - \delta^{-1} + T_r,
\end{equation}
with 
$|T_r|\le C\eps^{2/3}\delta^{-3}$, for some constant $C$ independent of $u_0$, as $\eps \to 0$.

For conveneience, we define \[
T_\infty = T_\infty(\eps; u_0)= \eps^{-1}\delta + \eps^{-1/3}\Omega_\infty(u_0),
\]
so that $T = T_\infty - \delta^{-1} + T_r$.

\subsection{Equation for \texorpdfstring{$w_r$}{wr} and rescaling}\label{equation_wr}
We now plug in the anstaz $u = u_* + w_r$ into equation \eqref{ori_eqn}, and derive the equation for $w_r$
\begin{align}\label{eqn_wr}
\begin{split}
w_r' - 2u_*w_r &= w_r^2 + f(u_*+w_r, \mu; \eps) := R_r(w_r; \eps,u_0),
\end{split}
\end{align}

moreover, we enforce the boundary condition $u(T; u_0) = \delta$, hence this gives the boundary condition for $w_r$ at $t=T$:
\begin{equation}\label{Bc_w_r}
w_r(T;u_0) = 0,
\end{equation}


therefore, equation \eqref{eqn_wr} is posed on the interval $t \in (\eps^{-1}\delta, T)$, with boundary condition \eqref{Bc_w_r}.

Next, we rescale equation \eqref{eqn_wr} into $\sigma$-time variable by using the $t$ to $\sigma$-time rescaling in section \ref{t_sigma}, and obtain
\begin{equation}\label{rescl_wr}
\left(\frac{d}{d\sigma} - a(\sigma; \eps, u_0)\right) W_r =\eps^{-1/3}\varphi \mathcal{R}_r(W_r; \eps,u_0),
\end{equation}
where
\begin{itemize}
\item The term $a(\sigma;\eps, u_0)$ is defined as and has asymptotics
\[
a(\sigma; \eps, u_0) := 2\varphi(\sigma)u_R(\psi(\sigma); u_0) =  2+\rmO(e^{-2\sigma}) \text{ as }\sigma \to \infty,
\]
we remark that this convergence as $\sigma \to \infty$ is uniform in $u_0$ due to the definition of our time-rescaling.

\item The function $W_r(\sigma)$ is the rescaled version of $w_r(t)$ in the $\sigma$-variable, $w_r(t) = w_r(\eps^{-1/3}\psi(\sigma)+\eps^{-1}\delta) = W_r(\sigma)$. $U_*$ is similarly the rescaled version of $u_*$,  $U_*(\sigma;u_0)= u_*(t;u_0) = \eps^{1/3}u_R(\psi(\sigma;u_0);u_0)$.

\item The function $\Ral_r$ is a rescaled version of $R_r$ such that 
\[
\mathcal{R}_r(W_r;\eps,u_0) = W_r^2 + f(U_*+W_r, \mu ; \eps),
\]
\end{itemize}
 

To get the corresponding boundary condition of \ref{Bc_w_r}, we need to know the corresponding $\sigma$-time for the $t$-time interval $t\in (\eps^{-1}\delta, T)$.

At $t = \eps^{-1}\delta$, the corresponding $\sigma$ time is at $\sigma=\sigma_m$, from its definition in section \ref{t_sigma}.

At $t=T$, we have $\eps^{1/3}(T-\eps^{-1}\delta) = \Omega_\infty -\eps^{1/3}\delta^{-1}+\eps^{1/3}T_r=\psi(\sigma_T) = \Omega_\infty-e^{-\sigma_T}$ from \eqref{T_exp}, hence, for $\eps$ small enough, we get that the corresponding $\sigma$-time to $t=T$ is 
\begin{equation}\label{def_sigm_T}
\sigma_T=\sigma_T(u_0)= -\log(\eps^{1/3}(\delta^{-1}-T_r)) = -\log(\eps^{1/3}\delta^{-1}) - \log(1-\delta T_r),
\end{equation}
Then, the complete boundary value problem we wish to solve is
\begin{align}\label{wr_bp}
\begin{split}
\frac{d}{d\sigma} W_r - a(\sigma;\eps,u_0)W_r &= \eps^{-1/3}\varphi \Ral_r(W_r), \text{ } \sigma \in (\sigma_m, \sigma_T),\\
W_r(\sigma_T) &= 0.
\end{split}
\end{align}




\subsection{Linear equation and norms}




Our goal now is to solve \eqref{wr_bp} on an appropriate function space, to do so we first slightly enlarge the time interval $(\sigma_m, \sigma_T)$ where the boundary value problem is posed.

From the definition of $\sigma_T$ \eqref{def_sigm_T}, we see that
\begin{equation}\label{sigm_T_asy}
|\sigma_T -(-\log(\eps^{1/3}\delta^{-1}))| \le |\log(1-\delta T_r)| \le  C|\delta T_r|\le C\eps^{2/3}\delta^{-2},
\end{equation} 
for some constant $C$ independent of $u_0$.

We now define $\sigma_{\inf}$ and $\sigma_{\sup}$ as follows:
\[
\sigma_{\inf} = \inf_{|u_0-\bar{u}_0|<\eta}\sigma_m(u_0), \hspace{0.3in} \sigma_{\sup} = \sup_{|u_0-\bar{u}_0|<\eta}\sigma_T(u_0).
\]
From the definition of $\sigma_m$ in section \ref{t_sigma} and \eqref{sigm_T_asy} we have
\[
M \approx\sigma_{\inf} , \hspace{0.2in} \sigma_{\sup} \approx -\log(\eps^{1/3}\delta^{-1}),
\] with the error indepedent of $u_0$. 
Therefore we introduce the function space below:
\begin{align*}
\mathcal{C}_{r} &= \left\{ w(\sigma) : \sup_{\sigma_{\sup}\ge \sigma\ge \sigma_{\inf}} \left|\eps^{(\alpha-2)/3} e^{(\alpha-2)\sigma}w(\sigma)\right| < \infty \right\}. \\
%--------------------------------------
\end{align*}

We establish the invertibility of the linear operator $A_r$ which acts on $w \in \mathcal{C}_r$ as 
\[
A_r w = \left( \frac{d}{d\sigma}w-a(\sigma;\eps, u_0) w,  w(\sigma_T)\right),
\]
in the following
\begin{proposition}\label{inv_A_r}
$A_r=A_r(u_0,\eps) : \mathcal{D}\subset \mathcal{C}_r \to \mathcal{C}_r\times \mathbb{R}$ and is invertible, with its inverse smoothly depends and uniformly bounded in $u_0, \eps$. 
\end{proposition}
\begin{proof}
Consider the conjugate opertor of $A_r$, given by
\begin{align*}
\tilde{A}_r v &= \left( e^{(\alpha-2)\sigma}\left(\frac{d}{d\sigma}-a(\sigma;\eps,u_0)\right)e^{(2-\alpha)\sigma} v, v(\sigma_T) \right) \\
&= \left( \frac{d}{d\sigma}v -(\alpha+a(\sigma;\eps,u_0)-2)v, v(\sigma_T) \right),
\end{align*}
for $v(\sigma)=\eps^{(\alpha-2)/3}e^{(\alpha-2)\sigma}w(\sigma) \in \mathcal{C}([\sigma_{\inf}, \sigma_{\sup}])$. 

The associated conjugate equation of 
\[
A_r w = (f,w_T) \text{ with } f \in \mathcal{C}_r, w_T \in \mathbb{R},
\] is 
\[
\tilde{A}_r v = (\eps^{(\alpha-2)/3}e^{(\alpha-2)\sigma} f,v_T) \text{ with } \eps^{(\alpha-2)/3}e^{(\alpha-2)\sigma} f \in C , v_T = \eps^{(\alpha-2)/3} e^{(\alpha-2)\sigma_T}w_T.
\]
 
Since $\alpha > 0$ and $|a(\sigma;\eps,u_0) -2| = \rmO(e^{-2\sigma}) \to 0$ as $\sigma \to \infty$ uniformly in $\eps, u_0$. We apply lemma \ref{lin_bv} with $L= \sigma_T$ to conclude that there exists a constant $C$ independent of $\eps, u_0$ with 
\[
\|w\|_{\mathcal{C}_r} = |v|_\infty \le C(|\eps^{(\alpha-2)/3}e^{(\alpha-2)\sigma} f |_{\infty}+|v_T|) \le C(\|f\|_{\mathcal{C}_r}+|w_T|),
\]
notice $|v_T| \le w_T$ by the asymptotics of $\sigma_T  = \rmO(-\log(\eps^{1/3}))$. By the definition of $\sigma_{\inf}$ and $\sigma_{\sup}$, $A_r$ is uniformly invertible in $u_0$ on $\mathcal{C}_r$, this finishes the proof of the proposition.
\end{proof}

\iffalse
\begin{proof}
Fix $\eta$ small, then there is $\eps_0>0$ such that for $\eps<\eps_0$, we can choose $\sigma_* \in (\sigma_{\inf}, \sigma_{\sup})$ so that $|a(\sigma; \eps, u_0)-2| \le \eta$ for $\sigma \ge \sigma_*$. As a result there is a constant $C$, independent of $\eps$ and $u_0$, such that $|\sigma_*-\sigma_{\inf}|\le C$. In fact, we can choose $\sigma_* \le \sigma_T$ since $|\sigma_T-\sigma_{\sup}| \le |\sigma_T - (-\log(\eps^{1/3}\delta^{-1})|\le C\eps^{2/3}\delta^{-2}$.

For $\sigma_{\sup} \ge \sigma \ge \sigma_*$, the linear operator $A_r$ is a small perturbation of the invertible linear operator $w \to \left(\frac{dw}{d\sigma}-2w, w(\sigma_T) \right)$. Indeed, for $f \in \mathcal{C}_r$, the equation $\frac{d}{d\sigma}w-2w = f$
has the solution
\[
w(\sigma) = e^{2(\sigma-\sigma_T)}w(\sigma_T)+\int_{\sigma_T}^\sigma e^{2(\sigma-s)}f(s)ds,
\]
moreover, it holds that $\|w\|_{\mathcal{C}_r} \le C(\delta^{\alpha-2}|w(\sigma_T)| + 2\alpha^{-1}\|f\|_{\mathcal{C}_r})$ for some constant $C$ independent of $u_0,\eps$, which shows the inverse is independent of $\eps$ and $u_0$. If the prescribed value at $w(\sigma_T)$ is of order $\delta^{2-\alpha}$.

For $\sigma < \sigma_*$, we can directly write down the solution of $(d/d\sigma - a)w = f$ as
\[
w(\sigma) = \exp\left(\int^{\sigma}_{\sigma_*} a(\tau)d\tau\right) w(\sigma_*) + \int_{\sigma_*}^{\sigma} \exp\left(-\int_{\sigma}^{s}a(\tau)d\tau\right)f(s)ds 
\]
 Again, the inverse is bounded uniformly in $\eps, u_0$ because of the bound $|\sigma_*-\sigma_{\inf}|\le C$. To see this, note $w(\sigma_*)$ can be evaluated using the solution on the intveral $(\sigma_*,\sigma_{\sup})$, which satisfies $w(\sigma_*) \lar \eps^{\frac{2-\alpha}{3}}e^{(2-\alpha)\sigma_*}$, so 
\[
\left\|\exp\left(\int^{\sigma}_{\sigma_*} a(\tau)d\tau\right) w(\sigma_*)\right\|_{\mathcal{C}_r} \le e^{(2-\alpha)(\sigma_*-\sigma)}\exp\left(\int^{\sigma}_{\sigma_*} a(\tau)d\tau\right) \lar e^{(2-\alpha)C},
\]
and
\[
\left\| \int_{\sigma_*}^{\sigma} \exp\left(-\int_{\sigma}^{s}a(\tau)d\tau\right)f(s)ds\right\|_{\mathcal{C}_r} \le \int_{\sigma_*}^{\sigma} \exp\left(-\int_{\sigma}^{s}a(\tau)d\tau\right)e^{(2-\alpha)(s-\sigma)} \|f\|_{\mathcal{C}_r}ds \lar e^{(2-\alpha)C}\|f\|_{\mathcal{C}_r}
\]
Combining the case $\sigma< \sigma_*$ and $\sigma \ge \sigma_*$ together, we conclude that $A_r$ is uniformly invertible in $\eps$ and $u_0$ on the space $\mathcal{C}_r$.
\end{proof}
\fi

\subsection{Nonlinear estimates}

In this section we estimate the nonlinear term
\[
\mathcal{R}_r(W_r) = W_r^2 + f(U_*+W_r, \mu ; \eps),
\]
in the $\mathcal{C}_r$ norm to prove 
\begin{proposition}\label{nl_est_r}
If $W_r \in \mathcal{C}_{r}$, then $\eps^{-1/3}\varphi \mathcal{R}_r(W_r) \in \mathcal{C}_{r}$, and
\begin{align}
\|\eps^{-1/3}\varphi \mathcal{R}_r \| = \rmO(\delta^{\alpha}).
\end{align}
\end{proposition}
\begin{proof}

Proposition \ref{ric_exp} shows
\begin{align*}
U_*(\sigma;u_0) =  \eps^{\frac{1}{3}}(e^\sigma+e^{-\sigma} r(e^{-\sigma}; u_0)   ) \text{ as }\sigma \to \infty,
\end{align*}
therefore 
\[
|U_*(\sigma)| \lar \eps^{\frac{1}{3}}e^\sigma \le \eps^{1/3}e^{\sigma_{\sup}} = \rmO(\delta) \text{ for all } \sigma \ge \sigma_{\inf}.
\]

From the definition of the time rescaling in section \ref{t_sigma} we have
\[
\mu = \eps t-\delta  =\eps^{2/3}\psi(\sigma) \lar \eps^{2/3}.
\]
As $W_r \in \mathcal{C}_r$, we have 
\[
|W_r(\sigma)| \lar \eps^{\frac{2-\alpha}{3}} e^{(2-\alpha)\sigma} \ll U_*(\sigma), \text{ for } \sigma \in (\sigma_{\inf}, \sigma_{\sup}).
\]

Using these facts, we have
\[
\|\eps^{-\frac{1}{3}}\varphi W_r^2\|_{\mathcal{C}_r}=\sup |\eps^{-\frac{1}{3}} \varphi W_r| \lar \eps^{\frac{1-\alpha}{3}} e^{(1-\alpha)\sigma} \lar \eps^{\frac{1-\alpha}{3}} e^{(1-\alpha)\sigma_{\sup}} \lar \delta^{1-\alpha},
\]

as $f(u,\mu; \eps) = \rmO(\eps(1+u+\mu+u^2),u\mu,\mu^2,u^3)$ and since $U_*,W_r,\mu$ are all small in sup norm, we have

\[
f(U_*+W_r, \mu ;\eps) = \rmO(\eps, (U_*+W_r)\mu, \mu^2, (U_*+W_r)^3 ) = \rmO(\eps, U_*\mu, \mu^2, U_*^3),
\]

hence
\[
\|\eps^{-\frac{1}{3}}\varphi \eps \|_{\mathcal{C}_r}=\sup |\eps^{-\frac{1}{3}} \varphi  \cdot \eps \cdot \eps^{(\alpha-2)/3}e^{(\alpha-2)\sigma}| \lar \eps^{\alpha/3} e^{(\alpha-3)\sigma_m} = \rmO(\eps^{\alpha/3}) ,
\]

\[
\|\eps^{-\frac{1}{3}}\varphi U_*\mu \|_{\mathcal{C}_r} =\sup |\eps^{-\frac{1}{3}} \varphi \cdot U_*\mu \cdot \eps^{(\alpha-2)/3}e^{(\alpha-2)\sigma} | \lar \eps^{\alpha/3} e^{(\alpha-2)\sigma_m}  =\rmO( \eps^{\alpha/3}) ,
\]

\[
\|\eps^{-\frac{1}{3}}\varphi \mu^2 \|_{\mathcal{C}_r} \lar \|\eps^{-\frac{1}{3}}\varphi \eps \|_{\mathcal{C}_r} \text{ since } \mu^2 = \rmO(\eps^{4/3}) \ll \eps, 
\]
and lastly
\[
\|\eps^{-\frac{1}{3}}\varphi U_*^3\|_{\mathcal{C}_r} \lar \eps^{\frac{\alpha}{3}} e^{\alpha\sigma} \lar \eps^{\frac{\alpha}{3}} e^{\alpha\sigma_{\sup}} = \rmO(\delta^\alpha).
\]
Combining all the estimates we conclude that 
\[
\|\eps^{-1/3}\varphi \Ral_r(W_r) \| = \max\{ \rmO(\delta^\alpha), \rmO(\delta^{1-\alpha})\},
\]
since we assumed $\alpha \ll 1$, it follows that $\|\eps^{-1/3}\varphi \Ral_r(W_r) \| = \rmO(\delta^\alpha)$, this completes the proof.
\end{proof}

\subsection{Fixed point argument and the proof of Theorem \ref{thm_r}}
In this section we prove theorem \ref{thm_r} by setting up an appropriate fixed point argument.
\begin{proof}[Proof of theorem \ref{thm_r}]
Items \ref{thm_r_1} and \ref{thm_r_2} in the assertion of the theorem has been demonstrated in section \ref{exit_time} and \ref{equation_wr}. Items \ref{thm_r_3} and \ref{thm_r_4} is a direct consequence of the fact that $W_r \in \mathcal{C}_r$, to prove this, we first rewrite equation \eqref{rescl_wr} and the boundary condition $W_r(\sigma_T)=w_T$ as 
\[
F_r(W_r, W_T; \eps, u_0) = 0,
\]
where $F_r : \mathcal{C}_r\times \mathbb{R} \to \mathcal{C}_r\times \mathbb{R}$ is defined as
\begin{align*}
F_r(W_r,w_T;\eps,u_0) &= A_rW_r - \left(\eps^{-1/3}\varphi \Ral_r(W_r), w_T \right)\\
&=\left( \frac{d}{d\sigma}W_r-aW_r - \eps^{-1/3}\varphi \Ral_r(W_r), W_r(\sigma_T)-w_T \right).
\end{align*} 
%Now we are ready to use a fixed point argument to solve the equation 
%\[
%A_r W_r = (\eps^{-1/3}\varphi \mathcal{R}_r(W_r), w_T).
%\]


Let $X = \mathcal{C}_r \times (-\delta_1,
\delta_1)$, where $\delta_1 = \rmO(\delta^{2-\alpha})$ is small, we introduce the solution map $\mathcal{S}: X \to \mathcal{C}_r\times \mathbb{R}$ as follows:
\[
\mathcal{S}(W_r,w_T; \eps, u_0) = (W_r-A_r^{-1}F_r(W_r,w_T;\eps, u_0), w_T),
\]

From propositions \ref{inv_A_r} and \ref{nl_est_r}, we conclude 
\begin{itemize}
\item $\|\mathcal{S}(0,0;\eps,u_0) \|= \|\left( -A_r^{-1}F_r(0,0;\eps,u_0),0\right)\| \le \|A_r^{-1}\|\|F_r(0,0;\eps,u_0)\| \lar |\|\eps^{-1/3}\varphi \Ral_r(0)\| \lar \delta^\alpha$, uniformly in $\eps$ and $u_0$.

\item $\mathcal{S}$ is a smooth map in $W_r,w_T$ as well as the parameters $\eps, u_0$.

\item Since the derivative of $f(U_*+W_r,\mu;\eps)$ with respect to $W_r$ is $D_{W_r} f(U_*+W_r,\mu;\eps)=\rmO(\mu, U_*^2)$, the linearization of $\mathcal{S}$ at $(0,0)$, $D_{(W_r,w^*)} \mathcal{S}(0,0)$ satisfies
\[
\|D_{(W_r,w^*)} \mathcal{S}(0,0)\|_{op} \lar \sup|\eps^{-1/3}\varphi(\mu+U_*^2)| = \rmO(\delta),
\]
where $\|\cdot\|_{op}$ denotes the operator norm of the associated linear operator.
Moreover, for $\|W_r\|$ small enough and $|w_T|\le \delta_1$, we have $D_{(W_r,w_T)}\mathcal{S}(W_r,w_T;\eps,u_0) =  D_{(0,0)}\mathcal{S}(W_r,w_T;\eps,u_0)+\rmO(\|W_r\|_{\mathcal{C}_r})$, which is uniformly small in $\eps$ and $u_0$.

\end{itemize}

Therefore, for $(W_r,w_T)$ in a small ball of $X$, we can apply an iteration scheme and utilize the Banach fix point theorem to the existence of a fixed point, hence a solution to equation \eqref{rescl_wr} exists. Moreover, this solution depends smoothly on the parameter $\eps, u_0$. By picking $w_T = 0$, we have shown that a unique fixed point $W_r \in \mathcal{C}_r$ exists and solves equation \eqref{rescl_wr}.


Finally, to prove item \ref{thm_r_5} we need to estimate the Lipschitz constant for the map 
\[
\Psi : u_0 \mapsto w_r(\eps^{-1}\delta; u_0)=W_r(\sigma_m;u_0),
\]
 which maps from a small interval $I$ containing $\bar{u}_0$ to $\mathbb{R}$. We can write $\Psi$ as the composition of two maps $\Psi = \Psi_1 \circ \Psi_2$ where $\Psi_2 : I \to \mathcal{C}_r$ is the map 
\[
 u_0 \mapsto W_r(\sigma; u_0),
\] 
and $\Psi_1 : \mathcal{C}_r \to \mathbb{R}$ is the evaluation map
\[
  W_r(\sigma; u_0) \mapsto W_r(\sigma_m, u_0).
\]
 
To estimate the number $\text{Lip}_{u_0} \Psi$, we need to estimate the number $\text{Lip}_{u_0} \Psi_2$ and $\text{Lip } \Psi_1$.

To estimate $\text{Lip}_{u_0} \Psi_2$, it suffices to estimate the following two quantities
\[
C_1 = \text{Lip}_{W_r} \mathcal{S}, \text{ and }C_2 = \text{Lip}_{u_0} \mathcal{S},
\]
because $W_r$ is the fixed point of the map $\mathcal{S}$, which implies
\[
\text{Lip}_{u_0} \Psi_2 \le  C_2/(1-C_1).
\]

From the definition of $\mathcal{S}$, we see that
\[
C_1 \le \text{Lip}_{W_r} |\eps^{-1/3}\varphi R_r(W_r)|\le \text{Lip}_{W_r} |\eps^{-1/3}\varphi W_r^2| \le\sup_{W_r \in \mathcal{C}_r} |\eps^{-1/3}\varphi W_r | = \rmO(\delta^{1-\alpha}),
\]
where the last line follows from proposition \ref{nl_est_r}.


To estimate $C_2$. We notice that 
\[
C_2 \le \text{ Lip }_{u_0} |\eps^{-1/3}\varphi U_*^3(\sigma;u_0) |.
\]
However if $u_0, \tilde{u}_0$ are two different numbers near $\bar{u}_0$,
\[
\|\eps^{-1/3}\varphi [U_*^3(\sigma;u_0)-U_*^3(\sigma;\tilde{u}_0)] \|_{\mathcal{C}_r} \le \|\eps^{-1/3}\varphi U_*^2 \|_{\mathcal{C}_r} \sup|U_*(\sigma;u_0)-U_*(\sigma;\tilde{u}_0)|,
\] 

proposition \ref{ric_exp} shows
$U_*(\sigma;u_0)= \eps^{1/3}(e^\sigma + e^{-\sigma} r(e^{-\sigma}; u_0))$ for $\sigma$ large, hence 
\[
\partial_{u_0} U_*(\sigma;u_0) \le C\eps^{1/3},
\]
for some constant independent of $u_0$, on the other hand
\[
\|\eps^{-1/3}\varphi U_*^2 \|_{\mathcal{C}_r}  = \rmO(\eps^{(\alpha-1)/3}),
\]
so we conclude that
\[
\|\eps^{-1/3}\varphi [U_*^3(\sigma;u_0)-U_*^3(\sigma;\tilde{u}_0)] \|_{\mathcal{C}_r} \le C\eps^{\alpha/3}|u_0 - \tilde{u}_0|,
\]
or $C_2 = \rmO(\eps^{\alpha/3})$. Hence 
\[
\text{Lip}_{u_0} \Psi_2 = \rmO(\eps^{\alpha/3}).
\]

On the other hand, the evulation map $\Psi_1$ is a linear map which satisfies
\[
|W(\sigma_m) -\widetilde{W}(\sigma_m) | \le\|W - \widetilde{W}\|_{\mathcal{C}_r}\eps^{(2-\alpha)/3} e^{(2-\alpha)\sigma_m} \lar \eps^{(2-\alpha)/3}\|W-\widetilde{W}\|,
\]
 from the definiton of its norm, therefore
\[
\text{Lip } \Psi_1 = \rmO(\eps^{(2-\alpha)/3}),
\]
combine the two estimates we conclude that 
\[
\text{Lip}_{u_0} \Psi \le \left( \text{Lip}_{u_0} \Psi_2 \right) \left( \text{Lip } \Psi_1 \right) = \rmO(\eps^{2/3}),
\] 
which comples the proof of Theore \ref{thm_r}.
\end{proof}

\section{Region B}\label{sec_B}

Region B corresponds to the $t$-time interval $ t^*< t< \eps^{-1}\delta$.

\subsection{Ansatz in region B}

\begin{theorem}\label{thm_l}
For all $\delta, \alpha$ small enough, there exists $\eps_2,C$, such that for all $0<\eps <\eps_2$, and a solution to \eqref{euler_ori_eqn} of the form
\[
u_B(t ) = \bar{u}_*(t) + w_\ell(t),
\]
exists on the time interval $t \in (t^*, \eps^{-1}\delta)$, where $w_\ell$ satisfies
%\begin{enumerate}[label=\textnormal{(\arabic*)}]
%\item 
\begin{equation}\label{thm_l_1}
w_\ell(t) \le C\eps^{(2-\alpha)/3} | \eps^{1/3}(t-\eps^{-1}\delta)+1|, \hspace{0.2in} w_\ell(\sigma^*) =  w^* = \rmO(\eps^{1/2-\alpha/3}).
\end{equation}
 
%\item \label{thm_l_2} $w_\ell(t) \le $ 
%\end{enumerate}
\end{theorem}

We prove this theorem in the rest of this section.
\subsection{Equation of \texorpdfstring{$W_{\ell}$}{Well} and rescaling}

As before, we plug in the ansatz into \eqref{euler_ori_eqn} to derive the equation satisfied by $w_\ell$.
\begin{align}\label{Eqn_wl}
\begin{split}
w_{\ell}' -2\bar{u}_* w_\ell &= w_\ell^2 + f(\bar{u}_*+w_\ell, \mu; \eps):=R_\ell(w_\ell,\mu;\eps).
\end{split}
\end{align}

We want to solve this equation on $t\in (t^*, \eps^{-1}\delta)$. 
Following previous steps, we next rescale the equation to the $\sigma$-time variable using the time rescaling map $\psi = \psi(\sigma; \bar{u}_0)$ and we obtain
\begin{equation}\label{rescl_wl}
\frac{d}{d\sigma} W_\ell - b(\sigma)W_\ell = \eps^{-1/3}\varphi \Ral_\ell(W_\ell,\mu,\eps),
\end{equation}

where 
\begin{itemize}
\item The equation is posed on $\sigma \in (\sigma^*, \sigma_m(\bar{u}_0))$ where $\sigma^* \approx -\eps^{-1/4}$ and $\sigma_m(\bar{u}_0) := \bar{\sigma}_m$ follows the notation used in section \ref{t_sigma}.

\item The term $b(\sigma)$ is defined and has asymptotics:
\[
b(\sigma) := 2u_R(\psi(\sigma))\varphi(\sigma) = -2 + \rmO(|\sigma|^{-1}),
\]
as $\sigma \to -\infty$. Again, the convergence is uniform in $\eps$.

\item The function $W_\ell(\sigma)$ is the rescaled version of $w_\ell(t)$ in the $\sigma$-variable, $w_\ell(t) = w_\ell(\eps^{-1/3}\psi(\sigma)+\eps^{-1}\delta) = W_\ell(\sigma)$. $\bar{U}_*$ is similarly the rescaled version of $\bar{u}_*$,  $\bar{U}_*(\sigma)= \bar{u}_*(t) = \eps^{1/3}\bar{u}_R(\psi(\sigma ) )$.

\item The function $\Ral_\ell$ is a rescaled version of $R_\ell$ such that $\Ral_\ell(W_r;\eps,u_0) = W_\ell^2 + f(\bar{U}_*+W_\ell, \mu;\eps),$ 
\end{itemize}

We also need to specify the boundary value at the left end point $\sigma = \sigma^*$, the complete system we want to solve is
\begin{align}\label{wl_bp}
\begin{split}
\frac{d}{d\sigma} W_\ell - b(\sigma)W_\ell &= \eps^{-1/3}\varphi \Ral_\ell(W_\ell), \text{ }\sigma \in (\sigma^*, \bar{\sigma}_m),\\
W_\ell(\sigma^*) &= w^*.
\end{split}
\end{align}

\subsection{Linear equation and norms}

Similarly, the proof of theorem \ref{thm_l} consists of solving \eqref{rescl_wl} via a fixed point argument on the following function space:
\[
\mathcal{C}_{\ell} = \left\{ w(\sigma) : \sup_{\sigma^*<\sigma<\bar{\sigma}_m} |\eps^{(\alpha-2)/3}\langle\sigma \rangle^{-2/3} w(\sigma)|<\infty \right\}.
\]

%The homogeneous solution $u$, which solves the equation $u_\sigma = b(\sigma) u$ on the whole real line will not belong to this space. This  means we can prescribe a boundary condition at $\sigma = \sigma^*$

To begin with, let us define the operator $A_\ell$ by 
\[
A_\ell w = \left(\frac{d}{d\sigma}w - b(\sigma)w, w(\sigma^*)\right),
\] 
for $w \in \mathcal{D}(A_\ell) \subset \mathcal{C}_\ell.$

\begin{proposition} \label{inv_A_l}
$A_\ell : \mathcal{D}(A_\ell) \subset C_{W_\ell} \to C_{W_\ell} \times \mathbb{R}$, and $A_\ell$ is bounded invertible with its inverse uniformly bounded in $\eps$.
\end{proposition}

\begin{proof}
Similar to the proof of proposition \ref{inv_A_r}, let $v(\sigma) =\eps^{(\alpha-2)/3} \langle\sigma \rangle^{-2/3}w(\sigma)$, we consider the conjugate linear operator
\begin{align*}
\tilde{A}_\ell v &= \left( \langle \sigma\rangle^{-2/3}\left(\frac{d}{d\sigma}-b(\sigma;\eps)\right)\langle \sigma\rangle^{2/3} v, v(\sigma^*) \right) \\
&= \left( \frac{d}{d\sigma}v -\tilde{b}(\sigma;\eps)v, v(\sigma^*) \right),
\end{align*}

where $\tilde{b}$ satisfies 
\[
\tilde{b} = b(\sigma;\eps)-\frac{2}{3}\langle \sigma\rangle^{-1} =-2 + \rmO(|\sigma|^{-1}) \to -2,
\]
uniformly in $\eps$ as $\sigma \to -\infty$.

The associated conjugate equation of 
\[
A_\ell w = (f,w^*) \text{ with } f \in \mathcal{C}_\ell, w^* \in \mathbb{R} , 
\] is 
\[
\tilde{A}_\ell v = (\eps^{(\alpha-2)/3}\langle \sigma \rangle^{-2/3} f,v^*) \text{ with } \eps^{(\alpha-2)/3}\langle \sigma \rangle^{-2/3} f \in C, v^* = \eps^{(\alpha-2)/3}\langle \sigma^* \rangle^{-2/3} w^*.
\]

Again we apply lemma \ref{pert_inv} to conclude that there exist a constant $C$ independent of $\eps$ such that
\[
\|w\|_{\mathcal{C}_\ell} = |v|_\infty \le C(|\eps^{(\alpha-2)/3}\langle \sigma \rangle^{-2/3} f |_{\infty}+| v^*|) = C(\|f\|_{\mathcal{C}_\ell}+\eps^{\alpha/3-1/2}|w^*|),
\]
which shows the claim, provided that $|w^*| = \rmO(\eps^{1/2-\alpha/3})$.
\end{proof}

\iffalse
\begin{proof}
Similar to the proof of proposition \ref{inv_A_r}, we may find $\sigma^{**} \in (\sigma^*, \bar{\sigma}_m)$ so that $|b(\sigma)-(-2)|<\eta$ for any small $\eta$ given provided $\sigma < \sigma^{**}$. Moreover, this $\sigma^{**}$ can be chosen to be independent of $\eps$ because $\sigma^* = \rmO(\eps^{-1/4})$ and $\bar{\sigma}_m = \rmO(1)$.

Then, for $\sigma \in (\sigma^*, \sigma^{**})$, $A_\ell$ is a perturbation of the invertible operator 
\[
A^{'}_\ell: w \mapsto \left(\frac{d}{d\sigma}w+2w, w(\sigma^*)\right),
\]
 which can be seen as follows: for $f \in \mathcal{C}_\ell$, consider the initial value problem
 \[
 \frac{d}{d\sigma}w + 2w = f, \hspace{0.2in} w(\sigma^*) = w^*,
 \]
 which has solution 
 \[
 w(\sigma) = e^{2(\sigma^*-\sigma)}w^* + \int_{\sigma^*}^\sigma e^{2(\tau-\sigma)} f(\tau) d\tau.
 \]
 
 Notice that
 \[
 \|e^{2(\sigma^*-\sigma)} w^*\|_{\mathcal{C}_\ell}  \le |\langle \sigma^{**}\rangle|^{-2/3} |w^* \eps^{(\alpha-2)/3}| \le |w^*\eps^{(\alpha-2)/3}|
 \]
 and
 \[
\left\| \int_{\sigma^*}^{\sigma} e^{2(\tau-\sigma)} f(\tau)d\tau \right\|_{\mathcal{C}_\ell} \le \sup_{\sigma^* \le \sigma \le \sigma^{**}}\left|\langle \sigma \rangle^{-2/3} \int_{\sigma^*}^\sigma e^{2(\tau-\sigma)} \langle \tau \rangle^{2/3} \|f\|_{\mathcal{C}_\ell} d\tau\right| \le C\|f\|_{\mathcal{C}_\ell} \sup_{\sigma \in (\sigma^*, \sigma^{**})} (\sigma^*/\sigma)^{2/3} e^{2(\sigma^*-\sigma)},
 \] 
this shows the invertibility of $A'_\ell$ provided $|w^*| = \rmO(\eps^{2/3})$. Which, in turn, shows the invertibility of $A_\ell$ for $\sigma \in (\sigma^*, \sigma^{**})$.
 
 
 Next for $\sigma \in (\sigma^{**}, \bar{\sigma}_m)$, we can directly write down the solution as
 \[
w(\sigma) = \exp\left(\int^{\sigma}_{\sigma^{**}} b(\tau)d\tau\right) w(\sigma^{**}) + \int_{\sigma^{**}}^{\sigma} \exp\left(-\int_{\sigma}^{s}b(\tau)d\tau\right)f(s)ds,
\]
similar to the reasoning of proposition \ref{inv_A_r}, we have
\[
\left\|\exp\left(\int^{\sigma}_{\sigma^{**}} b(\tau)d\tau\right) w(\sigma^{**})\right\|_{\mathcal{C}_\ell} \le C\left(\frac{\langle \sigma^{**}\rangle}{\langle \bar{\sigma}_m\rangle}\right)^{2/3}
\]
and
\[
\left\| \int_{\sigma^{**}}^{\sigma} \exp\left(-\int_{\sigma}^{s}a(\tau)d\tau\right)f(s)ds\right\|_{\mathcal{C}_\ell} \le  \langle \sigma \rangle^{-2/3}\int_{\sigma^{**}}^{\sigma} \langle s \rangle^{2/3} \|f\|_{\mathcal{C}_\ell}ds \le C\|f\|_{\mathcal{C}_\ell}
\]
for some constant $C$ independent of $\eps$.

Combing the cases for $\sigma \in (\sigma^*, \sigma^{**})$ and $\sigma \in (\sigma^{**}, \bar{\sigma}_m)$, we have proved the proposition.
\end{proof}
\fi

\subsection{Nonlinear estimates}

For $\sigma \in (\sigma^* , \bar{\sigma}_m)$, we will estimate the nonlinear term 
\[
\Ral_{\ell} (W_\ell)=\eps^{-1/3}\varphi(\sigma)\left[ W_\ell^2 + f(\bar{U}_*+W_\ell, \mu;\eps)\right],
\]
 in the $\mathcal{C}_\ell$ norm. As a result, we have

\begin{proposition}\label{nl_est_l}
If $W_\ell \in \mathcal{C}_{\ell}$, then $ \eps^{-1/3}\varphi \Ral_\ell(W_\ell(\sigma))  \in \mathcal{C}_\ell$ and $\| \eps^{-1/3}\varphi \Ral_\ell \|_{\mathcal{C}_\ell} = \rmO(\eps^{\alpha/3})$.
\end{proposition}

\begin{proof}
From the asymptotics \eqref{ric_asy}, we have that
\[
\bar{U}_*(\sigma) = \eps^{1/3} \bar{u}_R(\psi(\sigma)) \lar |\eps \sigma|^{1/3}\le \eps^{1/4}, \hspace{0.2in} \varphi(\sigma) \lar \langle\sigma\rangle^{-1/3} ,
\]
for $\sigma^*\le \sigma \le \bar{\sigma}_m$.

Also, for $\sigma$ in this range, we have $\mu = \eps t -\delta = \eps^{2/3}\psi(\sigma)$ which from definition of the time resclaing in \ref{t_sigma}  that it satisfies
\[
|\mu| \lar |\eps \sigma|^{2/3} \le \eps^{1/2}.
\]
If $W_\ell \in \mathcal{C}_\ell$, then it is true that
\[
|W_\ell(\sigma)| \lar \eps^{(2-\alpha)/3} \langle \sigma \rangle^{2/3} \ll \bar{U}_*(\sigma), 
\]
also recall $|\sigma^*| = \rmO(\eps^{-1/4})$ and $|\bar{\sigma}_m| = \rmO(1)$, from these facts we have
\[
\|\eps^{-1/3}\varphi W_\ell^2\|_{\mathcal{C}_\ell} \lar \sup_{\sigma \in (\sigma^{*},\bar{\sigma}_m)} \eps^{(1-\alpha)/3}\langle \sigma\rangle^{1/3}  \le \eps^{(1-\alpha)/3} \langle \sigma^{*} \rangle^{1/3} = \rmO(\eps^{(3-4\alpha)/12}),
\]
As $f(u,\mu; \eps) = \rmO(\eps(1+u+\mu+u^2),u\mu,\mu^2,u^3)$, we have

\[
f(\bar{U}_*+W_\ell, \mu ;\eps) = \rmO(\eps, (\bar{U}_*+W_\ell)\mu, \mu^2, (\bar{U}_*+W_\ell)^3 ) = \rmO(\eps, \bar{U}_*\mu, \mu^2, \bar{U}_*^3)
\]

Therefore we have the following estimates:
\[
\|\eps^{-1/3}\varphi \eps \|_{\mathcal{C}_\ell} \lar \sup_{\sigma \in (\sigma^{*},\bar{\sigma}_m)} \eps^{-1/3} \varphi(\sigma) \eps (\eps^{(\alpha-2)/3}\langle \sigma\rangle^{-2/3}) \lar \eps^{\alpha/3} \langle\sigma\rangle^{-1} =\rmO(\eps^{\alpha/3}),
\]

\[
\|\eps^{-1/3}\varphi \bar{U}_*\mu \|_{\mathcal{C}_\ell} \lar \sup_{\sigma \in (\sigma^{*},\bar{\sigma}_m)} \eps^{-1/3}\varphi |\eps\sigma|^{2/3}|\eps\sigma|^{1/3}\eps^{(\alpha-2)/3}\langle \sigma\rangle^{-2/3}  = \rmO(\eps^{\alpha/3}),
\]

\[
\|\eps^{-1/3}\varphi \mu^2 \|_{\mathcal{C}_\ell} \lar \|\eps^{-1/3}\varphi \eps \|_{\mathcal{C}_\ell} \text{ since } \mu^2 \lar \eps,
\]
and 
\[
\|\eps^{-1/3}\varphi \bar{U}_*^3 \|_{\mathcal{C}_\ell}  \lar \sup_{\sigma \in (\sigma^{*},\bar{\sigma}_m)}\eps^{(\alpha-2)/3}\langle \sigma\rangle^{-2/3} |\eps \sigma|^{2/3} = \rmO(\eps^{\alpha/3}).
\]

Combining all the estimates we conclude that $\|\eps^{-1/3}\varphi \mathcal{R}_\ell(W_\ell)\|_{\mathcal{C}_\ell} = \rmO(\eps^{\alpha/3})$ if $W_\ell \in \mathcal{C}_\ell$.
\end{proof}

\subsection{Fixed point argument and the proof of Theorem \ref{thm_l}}
We are ready to prove theorem \ref{thm_l}.
\begin{proof}[Proof of theorem \ref{thm_l}]
The proof consists of rewriting equation \eqref{wr_bp} as a fixed point equation. Using the linear operator $A_\ell$, we define
$F_\ell : \mathcal{C}_\ell \times \mathbb{R} \to \mathcal{C}_\ell \times \mathbb{R}$ as
\[
F_\ell(W_\ell, w^*; \eps) = A_\ell W_\ell - (\eps^{-1/3}\varphi \Ral_\ell(W_\ell), w^*),
\]

Let $X = \mathcal{C}_\ell \times (-\eps^{1/2-\alpha/3},\eps^{1/2-\alpha/3})$, we introduce the map $\mathcal{S}: X \to \mathcal{C}_\ell\times \mathbb{R}$ as follows:
\[
\mathcal{S}(W_\ell,w^*; \eps) = (W_\ell-A_\ell^{-1}F_\ell(W_\ell,w^*;\eps), w^*).
\]
From propositions \ref{inv_A_l} and \ref{nl_est_l}, we conclude 
\begin{itemize}
\item $\|\mathcal{S}(0,0;\eps) \|= \|\left( -A_\ell^{-1}F_\ell(0,0;\eps),0\right)\| \le \|A_\ell^{-1}\|\|F_\ell(0,0;\eps)\| \lar \|\eps^{-1/3}\varphi \Ral_\ell(0)\| \lar \eps^{\alpha/3}$, uniformly in $\eps$.

\item $\mathcal{S}$ is a smooth map in $W_\ell,w^*$ as well as the parameters $\eps$.

\item Since the derivative of $f(\bar{U}_*+W_\ell,\mu;\eps)$ with respect to $W_\ell$ is $D_{W_\ell} f(\bar{U}_*+W_\ell,\mu;\eps)=\rmO(\mu, \bar{U}_*^2)$, the linearization of $\mathcal{S}$ at $(0,0)$, $D_{(W_\ell,w^*)} \mathcal{S}(0,0)$ satisfies
\[
\|D_{(W_\ell,w^*)} \mathcal{S}(0,0)\|_{op} \lar \sup|\eps^{-1/3}\varphi(\mu+\bar{U}_*^2)| = \rmO(\eps^{1/4}),
\]
where $\|\cdot\|_{op}$ denotes the operator norm of the associated linear operator.
Moreover, for $\|W_{\ell}\|$ small enough and $|w^*| =\rmO(\eps^{1/2-\alpha/3})$, we have $D_{(W_\ell,w^*)}\mathcal{S}(W_\ell,w^*;\eps) =  D_{(0, 0)}\mathcal{S}(W_\ell,w^*;\eps)+\rmO(\|W_\ell\|_{\mathcal{C}_\ell})$, which is uniformly small in $\eps$.
\end{itemize}

Therefore, for $(W_\ell, w^*)$ in a small ball of $X$, we apply Banach's fixed point argument to the map $\mathcal{S}$ to obtain a solution of equation \eqref{wl_bp} with $W_\ell \in \mathcal{C}_\ell$ and $w^* = \rmO(\eps^{1/2-\alpha/3})$. Scaling back from $\sigma$ to $t$-time, we obtain claim (\ref{thm_l_1}).
\end{proof}


\section{Region C}\label{sec_C}
This region corresponds to the $t$-time interval $0<t<t^*$. Recall $t^*$ is the (left) gluing time which corresponds to when $\sigma = \eps^{-1/4}=:\sigma^*$.

\subsection{Ansatz in region C}

%\begin{equation} \label{crit_mfld}
%\mu + u^2 + \tilde{f}(u,\mu,0) =  0
%\end{equation}


Then, the ansatz in Region $C$ takes the form 
\[
u_C(t) = u_s(t)  +w_s(t).
\]

The properties of the correction term $w_s(t)$ are summerized in the following 
\begin{theorem}\label{thm_s}
For all $\delta, \alpha$ small enough, there exists $\eps_3,C$, such that for all $0<\eps <\eps_3$, and a solution to \eqref{euler_ori_eqn} of the form
\[
u_C(t ) = u_s(t) + w_s(t),
\]
exists on the time interval $t \in (0,t^*)$, such that $w_s$ satisfies
%\begin{enumerate}[label=\textnormal{(\arabic*)}]
%\item 
\begin{equation}\label{thm_s_1}
w_s(t) \le C\eps^{1-\alpha/3} (\eps t -\delta)^{-1}, \hspace{0.2in} w_s(0) = w_0 = \rmO(\delta^{-1}\eps^{1-\alpha/3}).
\end{equation}
 
%\item \label{thm_l_2} $w_\ell(t) \le $ 
%\end{enumerate}
\end{theorem}

This theorem will be proved in the rest of this section.

\subsection{Equation of \texorpdfstring{$W_{s}$}{Ws} and rescaling }

Once again, we plug $u_C$ into \eqref{euler_ori_eqn} and use \eqref{singular} to obtain the equation satisfied by $w_s$.
\begin{align}\label{Eqn_ws}
\begin{split}
\frac{d}{dt} w_{s} -2u_sw_s &=   w_s^2 + f(u_s+w_s, \mu; \eps)-f(u_s, \mu; 0) - \frac{d}{dt}u_s:= R_s(w_s),
\end{split}
\end{align}

which is posed on $t\in (0, t^*)$.

Rescaling to $\sigma$ time, we obtain
\begin{equation}
\frac{d}{d\sigma} W_s - c(\sigma)W_s = \eps^{-1/3}\varphi \Ral_s(W_s),
\end{equation}


Where 
\begin{itemize}
\item The equation is posed on $\sigma \in \left(\sigma_0, \sigma^*\right).$ where $\sigma^* \approx -\eps^{-1/4}$ is the left most point in region $B$ and $\sigma_0  =-\frac{2}{3}\delta^{3/2}\eps^{-1}$.

\item 
Like $W_r$ and $W_\ell$ in region $A$ and region $B$, the function $W_s(\sigma)$ is the rescaled version of $w_s(t)$ in the $\sigma$-variable, $w_s(t) = w_s(\eps^{-1/3}\psi(\sigma)+\eps^{-1}\delta) = W_s(\sigma)$. 

Rescaling to $\sigma$ variable leads to the asymptotics of the rescaled version of $u_s(t)$:
\begin{equation}\label{sing_exp}
U_s(\sigma) :=u_s(\eps^{-1/3}\psi(\sigma)+\eps^{-1}\delta)= -\left(\frac{3}{2}\eps \sigma\right)^{1/3} + \rmO(|\eps \sigma|^{2/3} ).
\end{equation}

\item The function $\Ral_s$ is likewise a rescaled version of $R_s$ such that 
\[
\Ral_s(W_s;\eps) = W_s^2+ f(U_s+W_s, \mu ; \eps)-f(U_s,\mu;0)- \eps^{1/3}\varphi^{-1} \frac{d}{d\sigma}U_s(\sigma),
\] 


\item The term $c(\sigma)$ is defined and has the asymptotics:
\[
c(\sigma) = 2\eps^{-\frac{1}{3}}U_s(\sigma)\varphi(\sigma) = -2 + \rmO(\eps^{1/3}|\sigma|^{1/3}),
\]
as $\sigma \to -\infty$.


\end{itemize}
 
Once again we need to specify the boundary value at the left end point $\sigma = \sigma_0$, the complete system we want to solve is
\begin{align}\label{ws_bp}
\begin{split}
\frac{d}{d\sigma} W_s - c(\sigma)W_s &= \eps^{-1/3}\varphi \Ral_s(W_s), \text{ }\sigma \in (\sigma_0,\sigma^*),\\
W_s(\sigma_0) &= w_0.
\end{split}
\end{align}





\subsection{Linear equation and norms}
The proof of theorem \ref{thm_s} will be complete if we are able to solve \eqref{ws_bp} using a fixed point argument similar to region $A$ and $B$. The function space on which we will solve the $W_s$ equation via a fixed point argument is:
\[
\mathcal{C}_s = \left\{ w(\sigma) : \sup_{\sigma_0 < \sigma <\sigma^*} |\eps^{\frac{\alpha}{3} -1 }\langle \eps\sigma \rangle^{\frac{2}{3}} w(\sigma)|<\infty \right\},
\]


%Variation of constants gives the formula
%\begin{equation}\label{solution2}
%W_s(\sigma) = \exp\left(\int_\tau^\sigma c(\rho)d\rho\right)W_s(\tau) + \int_\tau^\sigma \exp\left(\int_s^\sigma c(\rho)d\rho\right)\eps^{-\frac{1}{3}}\varphi R_s(W_s)	 ds
%\end{equation}
And similarly we define the linear operator $A_s$ which acts on $w \in \mathcal{D}(A_s) \subset \mathcal{C}_s$ by \[
A_s w = \left(\frac{d}{d\sigma}w - cw, w(\sigma_0)\right).
\]

\begin{proposition}\label{inv_A_s}
$A_s : \mathcal{D}(A_s) \subset \mathcal{C}_s \to \mathcal{C}_s \times \mathbb{R}$, and $A_s$ is bounded invertible with its inverse uniformly bounded in $\eps$.
\end{proposition}
\begin{proof}Unlike the case for linear operator $A_r$ and $A_\ell$, lemma \ref{pert_inv} cannot be directly used for the operator $A_s$ since from the asymptotics of $c$ we see that $c(\sigma)$ does not converge to $-2$ as $\sigma \to -\infty$, in fact, it diverges to $-\infty$ as $\sigma \to -\infty$. 
However, for $\sigma \in (\sigma_0, \sigma^*)$, we have
\[
|c(\sigma) - (-2) | \lar |\eps\sigma|^{1/3} \lar \delta^{1/2},
\]
hence for $\delta$ small, $A_s$ is a small perturbation of the operator
\[
L_s : w \mapsto \left( \frac{d}{d\sigma}w+2w, w(\sigma_0)\right).
\]

To see the invertibility of $L_s$ on the weighted space $\mathcal{C}_s$, let $v(\sigma) = \eps^{\alpha/3-1}\langle \eps \sigma \rangle^{2/3}w(\sigma)$ and consider the conjugate linear opeartor
\begin{align*}
\tilde{L}_s: v &\mapsto \left( \langle \eps \sigma \rangle^{-2/3}\left(\frac{d}{d\sigma} + 2\right) \langle \eps \sigma \rangle^{2/3} v(\sigma), v(\sigma_0) \right)\\
&= \left( \left(\frac{d}{d\sigma}+2+\frac{2}{3}\eps\langle \eps\sigma \rangle^{-1}\right)  v, v(\sigma^*) \right),
\end{align*}
which acts on $v \in \mathcal{C}[\sigma_0, \sigma^*]$. 

Hence, the conjugate linear equation of 
\[
L_s w = (f,w_0),
\]
is 
\[
\tilde{L}_s v = (\tilde{f}, v_0),
\]
with $v_0 = \eps^{\alpha/3-1}\langle \eps \sigma_0 \rangle^{2/3} w_0$ and $\tilde{f} = \eps^{\alpha/3-1}\langle \eps \sigma \rangle^{2/3}f$, which is a differential equation of the form
\[
\left( \frac{d}{d\sigma}+2+\rmO(\eps) \right) v =f, \hspace{0.2in} v(\sigma_0) = v_0.
\]
Its linear part is yet another small perturbation of the linear operator $\frac{d}{d\sigma}+2$ on the uniform space $\mathcal{C}[\sigma_0, \sigma^*]$, integrating this equation yields
\[
|v|_\infty \le C(|v_0|+|f|_\infty),
\]
for some constant $C$ independent of $\eps$. Equivalently, in terms of $w$ we have
\[
\|w\|_{\mathcal{C}_s} \le C(|\delta \eps^{\alpha/3-1}w_0|+ \|f\|_{\mathcal{C}_s}),
\]
which shows the invertibility of $L_s$ and uniformity of the inverse in $\eps$, provided that $w_0 = \rmO(\delta^{-1} \eps^{1-\alpha/3})$. The same property hence holds for $A_s$ as well.
\end{proof}
\subsection{Nonlinear estimates}
For $\sigma \in (\sigma_0, \sigma^*)$, we estimate the nonlinear term $\eps^{-1/3}\varphi \Ral_s(W_s(\sigma))$ in the $\mathcal{C}_s$ norm. Similar to proposition \ref{nl_est_r} and \ref{nl_est_l}, we have
\begin{proposition}\label{nl_est_s} If $W_s \in \mathcal{C}_s$, then $\eps^{-\frac{1}{3}}\varphi \Ral_s(W_s(\sigma))  \in \mathcal{C}_{s}$ and $\| \eps^{-\frac{1}{3}}\varphi \Ral_s \|_{\mathcal{C}_s} = \rmO(\delta^{1/2})$.
\end{proposition}

\begin{proof}
Recall that
\[
\eps^{-1/3}\varphi\Ral_s(W_s;\eps) = \eps^{-1/3}\varphi\left[  W_s^2+ f(U_s+W_s, \mu ; \eps)-f(U_s,\mu;0) \right]- \frac{d}{d\sigma}U_s(\sigma),
\]
From the definition of $\psi(\sigma)$ we have
\[
|\varphi(\sigma)|  \lar | \sigma |^{-1/3},
\]
and $\mu = \eps t -\delta = \eps^{2/3} \psi(\sigma)$ satisfies
\[
|\mu | \lar |\eps\sigma|^{2/3}
\] 
From \eqref{sing_exp}, we have that
\begin{align*}
U_s(\sigma)  &= -\left(\frac{3}{2}\eps \sigma\right)^{1/3} + \rmO(|\eps \sigma|^{2/3} ),
\\
\frac{d}{d\sigma}U_s(\sigma) &= -\frac{1}{2}\eps(\eps\sigma)^{-2/3} + \rmO(\eps|\eps \sigma|^{-1/3}),
\end{align*}

and for $W_s \in \mathcal{C}_s$, it holds that
\[
|W_s(\sigma)| \lar \eps^{1-\alpha/3}\langle \eps \sigma\rangle^{-2/3}.
\]

Hence we have the following estimates:
\[
\left\|\frac{d}{d\sigma}U_s(\sigma) \right\|_{C_s}  \lar \sup_{\sigma \in (\sigma_0, \sigma^{*})}\eps^{\alpha/3-1}\langle \sigma\rangle^{2/3} \eps|\eps \sigma|^{-2/3} = \rmO(\eps^{\alpha/3}),
\]

\[
\|\eps^{-1/3}\varphi W_s^2(\sigma) \|_{C_s}  \lar \sup_{\sigma \in (\sigma_0, \sigma^{*})}\eps^{-1/3}|\sigma|^{-1/3}\eps^{1-\alpha/3}\langle \eps \sigma\rangle^{-2/3} = \rmO(\eps^{\frac{1}{4}-\alpha/3}),
\]
The term $\eps^{-1/3}\varphi[ f(U_s+W_s, \mu; \eps)-f(U_s,\mu; 0) ]$ is estimated as follows, we decompose $f(U_s+W_s,\mu;\eps)-f(U_s,\mu;0)$ into $f_1 + f_2$, where
\[
f_1 = f(U_s+W_s,\mu;\eps) - f(U_s+W_s,\mu;0)
\]
and 
\[
f_2 = f(U_s+W_s,\mu;0)  - f(U_s, \mu; 0)
\]

since $f(u,\mu ; \eps) = \rmO(\eps(1+u+\mu+u^2), u\mu,\mu^2,u^3)$, we have
\[
f_1 = \rmO(\eps(1+U_s+W_s+\mu+U_s^2)) = \rmO(\eps)
\]
and 
\[
f_2 = \rmO( W_s\mu + U_sW_s^2+U_s^2W_s+W_s^3).
\]
For $f_1$ we simply esimate
\[
\| \eps^{-1/3}f_1\|_{C_s} \lar \sup_{\sigma \in (\sigma_0, \sigma^{*})} \eps^{-1/3}\varphi (\eps) \eps^{\alpha/3-1}|\eps\sigma|^{2/3} \lar \eps^{\alpha/3}|\eps\sigma_0|^{1/3} = \rmO(\eps^{\alpha/3})
\]

For $f_2$ we have
\[
\|\eps^{-1/3}\varphi W_s\mu\|_{C_s} \lar \sup_{\sigma \in (\sigma_0, \sigma^{*})} \eps^{-1/3}|\sigma|^{-1/3}|\mu| \lar |\eps \sigma_0|^{1/3} = \rmO(\delta^{1/2})
\]
\[
\|\eps^{-1/3}\varphi U_s^2W_s(\sigma) \|_{C_s}  \lar \sup_{\sigma \in (\sigma_0, \sigma^{*})}\eps^{-1/3}|\sigma|^{-1/3} |\eps \sigma|^{2/3} \lar |\eps \sigma_0|^{1/3}= \rmO(\delta^{1/2}),
\]
\[
\|\eps^{-1/3}\varphi U_sW_s^2(\sigma) \|_{C_s}  \lar \sup_{\sigma \in (\sigma_0, \sigma^{*})}\eps^{-1/3}|\sigma|^{-1/3} |\eps \sigma|^{1/3}\eps^{1-\alpha/3}|\eps\sigma|^{-2/3} = \rmO(\eps^{1/2-\alpha/3}),
\]
and
\[
\|\eps^{-1/3}\varphi W_s^3(\sigma) \|_{C_s}  \lar \sup_{\sigma \in (\sigma_0, \sigma^{*})}\eps^{-1/3}|\sigma|^{-1/3} [\eps^{1-\alpha/3}|\eps\sigma|^{-2/3}]^2 = \rmO(\eps^{3/4-2\alpha/3}).
\]
Hence $\|\eps^{-1/3}f_2\|_{C_s} = \rmO(\delta^{1/2})$, combining all the estimates we conclude that 
\[
\|\eps^{-1/3}\varphi \Ral_s(W_s) \| = \rmO(\delta^{1/2}),
\]
which finishes the proof.
\end{proof}


\subsection{Fixed point argument and the proof of Theorem \ref{thm_s}}
\begin{proof}[Proof of theorem \ref{thm_s}]
The proof is almost identical to the proof of \ref{thm_l}. Using the linear operator $A_s$, we define the nonlinear operator
$F_s : \mathcal{C}_s \times \mathbb{R} \to \mathcal{C}_s \times \mathbb{R}$ as
\[
F_s(W_s, w_0; \eps) := A_s W_s - (\eps^{-1/3}\varphi \Ral_s(W_s), w_0),
\]

Let $X = \mathcal{C}_s \times (-\delta^{-1}\eps^{1-\alpha/3},\delta^{-1}\eps^{1-\alpha/3})$, we introduce the map $\mathcal{S}: X \to \mathcal{C}_s \times \mathbb{R}$ as follows:
\[
\mathcal{S}(W_s,w_0; \eps) := (W_s-A_s^{-1}F_s(W_s,w_0;\eps), w_0).
\]
We conclude from proposition \ref{inv_A_s} and \ref{nl_est_s} that:
\begin{itemize}
\item $\|\mathcal{S}(0,0;\eps) \|= \|\left( -A_s^{-1}F_s(0,0;\eps),0\right)\| \le \|A_s^{-1}\|\|F_s(0,0;\eps)\| \lar \|\eps^{-1/3}\varphi \Ral_s(0)\| \lar \eps^{\alpha/3}$, uniformly in $\eps$.

\item $\mathcal{S}$ is a smooth map in $W_s,w_0$ as well as the parameter $\eps$.

\item Since the derivative of $f(U_s+W_s,\mu;\eps)$ with respect to $W_s$ at $W_s=0$ is $D_{W_s} f(U_s+W_s,\mu;\eps)=\rmO(\mu, U_s^2)$, the linearization of $\mathcal{S}$ at $(0,0)$, $D_{(W_s,w_0)} \mathcal{S}(0,0)$ satisfies
\[
\|D_{(W_s,w_0)} \mathcal{S}(0,0)\|_{op} \lar \sup_{\sigma \in (\sigma_0,\sigma^*)}|\eps^{-1/3}\varphi(\mu+U_s^2)| = \rmO(\delta^{1/2}),
\]
where $\|\cdot\|_{op}$ denotes the operator norm of the associated linear operator.
Moreover, for $\|W_{s}\|$ small enough and $|w_0| =\rmO(\delta^{-1}\eps^{1-\alpha/3})$, we have $D_{(W_s,w_0)}\mathcal{S}(W_s,w_0;\eps) =  D_{(W_s,w_0)}\mathcal{S}(0,0;\eps)+\rmO(\|W_\ell\|_{\mathcal{C}_\ell})$, which is uniformly small in $\eps$.


\end{itemize}

Therefore, for $(W_s, w_0)$ in a small ball of $X$, we apply Banach's fixed point argument to the map $\mathcal{S}$ to obtain a solution of equation \eqref{ws_bp} with $W_s \in \mathcal{C}_s$ and $w_0 = \rmO(\eps^{1-\alpha/3})$. Scaling back from $\sigma$ to $t$-time, we obtain claim (\ref{thm_s_1}).
\end{proof}


\section{Gluing}\label{sec_glue}
So far we have showed solutions of the form $u_A,u_B,$ and $u_C$ exists on region $A$, $B$ and $C$, respectively. 

To show a solution to \eqref{ori_eqn} and \eqref{ori_bc} exists on the whole interval $t\in (0, T)$. We first show that the solution $u_A$, $u_B$ and $u_C$ match at the boundary points of region $A$, $B$, and $C$.

Starting with the left most point in region $C$, Theorem \ref{thm_s} shows \eqref{ori_eqn} has
a solution of the form
\[
u_C(t) = u_s(t) + w_s(t;w_0),
\]
exists, provided that we pick $w_0= \rmO(\delta^{-1}\eps^{1-\alpha/3})$.

At the $t^*$, the right end of region $C$, we see $u_C(t^*) = u_s(t^*) + w_s(t^*; w_0)$ has the following expansion:
\begin{align*}
u_s(t^*) &= -\sqrt{\delta-\eps t^*}+\rmO(|\delta-\eps t^*|) = -\eps^{1/4}+\rmO(\eps^{1/2}),\\
w_s(t^*) &\lar \eps^{1-\alpha/3}(\eps t^*-\delta)^{-1} =  \rmO(\eps^{1/2-\alpha/3}),\\
\implies u_C(t^*) &= -\eps^{1/4} + \rmO(\eps^{1/2-\alpha/3}).
\end{align*}


Notice at $t^*$, $\bar{u}_*(t^*)$ has the following expansion in $\eps$:
\begin{align*}
\bar{u}_*(t^*) &= \eps^{1/3}\bar{u}_R(\eps^{1/3}(t^*-\eps^{-1}\delta)) = -\eps^{1/4}+ \rmO(\eps^{1/2}).
\end{align*}
Hence, if we set $w^* = u_s(t^*)-\bar{u}_*(t^*)+w_s(t^*)$, then $w^* = \rmO(\eps^{1/2-\alpha/3})$, we can apply Theorem \ref{thm_l} to get a solution of \eqref{ori_eqn} of the form
\[
u_B(t) = \bar{u}_*(t) + w_\ell(t; w^*),
\]
in region $B$. With the matching condition
\begin{equation} \label{match_bc}
u_C(t^*) = u_B(t^*).
\end{equation}
These implies, from Theorem \ref{thm_l}, that at the point $t=\eps^{-1}\delta$, the right end of region $B$, the value of $u_B(\eps^{-1}\delta) = \bar{u}_*(\eps^{-1}\delta)  + w_\ell(\eps^{-1}\delta; w^*)$ has the following expansion in $\eps$:
\begin{align*}
\bar{u}_*(\eps^{-1}\delta) &= \eps^{1/3}\bar{u}_R(0) = \eps^{1/3}\bar{u}_0, \\
w_\ell(\eps^{-1}\delta) &=  \rmO(\eps^{(2-\alpha)/3}), \\
\implies u_B(\eps^{-1}\delta) &= \eps^{1/3} \bar{u}_0+\rmO(\eps^{(2-\alpha)/3}).
\end{align*}

On the other hand, Theorem \ref{thm_r} proves a solution to \eqref{ori_eqn} of the form $u_A(t;u_0) = u_*(t; u_0) + w_r(t; u_0)$ exists for $ t \in [\eps^{-1}\delta, T]$ with $u_A(T;u_0) = \delta$, moreover, at $t=\eps^{-1}\delta$ the value $u_A(\eps^{-1}\delta; u_0)$ has the following expansion in $\eps$:
\begin{align*}
u_*(\eps^{-1}\delta; u_0) &= \eps^{1/3}u_R(0;u_0) =  \eps^{1/3}u_0,\\
w_r(\eps^{-1}\delta ; u_0) &= \rmO(\eps^{(2-\alpha)/3}),\\
\implies  u_A(\eps^{-1}\delta; u_0 ) &= \eps^{1/3}u_0 + \rmO(\eps^{(2-\alpha)/3}).
\end{align*}

Therefore, the matching condition at $\eps^{-1}\delta$ is
\begin{equation}\label{match_ab}
u_A(\eps^{-1}\delta; u_0) = u_B(\eps^{-1}\delta),
\end{equation}
Using the expansions we obtained, this amounts to solve the equation
\begin{equation}\label{match_ab_exp}
0= \eps^{1/3}(u_0 - \bar{u}_0) + w_r(\eps^{-1}\delta; u_0)-w_\ell(\eps^{-1}\delta),
\end{equation}
in the variable $u_0$.
Let $\phi(\eps; u_0) :=  w_r(\eps^{-1}\delta; u_0)-w_\ell(\eps^{-1}\delta)$, we conclude from Theorem \ref{thm_r} and Theorem \ref{thm_l} that
\[
\phi(\eps; u_0) = \rmO(\eps^{2-\alpha)/3}), 
\]
uniformly in $u_0$ and 
\[
\text{Lip}_{u_0} \phi(\eps; u_0) = \rmO(\eps^{2/3}).
\]
Hence we divide the right hand side of \eqref{match_ab_exp} by $\eps^{1/3}$, and apply the implicit function theorem around the point $(u_0=\bar{u}_0, \eps=0)$ to conclude that for $u_0$ such that $|u_0- \bar{u}_0| = \rmO(\eps^{(1-\alpha)/3})$ the matching condition \eqref{match_ab} is satisfied. In conclusion, we have shown


\pagebreak
\section*{Appendix}
\renewcommand{\thesubsection}{\Alph{subsection}}
We first show how to extend the asymptotics \eqref{ric_asy} to a more general family of solutions.
\subsection{A family of the solution to the Riccati equation }
\begin{proposition}\label{para_ric}
There exist blow up time $\Omega_\infty = \Omega_\infty(u_0)$ that depends smoothly on $u_0$ for $|u_0 - \bar{u}_0|<\eta$, $\eta$ small, with $\Omega_\infty(\bar{u}_0) = \Omega_0$, and the corresponding solution $u_R(s; u_0)$ to the Riccati equation \eqref{ric} of the form
\begin{equation}\label{ric_exp}
u_R(s;u_0) = \frac{1}{\Omega_\infty-s} +  (\Omega_\infty-s) r(\Omega_\infty-s;u_0),
\end{equation}
where the function $r$ is smooth in both variables and satisfies
\begin{equation}\label{ric_reminder}
r( \Omega_\infty-s; u_0) = -\frac{\Omega_\infty}{3} + \rmO(\Omega_\infty-s),
\end{equation}
as $s \to \Omega_\infty$.
\end{proposition}

\begin{proof}To get the dependence from $u_0$ to $\Omega_\infty$, we first add the equation $\frac{d}{ds}s=1$ to equation \eqref{ric} to get a autonomous $2-$dimensional system in the $(s,u)$ plane. Consider a small neighbourhood $I$ containing $\bar{u}_0$ on the vertical $u$-axis, then $u_R(s; u_0)$ is the trajectory that starts at $u_0 \in I$. The map $P_1 : I \to \mathbb{R}$ defined by $P_1(p) = u(2; p)$ is smooth in $p$, as the blow up time for $\bar{u}_R(s;\bar{u}_0)$ is $\Omega_0 >2$. Moreover, the image $P_1(I)$ is a finite interval on the vertical line $s=2$ containing $\bar{u}_R(2;u_0)$ bounded away from $0$, since the trajectory $u_R(s;\bar{u}_0)$ crosses the horizontal axis around $s=1$ and the vector field goes upwards in the first quardrant of the $(s,u)$-plane.

Denote $\tilde{u}_0:= P_1(u_0)$ for brevity (technically, the interval $P_1(I)$ is a small section of the line $s=2$, with a little abuse of notation, we identify $\tilde{u}_0$ with the second coordinate of the point $P_1(u_0)$). Again in the Riccati equation \eqref{ric}, we make a change of variable by setting $z(s) = 1/u(s)$, the equation $z$ satisfies is:
\[
\frac{d}{ds}z(s) = -z^2s -1.
\]
Let $J = \{ 1/\tilde{u}_0 \mid  \tilde{u}_0 \in P_1(I)\}$ and $z(s; 1/\tilde{u}_0)$ is the trajectory which starts at $1/\tilde{u}_0$. We claim that $z(s; 1/\bar{u}_0)$ reaches $0$ at a finite time $\Omega_\infty = \Omega_\infty(1/\bar{u}_0)$. To see this, first notice there is no equilibrium for the two dimensional system $\frac{d}{ds}s=1, \frac{d}{ds}z=-z^2s-1$. Then, on the boundary $s=2$, the vector field takes the form $(1,-2z^2-1)$, which makes any trajectory starting at a point on $J$ moving down towards the right. Moreover, the vector field $(1,-sz^2-1)$ always pointing down in the first qurdrant of the $(s,z)$ plane, so trajectories cannot go upwards. Lastly, the vector field crosses the horizontal axis non-tangentially, it identically equals $(1,-1)$ throughout the line $z=0$, hence, any trajectory which starts at a point on $J$ will cross $z=0$ in finite time at a unique point $\Omega_\infty = \Omega_\infty(1/\tilde{u}_0)$. The dependence of $\Omega_\infty$ on $1/\tilde{u}_0$ is smooth by the smooth dependence on initial conditions.

 We now define another map $P_2 : J \to \mathbb{R}$ by $P_2(1/\tilde{u}_0) = \Omega_\infty(1/\tilde{u}_0)$, we get a smooth map $P: I \to \mathbb{R}$ by the composition
 \[
 P =P_2 \circ f \circ P_1,
 \] 
 where $f(z) = 1/z$ is the inversion map. Since each of the map in the composition is smooth, $P: u_0 \mapsto \Omega_\infty = \Omega_\infty(u_0)$ is smooth as well.

To get the asymptotic expansion, we set $\xi = \Omega_\infty-s$, then $\tilde{z}(\xi)=z(\Omega_\infty-\xi)$ solves
\[
\frac{d}{d\xi} \tilde{z} = \tilde{z}^2(\Omega_\infty-\xi)+1,
\]
and $\tilde{z}(0) = 0$.

Hence we can assume the expansion for $\tilde{z}$ near $\xi=0$ is of the form
\[
\tilde{z} = \xi + z_2\xi^2+z_3\xi^3 + \rmO(\xi^4),
\]
for some constant $z_2,z_3$. Differentiating this expansion, use the equation $\tilde{z}$ solves and comparing coefficients, we find that $z_2 = 0, z_3 = \Omega_\infty/3$.  Changing back from $\tilde{z}(\xi)$ to $z=z(s)$ with $s = \Omega_\infty-\xi$ and recall $z(s) = 1/u(s)$, we find that $u_R(s;u_0)$ has expansion \eqref{ric_exp} with remainder $r$ satisfies \eqref{ric_reminder}.


\end{proof}

Next we show the main perturbation lemma used to prove the invertibility of the linearized operators at the ansatzs.
\subsection{Uniform invertibility of boundary value problems }
\begin{lemma}\label{pert_inv}
Consider the following boundary value problems 
\begin{subequations}
\label{lin_bv}
\begin{align}
       &\dot{u}(\sigma) = a(\sigma) u + f(\sigma), \hspace{0.2in} u(L)= u_L,         \label{eqn_pos_line}  \\
              &\dot{u}(\sigma) = b(\sigma) u + g(\sigma), \hspace{0.2in} u(-M)= u_M,         \label{eqn_neg_line}
\end{align}
\end{subequations}

where equation \eqref{eqn_pos_line} is posed on $\sigma \in [0,L]$ with $L>0$ and \eqref{eqn_neg_line} is posed on $\sigma \in [-M,0]$ with $M>0$. 

Assume $a(\sigma), b(\sigma)$ are continuous functions such that 
\begin{subequations}
\label{ode_asy}
\begin{align}
       &a(\sigma) \to a_+>0, \hspace{0.2in} \sigma \to \infty      ,  \label{ode_asy_a}  \\
       &b(\sigma) \to b_- < 0, \hspace{0.2in} \sigma \to -\infty,         \label{ode_asy_b}
\end{align}
\end{subequations}
then \eqref{lin_bv} has a unique solutions $u_a,u_b$ which satisfies
\begin{subequations}
\label{ode_est}
\begin{align}
       &|u_a|_\infty \le C_a(u_L+|f|_\infty),  \label{ode_est_a}  \\
       &|u_b|_\infty \le C_b(u_m+|g|_\infty),         \label{ode_est_b}
\end{align}
\end{subequations}
for some constants $C_a, C_b$ independent of $L$ and $M$.
\end{lemma}

\begin{proof}
We only prove the result for \eqref{eqn_pos_line} since the other case is similar. 

\textbf{Step I}

To begin with, consider the asymptotic equation 
\begin{equation}\label{asy_eq}
\dot{u} = a_+ u + f(\sigma),\hspace{ 0.5in } u(L) = 0.
\end{equation}
posed on $\sigma \in [0, L]$.
\eqref{ode_est_a} holds for \eqref{asy_eq} since in this case we have
\begin{align*}
u_a(\sigma) &= e^{a_+(\sigma-L)}u_L + \int_L^t e^{a_+(\sigma-s)} f(s)ds \\ 
&\le 2|u_L| + \left|\int_L^t e^{a_+(\sigma-s)}ds \right| |f|_\infty\\ 
&\le  2|u_L|+\frac{1}{a_+} \left|e^{t-L}-1\right||f|\infty\\
& \le 2(|u_L|+|f|_\infty ).
\end{align*}

\textbf{Step II}

Next, give $\eta>0$ small enough and independent of $L$, there exist $\sigma_* \le L$ such that $|a(\sigma)- a_+|< \eta$ for all $\sigma>\sigma*$. It is important to note here that one can choose $\sigma_*$ independent of $L$ as long as $L$ is large enough. A Neuman series argument shows that in this case the operator 
\[ u \mapsto
 \left(\frac{d}{dt}u-a(t)u, u(L)\right)
\] is a $\eta-$perturbation of the asymptotic operator
\[ u \mapsto
 \left(\frac{d}{dt}u-a_+u, u(L)\right),
\]
which acts on the space of coninuous functions $\mathcal{C}([\sigma_*,L])$ with uniform norm (with domain), hence \eqref{ode_asy_a} holds with the sup norm taken on $[\sigma_*, L]$.

\textbf{Step III}

Finally, for $\sigma \in [0,\sigma_*]$, the solution is given by the following formula
\[
u(\sigma) = \exp\left(\int^{\sigma}_{\sigma_*} a(\tau)d\tau\right) u(\sigma_*) + \int_{\sigma_*}^{\sigma} \exp\left(-\int_{\sigma}^{s}a(\tau)d\tau\right)f(s)ds 
\]
since $\sigma_*< \infty$ and does not depend on $L$, there exist a constant $C_1$ independet of $L$ so that 
\[
\max\left\{ \left|\exp\left(\int^{\sigma}_{\sigma_*} a(\tau)d\tau\right)\right|, \left| \int_{\sigma_*}^{\sigma} \exp\left(-\int_{\sigma}^{s}a(\tau)d\tau\right)\right| \right\} \le C_1,
\]
moreover, the value $u(\sigma_*)$ satisfies
\[
u(\sigma_*) \le \sup_{\sigma \in (\sigma_*,L)} |u(\sigma)| \le C_2(u_L + |f|_\infty)
\]
for some constant $C_2$ independent of $L$ from the conclusion in step 2.
Therefore on $[0,\sigma_*]$ the solution satisfies
\[
\sup_{\sigma \in [0,\sigma_*]}|u(\sigma)| \le C_1C_2(u_L+|f|_\infty) +C_1|f|_\infty \le C(u_L+|f|_\infty)
\]
where obviously $C$ does not depend on $L$. Therefore we conclude that
\[
\sup_{\sigma \in [0,L]} = |u|_\infty \le C(u_L+|f|_\infty)
\]
which is \eqref{ode_est_a}.
\end{proof}
%\subsubsection{Matching at \texorpdfstring{$\sigma^*$}{sigma^*} }

\end{document}
