\documentclass[letterpaper,11pt]{article}

\usepackage{ucs}
\usepackage[utf8x]{inputenc}
\usepackage{graphicx}
\usepackage{amsfonts}
\usepackage{dsfont}
\usepackage{amssymb}
\usepackage{amsmath}
\usepackage{amsthm}
\usepackage{enumerate}
\usepackage{stmaryrd}
\usepackage{fullpage}
\usepackage{ifthen}
\usepackage{subfigure}
\usepackage{epic}
\usepackage{authblk}
\usepackage{textcomp}
\usepackage[small]{caption}
\SetSymbolFont{stmry}{bold}{U}{stmry}{m}{n}


\usepackage[hypertexnames=false,colorlinks=true,linkcolor=blue,citecolor=blue]{hyperref}
\usepackage[numbers,comma,square,sort&compress]{natbib}
\usepackage[letterpaper,text={7in,9in},centering]{geometry}

\usepackage{bm}
\usepackage{color}
\usepackage{titlesec}
\setlength{\parindent}{0.0in}
\setlength{\parskip}{1.0ex plus0.2ex minus0.2ex}
\renewcommand{\baselinestretch}{1.1}
\graphicspath{{eps/}{pdf/}}
\setcaptionmargin{0.25in}
\def\captionfont{\itshape\small}
\def\captionlabelfont{\upshape\small}

\renewcommand{\labelenumi}{(\roman{enumi})}

\newcommand{\bqq}{\begin{equation}}
\newcommand{\eqq}{\end{equation}}
\newcommand{\bqs}{\begin{equation*}}
\newcommand{\eqs}{\end{equation*}}

\newcommand{\C}{\mathbb{C}}
\newcommand{\D}{\mathbb{D}}
\newcommand{\N}{\mathbb{N}}
\newcommand{\R}{\mathbb{R}} 
\newcommand{\Z}{\mathbb{Z}}

\newcommand{\rme}{\mathrm{e}}
\newcommand{\rmi}{\mathrm{i}}
\newcommand{\rmd}{\mathrm{d}}
\newcommand{\rmo}{{\scriptstyle\mathcal{O}}}
\newcommand{\rmO}{\mathcal{O}}
\newcommand{\eps}{\varepsilon}
\newcommand{\lar}{ \lesssim }


\newcommand{\Rho}{\bm{\rho}}
\newcommand{\bigma}{\bm{\sigma}}
\newcommand{\diag}{\operatorname{diag}}
\newcommand{\supp}{\operatorname{supp}}

\numberwithin{equation}{section}

\newenvironment{Hypothesis}[1]%
  {\begin{trivlist}\item[]{\bf Hypothesis #1 }\em}{\end{trivlist}}

\renewcommand{\arraystretch}{1.25}


% Define Theorem Styles ----------------------------------
\theoremstyle{plain}
\newtheorem{theorem}{Theorem}[section]
\newtheorem{proposition}[theorem]{Proposition}
\newtheorem{lemma}[theorem]{Lemma}
\newtheorem{corollary}[theorem]{Corollary}
\newtheorem{conjecture}[theorem]{Conjecture}
\newtheorem{main}[theorem]{Main Result}
\newtheorem{rmk}[theorem]{rmk}


\newcommand{\etal}{\textit{et al.}\ }

\newcommand{\greg}[1]{%
  {\color{blue}\textbf{Greg:} #1}%
 }
 
\newcommand{\arnd}[1]{%
  {\color{red}\textbf{Arnd:} #1}%
 }

\newenvironment{Proof}[1][.]%
 {\begin{trivlist}\item[]\textbf{Proof#1 }}%
 {\hspace*{\fill}$\rule{0.3\baselineskip}{0.35\baselineskip}$\end{trivlist}}

\renewcommand\labelitemi{$\bullet$}

\title{Passage through a fold without a phase space}
\author{author}
\date{2016}
\begin{document}

\section{Introduction}

Introduce something


\section{Model problem for passage through the fold}
The following problem will be studied using the gluing method instead of blow up.
\begin{align}
\label{model}
\begin{split}
\dot{u} &= \mu+u^2 +u^3\\
\dot{\mu} &= \eps
\end{split}
\end{align}
with boundary condition 
\begin{equation}\label{oBC}
u(T) = \delta \text{ and }\mu(0) =-\delta,
\end{equation}
 where $T$ is another parameter, the ``time of flight'' for the trajectory to shoot from $\mu = -\delta$ to $u = \delta$.

We first study the ``blow up '' problem, starting with rescale $ u = \eps^{1/3}u_1(\eps^{1/3}t)$ and $\mu = \eps^{2/3}\mu_1(\eps^{1/3}t)$. We get the new equations (set $\tau=\eps^{1/3}t$)

\begin{align}
\label{modelrs}
\begin{split}
\partial_\tau u_1 &= \mu_1+u_1^2 +(\eps^{2/3}u_1^4)\\
\partial_\tau \mu_1 &= 1 +(\eps^{1/3} u_1)
\end{split}
\end{align}
The new boundary condition is
\begin{equation}\label{BCs}
u_1(T) = \delta \eps^{-1/3}, \mu_1(0) = -\delta \eps^{-2/3}
\end{equation}

Then if we set $s = \tau - \delta\eps^{-2/3}$ and formally let $\eps \to 0$, equation \eqref{modelrs} has an explicit solution $u_1(\tau) = u_R(s)$ and $\mu_1(\tau) = s$. Where $u_R$ is the unique solution to the riccati equation $\partial_s u_R = s+u_R^2$ with the specific asymptotics [reference]. 

\begin{equation}
\label{ricasy}
u_R(s)=\begin{cases}
  (T_R-s)^{-1}+\rmO(T_R-s), \text{ as }s \to T_R \\
 -(-s)^{1/2} -\frac{1}{4}(-s)^{-1} + \rmO(|s|^{-3/2}), \text{ as }s \to -\infty
\end{cases}
\end{equation}

From this and the boundary condition \eqref{BCs}, we have the asymptotics for $T$:
\begin{equation}
T (\eps)= \delta \eps^{-1} + T_R\eps^{-1/3} - \delta^{-1} + \rmO(\eps^{2/3})
\end{equation}


Boundary condition $u_+(t=T)=\delta$, we derive the asymtotics for $T$,
\[
T = T(\eps) \sim \delta\eps^{-1/6} + t_* -\delta^{-1} = \eps^{-1/3}\Omega_0 +\eps^{-1}\delta -\delta^{-1}.
\]


Using the asymptotics for $\psi$ and $u_R$, we calculate that
\begin{equation}
\varphi(\sigma) =\begin{cases}
 (-\frac{3}{2}\sigma)^{-1/3}, \text{ as }\sigma \to -\infty\\
e^{-\sigma} , \text{ as }\sigma \to \infty.
\end{cases}
\end{equation}

\begin{equation}
u_R(\psi(\sigma)) =\begin{cases}
 -(-\frac{3}{2}\sigma)^{1/3}, \text{ as }\sigma \to -\infty\\
e^{\sigma} , \text{ as }\sigma \to \infty.
\end{cases}
\end{equation}

\begin{equation}
a(\sigma) =\begin{cases}
-2+ \rmO((-\sigma)^{-3/2}), \text{ as }\sigma \to -\infty,\\
2+ \rmO(e^{-2\sigma}), \text{ as }\sigma \to \infty.
\end{cases}
\end{equation}



\pagebreak

\section{summary for set up}
Equation
\begin{align}
\begin{split}
\frac{d}{dt}u(t) &= (\mu+u^2+u^3)(t) \\
\frac{d}{dt}\mu (t)&=  \eps 
\end{split}
\end{align}
with B.C.
\begin{equation}
\mu(0) = -\delta, \hspace{0.2in} u(T) = \delta.
\end{equation}

where $\delta,\eps, T$ are parameters.

\subsection{The Riccati solution}
This is taken from [Krupa, Szmolyan].

Consider the riccati equation
\begin{equation}\label{ric}
\frac{d}{dt}u(t) = t+u(t)^2
\end{equation}

\eqref{ric} is known to have a unique solution (here we denote by $u_R$) with the following asymptotics:
\[
u_R(t) = (\Omega_0-t)^{-1} + \rmO(|\Omega_0-t|)
\] as $t \to \Omega_0^-$ and
\[
u_R(t) = -\sqrt{-t} + \rmO( |t|^{-1})
\] as $t \to -\infty$.

Here the constant $\Omega_0$ is the smallest positive zero of a certain combination of Bessel functions of the first kind. 

\subsection{The \texorpdfstring{$t$}{t} to \texorpdfstring{$\sigma$}{sigma} time rescaling}

step 1: Define $\psi$ as
\[
\psi = \eps^{1/3}(t - \eps^{-1}\delta)
\]

step 2:
Take $M>0$ large, define $\sigma$ as
\begin{align*}
\psi = \psi(\sigma) =\begin{cases}
-(-\frac{3}{2} \sigma)^{2/3} , \text{ for }\sigma \le -M\\
\Omega_0 -e^{-\sigma}, \text{ for }\sigma \ge M,
\end{cases}
\end{align*}
and smooth interpolation in between so that $\psi(0) = 0$, here $\Omega_0$ is the blow-up time for $u_R(s)$, the unique solution to the ricatti equation that satisfy the asymptotics.

\[
\eps^{-1/3}\varphi \frac{d}{dt} = \frac{d}{d\sigma}
\]

We also define $\varphi(\sigma) := \frac{d}{d\sigma}\psi(\sigma) = e^{-\sigma} $ for $\sigma\ge M$ and is equal to $(-\frac{2}{3}\sigma)^{-1/3}$.

For convenience let the map $t \mapsto \sigma$ be denoted as $\rho$.

\iffalse
%\item Asymptotics for $u_R$ and $\varphi$.
\begin{equation*}
\varphi(\sigma) =\begin{cases}
 (-\frac{3}{2}\sigma)^{-1/3}, \text{ as }\sigma \to -\infty\\
e^{-\sigma} , \text{ as }\sigma \to \infty.
\end{cases}
\end{equation*}

\begin{equation*}
u_R(\psi(\sigma)) \to \begin{cases}
 -(-\frac{3}{2}\sigma)^{1/3}, \text{ as }\sigma \to -\infty\\
e^{\sigma} , \text{ as }\sigma \to \infty.
\end{cases}
\end{equation*}

\begin{equation*}
2u_R\varphi(\sigma) \to\begin{cases}
-2+ \rmO((-\sigma)^{-3/2}), \text{ as }\sigma \to -\infty\\
2+ \rmO(e^{-2\sigma}), \text{ as }\sigma \to \infty.
\end{cases}
\end{equation*}
%
\fi
\subsection{Region I}
In $\sigma$ variable, we divide the real line into two segments. In different regions we will have different ansatz.

Region I is defined by 
\[
\left\{ \sigma : -\frac{2}{3}\delta^{\frac{3}{2}}\eps^{-1}<\sigma<0 \right\}.
\] 
Which corresponds to the original time $t$ as 
\[
\left\{ t : 0<t< \eps^{-1}\delta \right\}.
\]
\subsubsection{Important times}
\begin{itemize}
\item $t= 0$
\item $t = t^*$, the (left) gluing time which corresponds to when $\sigma = -\eps^{-1/4}=:\sigma^*$, this is determined when the corresponding remainder term in the asympototics of $u_s$ and $u_\ell$ are equal
(which happens when \[ |\eps \sigma|^{2/3} = \eps^{1/3}|\sigma|^{-2/3}\implies |\sigma| = \eps^{-1/4},\]see the asympototic formula of $u_s$ and $u_\ell$ below)

\end{itemize}

\subsubsection{ansatz in region I}
The ansatz in region I takes the form
\[
u_I(t) = \chi_s(\rho(t))u_s(t) + \chi_l(\rho(t))u_l(t) + W_s(t)+W_l(t)
\]

Where 
\begin{itemize}
\item $u_s(t)$ denotes the ``singular'' branch that forms the slow manifold (critical manifold?) of the original system. It is defined via the relation
\[
u_s(t) = h(\mu(t))
\]
for some smooth function $h$ which solves
\begin{equation}\label{singular}
0 = \mu(t) + h(\mu(t))^2 + h(\mu(t))^3.
\end{equation}

It has the following asymptotics:
\begin{equation}\label{singularAsy}
u_s(t) = -\sqrt{\delta-\eps t} + \rmO(|\delta-\eps t|).
\end{equation}

The equivalent in $\sigma$ variable is
\begin{equation}\label{singularAsySig}
u_s(\sigma) = -\left(\frac{3}{2}\eps \sigma\right)^{1/3} + \rmO(|\eps \sigma|^{2/3} )
\end{equation}

\item $u_l(t)$ is defined by rescaling $u_R$ and restrict it for $t<\eps^{-1}\delta$. Specifically:
\begin{equation}\label{uldef}
u_l(t) = \eps^{1/3} u_R( \eps^{1/3}(t-\eps^{-1}\delta)).
\end{equation}
It solves the equation
\begin{equation}\label{uleq}
\frac{d}{dt}u_l (t) = \mu(t) + u_l^2(t),
\end{equation}
and has the asymptotics
\[
u_\ell = -\sqrt{\delta-\eps t} + \rmO(\eps(\delta-\eps t)^{-1}),
\]
or in equivalent $\sigma$ variable
\[
u_\ell(\sigma) = -\left(\frac{3}{2}\eps \sigma\right)^{1/3} + \rmO(\eps^{1/3}|\sigma|^{-2/3}).
\]

\item The cutoff functions $\chi_s$ and $\chi_l$ are functions of $\sigma$ directly, and they satisfy (for $\sigma \le 0$)
\begin{equation}\label{cutoffIs}
\chi_s(\sigma) =\begin{cases}
1,  \hspace{0.1in} \sigma \le \sigma^* -1\\
0 , \hspace{0.1in} \sigma \ge \sigma^* +1.
\end{cases}
\end{equation}
and
\begin{equation}\label{cutoffIl}
\chi_l(\sigma) =\begin{cases}
0,  \hspace{0.1in} \sigma \le \sigma^* -1\\
1 , \hspace{0.1in} \sigma \ge \sigma^* +1.
\end{cases}
\end{equation}

\item Norms

From notes:
\[
W_{\ell} \approx \eps^{(2-\alpha)/3}\langle \sigma \rangle^{2/3}
\]
and
\[
W_s \approx \eps^{1-\alpha/3}\langle \eps \sigma\rangle^{-2/3}
\]
are the weights we proposed.
\end{itemize} 

\subsubsection{ditrubution of terms}
\begin{align*}
W_s'+W_{\ell}' &= -\chi_s'(u_s - u_\ell) + \chi_s \mu + \chi_s u_s' \\
&+ (W_s+W_\ell+\chi_su_s+\chi_\ell u_\ell)^2+\\
&+(W_s+W_\ell+\chi_su_s+\chi_\ell u_\ell)^3
\end{align*}

use the fact that
\[
u'_{\ell} = \mu + u_\ell^2
\]
and
\[
\mu  + u_s^2 + u_s^3 = 0,
\]

we arrive at
\begin{align*}
W_s'+W_{\ell}' &= -\chi_s'(u_s - u_\ell) - \chi_s\chi_\ell (u_s-u_\ell)^2 + \chi_s u_s' +\\
&+ 2(\chi_su_s+\chi_\ell u_\ell)(W_s+W_\ell)+\\
&+ (W_s+W_\ell)^2+\\
&+(W_s+W_\ell+\chi_su_s+\chi_\ell u_\ell)^3-\chi_s u_s^3
\end{align*}

Here we propose $W_s$ solves the equation:
\begin{align*}
W_\ell' -2u_\ell W_\ell = \chi_s'(u_s-u_\ell)-(\chi_s\chi_\ell)(u_s-u_\ell)^2 + 
W_\ell^2 + \cdots
\end{align*}

But already $W_\ell^2$ term gives some trouble:

Rescale to $\sigma$ time, the above equation become
\[
\frac{d}{d\sigma} W_\ell - a(\sigma) W_\ell  = \eps^{-1/3}\varphi W_\ell^2 + \cdots
\]

where $a(\sigma) = 2u_R(\sigma)\varphi$, satisfies 
\[
a(\sigma) = -2 + \rmO(|\sigma|^{-3/2})
\]
as $\sigma \to -\infty$.
And $\varphi(\sigma) \to -(-3\sigma/2)^{-1/3}$ as $\sigma \to -\infty$.

So that in our proposed norm
\[
\| \eps^{-1/3}\varphi W_\ell^2\| \le \eps^{-1/3}|\sigma|^{-1/3}\eps^{(2-\alpha)/3} |\sigma|^{2/3} \le \eps^{(1-\alpha)/3} |\sigma|^{1/3}
\]

The problem is the range of $\sigma$, here there is no cutoff term that multiplies $W_\ell^2$, so $|\sigma| \le \eps^{-1}$ by the definition of region I at the begining. Which makes the above term to be $\eps^{-\alpha/3}$ large.

For other terms, $u_\ell^3$ is fine, and the residual term $(u_s - u_\ell)\chi_s'$, $\chi_s\chi_\ell(u_s - u_\ell)^2$ behave good with the purposed norm, so I think the main issue is from the $W_\ell^2$ term.
\section{Ansatz without the cutoff functions}
We will use ansatz without cut off functions.
\begin{itemize}
\item For $t \in (t^*, \eps^{-1}\delta)$ (corresponds to $ \sigma^*\le \sigma \le 0$), the ansatz takes the form $u = u_\ell  +W_\ell$.

\item For $t \in (0, t^* )$ (corresponds to $ -\frac{2}{3}\delta^{3/2}\eps^{-1}\le \sigma \le \sigma^*$), the ansatz takes the form $u = u_s  +W_s$.
\end{itemize}

\subsection{Equation of \texorpdfstring{$W_{\ell}$}{Well} }
\begin{align}
\begin{split}
W_{\ell}' -2u_\ell W_\ell &= W_\ell^2 + (u_\ell+W_\ell)^3
 \\
 &=  (3u_\ell^2)W_\ell + (1+3u_\ell)W_\ell^2 + W_\ell^3+u_\ell^3
\end{split}
\end{align}

We want to solve this equation on $t\in (t^*, \eps^{-1}\delta)$.

\subsubsection{Linear equation of \texorpdfstring{$W_{\ell}$}{Wl}}
Rescale to $\sigma$ variable:
\begin{equation}
\frac{d}{d\sigma} W_\ell - b(\sigma)W_\ell = \eps^{-1/3}\varphi R_\ell(W_\ell)
\end{equation}

Asymptotics for $b(\sigma)$:
\[
b(\sigma) = 2\eps^{-1/3}u_\ell(\psi(\sigma))\varphi(\sigma) = 2u_R(\psi(\sigma))\varphi(\sigma) = -2 + \rmO(|\sigma|^{-1})
\]
as $\sigma \to -\infty$.

Function space:
\[
C_{W_\ell} = \left\{ u(\sigma) \mid \sup |\eps^{\frac{\alpha-2}{3}}\langle\sigma \rangle^{-\frac{2}{3}} u(\sigma)|<\infty \right\}
\]

%The homogeneous solution $u$, which solves the equation $u_\sigma = b(\sigma) u$ on the whole real line will not belong to this space. This  means we can prescribe a boundary condition at $\sigma = \sigma^*$


Variation of constants gives the formula
\begin{equation}\label{solution1}
W_\ell(\sigma) = \exp\left(\int_\tau^\sigma b(\rho)d\rho\right)W_\ell(\tau) + \int_\tau^\sigma \exp\left(\int_s^\sigma b(\rho)d\rho\right)\eps^{-\frac{1}{3}}\varphi R_\ell(W_\ell)	 ds.
\end{equation}

For $\sigma \in (\sigma^* , 0)$, we will estimate the term $\eps^{-1/3}\varphi(\sigma)\left[ (3u_\ell^2)W_\ell + (1+3u_\ell)W_\ell^2 + W_\ell^3+u_\ell^3\right]$ under the integral given by the formula.

\begin{theorem}
$\sup_{\sigma^* \le \sigma \le 0} \eps^{-1/3}\varphi R_\ell(W_\ell(\sigma)) = \rmO(\eps^{?})$.
\end{theorem}

\begin{proof}
We start with the term $u_\ell^3$, we wish to show that
\[
\eps^{\frac{\alpha-2}{3}} \langle \sigma \rangle^{-\frac{2}{3}}\int_{\sigma^*}^\sigma \exp\left(\int_s^\sigma b(\rho)d\rho\right)\eps^{-\frac{1}{3}}\varphi(s) u_\ell^3(s) ds.
\]
is uniformly bounded in $\eps$.

Recall the asymptotics $u_\ell(s) \lar (\eps s)^{1/3} + \rmO(\eps^{1/3}|\sigma|^{-2/3})$, $\varphi(s) \le s^{-1/3}$ and the asymptotics of $b(\sigma)$ above, we find out that it is sufficient to estimate instead
\[
\eps^{\frac{\alpha-2}{3}} \langle \sigma \rangle^{-\frac{2}{3}} \int_{\sigma^*}^{\sigma}\eps^{2/3}e^{-2(\sigma-s)} s^{2/3}ds.
\]
This integral is given by the following incomplete gamma functions, and is bounded by
\[
\eps^{\frac{\alpha}{3}} \left[1-e^{-2(\sigma-\sigma^*)} (\sigma^*/\sigma)^{2/3}+ \langle \sigma\rangle^{-2/3}e^{-2\sigma}\left(\Gamma(\frac{2}{3},-2\sigma) -\Gamma (\frac{2}{3},-2\sigma^*) \right)\right] = \rmO(\eps^{\alpha/3}).
\]

Next, we estimate the term
\[
\eps^{-1/3}\varphi(\sigma) W_\ell^3.
\]
We need to show that
\[
\eps^{\frac{\alpha-2}{3}}\langle \sigma \rangle^{-\frac{2}{3}}\int_{\sigma^*}^{\sigma} \exp\left( \int_s^\sigma b(\rho)d\rho\right) \eps^{-\frac{1}{3}}\varphi(s)W_{\ell}^3(s)ds
\]
is uniformly bounded in $\eps$ given that $W_\ell \in C_{W_\ell}$.

Equivalently, it is sufficient to estimate
\[
\eps^{\frac{\alpha-2}{3}} \langle \sigma \rangle^{-\frac{2}{3}} \int_{\sigma^*}^\sigma e^{-2(\sigma-s)} \eps^{-\frac{1}{3}}\varphi (s)[\eps^{\frac{2-\alpha}{3}} \langle s \rangle^{\frac{2}{3}}]^3 ds,
\]
which turns out (?) to be bounded by
\[
\eps^{1-\frac{2\alpha}{3}} |\sigma| \le \eps^{1-\frac{2\alpha}{3}} |\sigma^*| = \rmO( \eps^{\frac{9-8\alpha}{12}} )
\]

Similarly, we have for the quadratic term
\[
\eps^{-1/3}\varphi(\sigma) W_\ell^2 = \rmO(\eps^{\frac{3-4\alpha}{12}} ), \hspace{0.2in} \eps^{-1/3}\varphi(\sigma)u_{\ell}W_{\ell}^2 =  \rmO(\eps^{\frac{2-\alpha}{3}}).
\]

And for the linear term
\[
\eps^{-1/3}\varphi(\sigma) u_\ell^2W_\ell = \rmO(\eps^{\frac{1}{4}}).
\]
\end{proof}


\subsection{Equation of \texorpdfstring{$W_{s}$}{Ws} }
\begin{align}
\begin{split}
W_{s}' -2u_sW_s &=   (3u_s^2)W_s + (1+3u_s)W_s^2 + W_s^3 - u_s'
\end{split}
\end{align}

We want to solve this equation on $t\in (0, t^*)$.
\subsubsection{Linear equation of \texorpdfstring{$W_{s}$}{Ws}}
\begin{equation}
\frac{d}{d\sigma} W_s - c(\sigma)W_s = \eps^{-1/3}\varphi R_s(W_s)
\end{equation}

Asymptotics for $c(\sigma)$:
\[
c(\sigma) = 2\eps^{-\frac{1}{3}}u_s(\psi(\sigma))\varphi(\sigma) = -2 + \rmO(\eps^{1/3}|\sigma|^{1/3})
\]
as $\sigma \to -\infty$.

Function space:
\[
C_{W_s} = \left\{ u(\sigma) \mid \sup |\eps^{\frac{\alpha}{3} -1 }\langle \eps\sigma \rangle^{\frac{2}{3}} u(\sigma)|<\infty \right\}
\]



Variation of constants gives the formula
\begin{equation}\label{solution2}
W_s(\sigma) = \exp\left(\int_\tau^\sigma c(\rho)d\rho\right)W_s(\tau) + \int_\tau^\sigma \exp\left(\int_s^\sigma c(\rho)d\rho\right)\eps^{-\frac{1}{3}}\varphi R_s(W_s)	 ds
\end{equation}


\end{document}