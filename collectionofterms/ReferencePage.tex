\documentclass[letterpaper,11pt]{article}

\usepackage{ucs}
\usepackage[utf8x]{inputenc}
\usepackage{graphicx}
\usepackage{amsfonts}
\usepackage{dsfont}
\usepackage{amssymb}
\usepackage{amsmath}
\usepackage{amsthm}
\usepackage{enumerate}
\usepackage{stmaryrd}
\usepackage{fullpage}
\usepackage{ifthen}
\usepackage{subfigure}
\usepackage{epic}
\usepackage{authblk}
\usepackage{textcomp}
\usepackage[small]{caption}


\usepackage[hypertexnames=false,colorlinks=true,linkcolor=blue,citecolor=blue]{hyperref}
\usepackage[numbers,comma,square,sort&compress]{natbib}
\usepackage[letterpaper,text={7in,9in},centering]{geometry}

\usepackage{bm}
\usepackage{color}
\usepackage{titlesec}
\setlength{\parindent}{0.0in}
\setlength{\parskip}{1.0ex plus0.2ex minus0.2ex}
\renewcommand{\baselinestretch}{1.1}
\graphicspath{{eps/}{pdf/}}
%\setcaptionmargin{0.25in}
\def\captionfont{\itshape\small}
\def\captionlabelfont{\upshape\small}

\renewcommand{\labelenumi}{(\roman{enumi})}

\newcommand{\bqq}{\begin{equation}}
\newcommand{\eqq}{\end{equation}}
\newcommand{\bqs}{\begin{equation*}}
\newcommand{\eqs}{\end{equation*}}

\newcommand{\C}{\mathbb{C}}
\newcommand{\D}{\mathbb{D}}
\newcommand{\N}{\mathbb{N}}
\newcommand{\R}{\mathbb{R}} 
\newcommand{\Z}{\mathbb{Z}}

\newcommand{\rme}{\mathrm{e}}
\newcommand{\rmi}{\mathrm{i}}
\newcommand{\rmd}{\mathrm{d}}
\newcommand{\rmo}{{\scriptstyle\mathcal{O}}}
\newcommand{\rmO}{\mathcal{O}}
\newcommand{\eps}{\varepsilon}

\newcommand{\Rho}{\bm{\rho}}
\newcommand{\bigma}{\bm{\sigma}}
\newcommand{\diag}{\operatorname{diag}}

\numberwithin{equation}{section}

\newenvironment{Hypothesis}[1]%
  {\begin{trivlist}\item[]{\bf Hypothesis #1 }\em}{\end{trivlist}}

\renewcommand{\arraystretch}{1.25}


% Define Theorem Styles ----------------------------------
\theoremstyle{plain}
\newtheorem{theorem}{Theorem}[section]
\newtheorem{proposition}[theorem]{Proposition}
\newtheorem{lemma}[theorem]{Lemma}
\newtheorem{corollary}[theorem]{Corollary}
\newtheorem{conjecture}[theorem]{Conjecture}
\newtheorem{main}[theorem]{Main Result}
\newtheorem{rmk}[theorem]{rmk}


\newcommand{\etal}{\textit{et al.}\ }

\newcommand{\greg}[1]{%
  {\color{blue}\textbf{Greg:} #1}%
 }
 
\newcommand{\arnd}[1]{%
  {\color{red}\textbf{Arnd:} #1}%
 }

\newenvironment{Proof}[1][.]%
 {\begin{trivlist}\item[]\textbf{Proof#1 }}%
 {\hspace*{\fill}$\rule{0.3\baselineskip}{0.35\baselineskip}$\end{trivlist}}

\renewcommand\labelitemi{$\bullet$}


\title{Passage through a fold without a phase space}
\author{author}
\date{2016}

\begin{document}
\section{Reference}
Equation
\begin{align}
\begin{split}
\frac{d}{dt}u(t) &= (\mu+u^2+u^3)(t) \\
\frac{d}{dt}\mu (t)&=  \eps 
\end{split}
\end{align}
with B.C.
\begin{equation}
\mu(0) = -\delta, \hspace{0.2in} u(T) = \delta.
\end{equation}

where $\delta,\eps, T$ are parameters.
\begin{enumerate}
\item $\bold{Region \hspace{0.05in}1} $
\begin{itemize}
\item Ansatz and rescale in time
\[
u_-(t)= \eps^{1/3}u_R(\tau(t)-\tau_0), \hspace{0.2in}\mu(t) = \eps t -\delta = \eps^{2/3}(\tau-\tau_0).
\] 
With 
\[
\tau(t)=\eps^{1/3}t, \hspace{0.2in} \tau_0 = \eps^{-2/3}\delta.
\]
After define $s:= \tau-\tau_0$, $(u_R, s)^T$ solves
\[
\frac{d}{ds} u_R(s) = s+u_R(s)^2, \hspace{0.2in} \frac{d}{ds} s=1 
\]
Which implies 
\[
\frac{d}{dt} u_-(t) = \mu(t) + u_-(t)^2
\]
Yet another rescaling in time $\sigma$, defined via 
\begin{align*}
s = \psi(\sigma) =\begin{cases}
-(-\frac{3}{2} \sigma)^{2/3} , \text{ for }\sigma \le -M\\
\Omega_0 -e^{-\sigma}, \text{ for }\sigma \ge M,
\end{cases}
\end{align*}
and smooth interpolation in between, here $\Omega_0$ is the blow-up time for $u_R(s)$. 

Note if $\varphi:=\frac{d}{d\sigma}\psi(\sigma)$, then
\[
\varphi\frac{d}{ds} = \frac{d}{d\sigma}, \text{ and }\eps^{-1/3}\varphi \frac{d}{dt} = \frac{d}{d\sigma}
\]
\item Asymptotics for $u_R$ and $\varphi$.
\begin{equation*}
\varphi(\sigma) =\begin{cases}
 (-\frac{3}{2}\sigma)^{-1/3}, \text{ as }\sigma \to -\infty\\
e^{-\sigma} , \text{ as }\sigma \to \infty.
\end{cases}
\end{equation*}

\begin{equation*}
u_R(\psi(\sigma)) \to \begin{cases}
 -(-\frac{3}{2}\sigma)^{1/3}, \text{ as }\sigma \to -\infty\\
e^{\sigma} , \text{ as }\sigma \to \infty.
\end{cases}
\end{equation*}

\begin{equation*}
2u_R\varphi(\sigma) to\begin{cases}
-2+ \rmO((-\sigma)^{-3/2}), \text{ as }\sigma \to -\infty\\
2+ \rmO(e^{-2\sigma}), \text{ as }\sigma \to \infty.
\end{cases}
\end{equation*}
\item FP argument
petrubation 
\[
u(t) = \eps^{1/3}(u_R+v)(\sigma), \hspace{0.2in} \mu(t) = \eps^{2/3}(s+\rho)(\sigma).
\]
Equation for $(v,\rho)$
\[
\frac{d}{d\sigma} v = 2(u_R\varphi) v+\varphi v^2 +\varphi\rho+\eps^{1/3}\varphi(u_R+v)^3, \hspace{ 0.2in } \rho = 0.
\]
\item Gluing time


the gluing time $\sigma_*$ is set to equal to $ \log(\eps^{-1/6}\delta )$, notice in terms of the original time $t$, this is at
\[
s(\sigma_*) = \Omega_0-\delta^{-1}\eps^{1/6} = \tau -\tau_0 = \eps^{1/3}t - \eps^{-2/3}\delta \implies   t=t_*:= \eps^{-1/3}[\Omega_0+\eps^{-2/3}\delta -\delta^{-1} \eps^{1/6}]
\]

We note then 
\[
 u_-(t_*) = \eps^{1/3}[(\Omega_0-(\Omega_0-\delta\eps^{1/6}))^{-1} + \rmO(\eps^{1/6})]  = \eps^{1/6}\delta^{-1}+\rmO(\eps^{1/2})
\]
\item norms

We will stop at  $\sigma=\sigma_*$, decide norm from the nonhomogeneous term $\eps^{1/3}\varphi u_R^3 $. We have for $0 \le \sigma \le \sigma_*$, that
\[
\sup_{\sigma \le \sigma_*} \eps^{1/3}\varphi u_R^3 \le \eps^{1/3}e^{2\sigma_*} = \delta = \rmO_\eps(1)
\]
This is the nonhomogeneous term, so we just need to use the usual sup norm.
\end{itemize} 


\item $\bold{Region \hspace{0.1in} 2}$ 
\begin{itemize}
\item Ansatz and rescale
\[
u_+(t) = (\delta\eps^{-1/6} +t_*-t)^{-1}
\]
This is designed so that $u_+(t_*) = \eps^{1/6}\delta^{-1}$ and 
\[
|u_-(t)-u_+(t)| \simeq \rmO(\eps^{2/3}|t-t_*|+\delta\eps^{1/2} ) = \rmO(\eps^{1/2})
\]
for $t$ close enough to $t_*$.

Recall that we stop when $u(t=T)=\delta$, we need to consider the interval $[t_*, T]$. Note the B.C. gives
\[
T \sim \delta\eps^{-1/6} + t_* -\delta^{-1} = \eps^{-1/3}\Omega_0 +\eps^{-1}\delta -\delta^{-1}
\]

We introduce the time $\xi$ with the scaling
\[
e^{-\xi} = u_+(t)^{-1}
\]
\item Asymptotics
\begin{equation*}
u_+(t) =e^\xi
\end{equation*}
As for $\mu(t) = \eps t -\delta$, we have
\begin{align*}
\mu &= \eps(t_*+\delta\eps^{-1/6}-u_+^{-1}) -\delta = \eps^{2/3}\Omega_0 +(\delta-\delta^{-1})\eps^{5/6} -\eps u_+^{-1}\\
&=\Omega_0\eps^{2/3} +(\delta-\delta^{-1})\eps^{5/6}-\eps e^{-\xi}
\end{align*}


\item Gluing time


We will glue at $\xi = \xi_*$, defined via
\[
e^{\xi_*} = u_+(t_*) = \eps^{1/6}\delta^{-1} \implies \xi_* = \log (\eps^{1/6}\delta^{-1})
\]

\item FP argument with ansatz
\[
u(t) = u_+(t) +w(t)
\]
By definition, $u_+(t)$ solves 
\[
\frac{d}{dt} u_+(t) = u_+(t)^2,
\]
Convert the equation in $\xi$ time via 
\[
e^{-\xi}\frac{d}{dt}  = \frac{d}{d\xi},
\]
so we get the equation for $w$ in $\xi$ variable
\[
\frac{d}{d\xi} w = e^{-\xi}\mu +2(e^{-\xi}u_+) w +e^{-\xi}w^2+e^{-\xi}(u_++w)^3, \hspace{0.2in} \frac{d}{d\xi} \mu = \eps e^{-\xi}
\]

\item norms

We have
\[ 
e^{-\xi}u_+^3 \sim e^{2\xi}
\]

also from definition of $u_+(t)$
\[
\mu(t)=\Omega_0\eps^{2/3} +(\delta-\delta^{-1})\eps^{5/6}-\eps e^{-\xi}
\]
Note $|e^{-\xi} | \le e^{-\xi_*} \le \delta \eps^{-1/6}$, so that
\[
\eps e^{-2\xi} \le \eps\delta^2\eps^{-1/3} \le \delta^2\eps^{2/3}  \implies |\eps e^{-\xi}\mu| =\rmO(\eps^{2/3})
\]
which suggests the nonhomogeneous term is dominated by $e^{-\xi}u_+^3$ and hence a $e^{-2\xi}$ weight in the norm.

In fact, due to the resonance of $e^{2\xi}$ with the linear part, we need to choose a slightly weaker norm, let $\eta \in ]0,1[$, and our weight will be $e^{-(2-\eta)\xi}$.
We check the nonhomogeneous term

\begin{align*}
e^{-(2-\eta)\xi} e^{-\xi}\mu &=e^{-(3-\eta)\xi} (\Omega_0 \eps^{2/3} + (\delta-\delta^{-1})\eps^{5/6}-\eps e^{-\xi} )\\
& \le e^{-(3-\eta)\xi} (\eps^{2/3}+\eps^{5/6}+\eps e^{-\xi}) \le  \\
&\sim \delta^{3-\eta}\eps^{\frac{\eta+1}{6}}
\end{align*}


We also check breifly the norm should work with the nonlinearity

quadratic
\[
\sup_{\xi\ge \xi_*} e^{-(2-\eta)\xi}|e^{-\xi}w^2| \le \|w\|\sup |e^{-\xi}w| \le \|w\| e^{-\xi}e^{(2-\eta)\xi} =\|w\|e^{(1-\eta)\xi}
\]
quadratic again
\[
\sup_{\xi\ge \xi_*} e^{-(2-\eta)\xi}|e^{-\xi}u_+w^2|\le \|w\| \sup |e^{-\xi}w u_+| \le \|w\|e^{(2-\eta)\xi}
\]
cubic
\[
\sup_{\xi\ge \xi_*} |e^{-(2-\eta)\xi}e^{-\xi}w^3| \le \|w\|\sup_{\xi \ge \xi_*} |e^{-\xi} w^2| \le \|w\| e^{-\xi}e^{(4-2\eta)\xi}
\]

Linear
\[
\sup_{\xi\ge \xi_*} e^{-(2-\eta)\xi}|e^{-\xi}u_+^2w| \le \|w\| \sup |e^{\xi}u_+^2| \le \|w\|e^{(1-\eta)\xi}
\]
The Lipschitz constant will be of order $e^{-\xi}w \sim e^{(1-\eta)\xi}$, which is small on the relevant interval $\xi_* \le \xi \le 0$.
\pagebreak

\end{itemize}
\end{enumerate}

\end{document}
