\documentclass[letterpaper,11pt]{article}

\usepackage{ucs}
\usepackage[utf8x]{inputenc}
\usepackage{graphicx}
\usepackage{amsfonts}
\usepackage{dsfont}
\usepackage{amssymb}
\usepackage{amsmath}
\usepackage{amsthm}
\usepackage{enumerate}
\usepackage{stmaryrd}
\usepackage{fullpage}
\usepackage{ifthen}
\usepackage{subfigure}
\usepackage{epic}
\usepackage{authblk}
\usepackage{textcomp}
\usepackage[small]{caption}
\SetSymbolFont{stmry}{bold}{U}{stmry}{m}{n}


\usepackage[hypertexnames=false,colorlinks=true,linkcolor=blue,citecolor=blue]{hyperref}
\usepackage[numbers,comma,square,sort&compress]{natbib}
\usepackage[letterpaper,text={7in,9in},centering]{geometry}

\usepackage{bm}
\usepackage{color}
\usepackage{titlesec}
\setlength{\parindent}{0.0in}
\setlength{\parskip}{1.0ex plus0.2ex minus0.2ex}
\renewcommand{\baselinestretch}{1.1}
\graphicspath{{eps/}{pdf/}}
\setcaptionmargin{0.25in}
\def\captionfont{\itshape\small}
\def\captionlabelfont{\upshape\small}

\renewcommand{\labelenumi}{(\roman{enumi})}

\newcommand{\bqq}{\begin{equation}}
\newcommand{\eqq}{\end{equation}}
\newcommand{\bqs}{\begin{equation*}}
\newcommand{\eqs}{\end{equation*}}

\newcommand{\C}{\mathbb{C}}
\newcommand{\D}{\mathbb{D}}
\newcommand{\N}{\mathbb{N}}
\newcommand{\R}{\mathbb{R}} 
\newcommand{\Z}{\mathbb{Z}}

\newcommand{\rme}{\mathrm{e}}
\newcommand{\rmi}{\mathrm{i}}
\newcommand{\rmd}{\mathrm{d}}
\newcommand{\rmo}{{\scriptstyle\mathcal{O}}}
\newcommand{\rmO}{\mathcal{O}}
\newcommand{\eps}{\varepsilon}
\newcommand{\lar}{ \lesssim }


\newcommand{\Rho}{\bm{\rho}}
\newcommand{\bigma}{\bm{\sigma}}
\newcommand{\diag}{\operatorname{diag}}
\newcommand{\supp}{\operatorname{supp}}

\numberwithin{equation}{section}

\newenvironment{Hypothesis}[1]%
  {\begin{trivlist}\item[]{\bf Hypothesis #1 }\em}{\end{trivlist}}

\renewcommand{\arraystretch}{1.25}


% Define Theorem Styles ----------------------------------
\theoremstyle{plain}
\newtheorem{theorem}{Theorem}[section]
\newtheorem{proposition}[theorem]{Proposition}
\newtheorem{lemma}[theorem]{Lemma}
\newtheorem{corollary}[theorem]{Corollary}
\newtheorem{conjecture}[theorem]{Conjecture}
\newtheorem{main}[theorem]{Main Result}
\newtheorem{rmk}[theorem]{rmk}


\newcommand{\etal}{\textit{et al.}\ }

\newcommand{\greg}[1]{%
  {\color{blue}\textbf{Greg:} #1}%
 }
 
\newcommand{\arnd}[1]{%
  {\color{red}\textbf{Arnd:} #1}%
 }

\newenvironment{Proof}[1][.]%
 {\begin{trivlist}\item[]\textbf{Proof#1 }}%
 {\hspace*{\fill}$\rule{0.3\baselineskip}{0.35\baselineskip}$\end{trivlist}}

\renewcommand\labelitemi{$\bullet$}

\title{Passage through a fold without a phase space}
\author{author}
\date{2016}
\begin{document}

\section{Introduction}

Introduce something


\section{Model problem for passage through the fold}
The following problem will be studied using the gluing method instead of blow up.
\begin{align}
\label{model}
\begin{split}
\dot{u} &= \mu+u^2 +u^3\\
\dot{\mu} &= \eps
\end{split}
\end{align}
with boundary condition 
\begin{equation}\label{oBC}
u(T) = \delta \text{ and }\mu(0) =-\delta,
\end{equation}
 where $T$ is another parameter, the ``time of flight'' for the trajectory to shoot from $\mu = -\delta$ to $u = \delta$.

We first study the ``blow up '' problem, starting with rescale $ u = \eps^{1/3}u_1(\eps^{1/3}t)$ and $\mu = \eps^{2/3}\mu_1(\eps^{1/3}t)$. We get the new equations (set $\tau=\eps^{1/3}t$)

\begin{align}
\label{modelrs}
\begin{split}
\partial_\tau u_1 &= \mu_1+u_1^2 +(\eps^{2/3}u_1^4)\\
\partial_\tau \mu_1 &= 1 +(\eps^{1/3} u_1)
\end{split}
\end{align}
The new boundary condition is
\begin{equation}\label{BCs}
u_1(T) = \delta \eps^{-1/3}, \mu_1(0) = -\delta \eps^{-2/3}
\end{equation}

Then if we set $s = \tau - \delta\eps^{-2/3}$ and formally let $\eps \to 0$, equation \eqref{modelrs} has an explicit solution $u_1(\tau) = u_R(s)$ and $\mu_1(\tau) = s$. Where $u_R$ is the unique solution to the riccati equation $\partial_s u_R = s+u_R^2$ with the specific asymptotics [reference]. 

\begin{equation}
\label{ricasy}
u_R(s)=\begin{cases}
  (T_R-s)^{-1}+\rmO(T_R-s), \text{ as }s \to T_R \\
 -(-s)^{1/2} -\frac{1}{4}(-s)^{-1} + \rmO(|s|^{-3/2}), \text{ as }s \to -\infty
\end{cases}
\end{equation}

From this and the boundary condition \eqref{BCs}, we have the asymptotics for $T$:
\begin{equation}
T (\eps)= \delta \eps^{-1} + T_R\eps^{-1/3} - \delta^{-1} + \rmO(\eps^{2/3})
\end{equation}


Boundary condition $u_+(t=T)=\delta$, we derive the asymtotics for $T$,
\[
T = T(\eps) \sim \delta\eps^{-1/6} + t_* -\delta^{-1} = \eps^{-1/3}\Omega_0 +\eps^{-1}\delta -\delta^{-1}.
\]


Using the asymptotics for $\psi$ and $u_R$, we calculate that
\begin{equation}
\varphi(\sigma) =\begin{cases}
 (-\frac{3}{2}\sigma)^{-1/3}, \text{ as }\sigma \to -\infty\\
e^{-\sigma} , \text{ as }\sigma \to \infty.
\end{cases}
\end{equation}

\begin{equation}
u_R(\psi(\sigma)) =\begin{cases}
 -(-\frac{3}{2}\sigma)^{1/3}, \text{ as }\sigma \to -\infty\\
e^{\sigma} , \text{ as }\sigma \to \infty.
\end{cases}
\end{equation}

\begin{equation}
a(\sigma) =\begin{cases}
-2+ \rmO((-\sigma)^{-3/2}), \text{ as }\sigma \to -\infty\\
2+ \rmO(e^{-2\sigma}), \text{ as }\sigma \to \infty.
\end{cases}
\end{equation}

so the linear operator $\mathcal{L} = \frac{d}{d\sigma} - A(\sigma)$, where $A(\sigma) \to A_\pm = \diag(\pm 2, 0)$ as $\sigma \to \pm \infty$. We need to find the right function spaces.



\pagebreak

\section{summary for set up}
Equation
\begin{align}
\begin{split}
\frac{d}{dt}u(t) &= (\mu+u^2+u^3)(t) \\
\frac{d}{dt}\mu (t)&=  \eps 
\end{split}
\end{align}
with B.C.
\begin{equation}
\mu(0) = -\delta, \hspace{0.2in} u(T) = \delta.
\end{equation}

where $\delta,\eps, T$ are parameters.

\subsection{The Riccati solution}
This is taken from [Krupa, Szmolyan].

Consider the riccati equation
\begin{equation}\label{ric}
\frac{d}{dt}u(t) = t+u(t)^2
\end{equation}

\eqref{ric} is known to have a unique solution (here we denote by $u_R$) with the following asymptotics:
\[
u_R(t) = (\Omega_0-t)^{-1} + \rmO(|\Omega_0-t|)
\] as $t \to \Omega_0^-$ and
\[
u_R(t) = -\sqrt{-t} + \rmO( |t|^{-1})
\] as $t \to -\infty$.

Here the constant $\Omega_0$ is the smallest positive zero of a certain combination of Bessel functions of the first kind. 

\subsection{The \texorpdfstring{$t$}{t} to \texorpdfstring{$\sigma$}{sigma} time rescaling}

step 1: Define $\psi$ as
\[
\psi = \eps^{1/3}(t - \eps^{-1}\delta)
\]

step 2:
Take $M>0$ large, define $\sigma$ as
\begin{align*}
\psi = \psi(\sigma) =\begin{cases}
-(-\frac{3}{2} \sigma)^{2/3} , \text{ for }\sigma \le -M\\
\Omega_0 -e^{-\sigma}, \text{ for }\sigma \ge M,
\end{cases}
\end{align*}
and smooth interpolation in between so that $\psi(0) = 0$, here $\Omega_0$ is the blow-up time for $u_R(s)$, the unique solution to the ricatti equation that satisfy the asymptotics.

We also define $\varphi(\sigma) := \frac{d}{d\sigma}\psi(\sigma)$.

For convenience let the map $t \mapsto \sigma$ be denoted as $\rho$.

\iffalse
%\item Asymptotics for $u_R$ and $\varphi$.
\begin{equation*}
\varphi(\sigma) =\begin{cases}
 (-\frac{3}{2}\sigma)^{-1/3}, \text{ as }\sigma \to -\infty\\
e^{-\sigma} , \text{ as }\sigma \to \infty.
\end{cases}
\end{equation*}

\begin{equation*}
u_R(\psi(\sigma)) \to \begin{cases}
 -(-\frac{3}{2}\sigma)^{1/3}, \text{ as }\sigma \to -\infty\\
e^{\sigma} , \text{ as }\sigma \to \infty.
\end{cases}
\end{equation*}

\begin{equation*}
2u_R\varphi(\sigma) \to\begin{cases}
-2+ \rmO((-\sigma)^{-3/2}), \text{ as }\sigma \to -\infty\\
2+ \rmO(e^{-2\sigma}), \text{ as }\sigma \to \infty.
\end{cases}
\end{equation*}
%
\fi
\subsection{Region I}
In $\sigma$ variable, we divide the real line into two segments. In different regions we will have different ansatz.

Region I is defined by $\{ \sigma : \sigma<0 \}$. Which corresponds to the original time $t$ as $\{ t : t< \eps^{-1}\delta\}$.
\subsubsection{Important times}
\begin{itemize}
\item $t= 0$
\item $t = t^*$, the (left) gluing time which corresponds to when $\sigma = \eps^{-1/4}=:\sigma^*$.

\end{itemize}

\subsubsection{ansatz in region I}
The ansatz in region I takes the form
\[
u_I(t) = \chi_s(\rho(t))u_s(t) + \chi_l(\rho(t))u_l(t) + W_s(t)+W_l(t)
\]

Where 
\begin{itemize}
\item $u_s(t)$ denotes the ``singular'' branch that forms the slow manifold (critical manifold?) of the original system. It is defined via the relation
\[
u_s(t) = h(\mu(t))
\]
for some smooth function $h$ which solves
\begin{equation}\label{singular}
0 = \mu(t) + h(\mu(t))^2 + h(\mu(t))^3.
\end{equation}

It has the following asymptotics:
\begin{equation}\label{singularAsy}
u_s(t) = -\sqrt{\delta-\eps t} + \rmO(|\delta-\eps t|).
\end{equation}

The equivalent in $\sigma$ variable is
\begin{equation}\label{singularAsySig}
u_s(\sigma) = -\left(\frac{3}{2}\eps \sigma\right)^{1/3} + \rmO(|\eps \sigma|^{2/3} )
\end{equation}

\item $u_l(t)$ is defined by rescaling $u_R$ and restrict it for $t<\eps^{-1}\delta$. Specifically:
\begin{equation}\label{uldef}
u_l(t) = \eps^{1/3} u_R( \eps^{1/3}(t-\eps^{-1}\delta))
\end{equation}
It solves the equation
\begin{equation}\label{uleq}
\frac{d}{dt}u_l (t) = \mu(t) + u_l^2(t)
\end{equation}


\item The cutoff functions $\chi_s$ and $\chi_l$ are functions of $\sigma$ directly, and they satisfy (for $\sigma \le 0$)
\begin{equation}\label{cutoffIs}
\chi_s(\sigma) =\begin{cases}
1,  \hspace{0.1in} \sigma \le \sigma^* -1\\
0 , \hspace{0.1in} \sigma \ge \sigma^* +1.
\end{cases}
\end{equation}
and
\begin{equation}\label{cutoffIl}
\chi_l(\sigma) =\begin{cases}
0,  \hspace{0.1in} \sigma \le \sigma^* -1\\
1 , \hspace{0.1in} \sigma \ge \sigma^* +1.
\end{cases}
\end{equation}

\item norms

From notes:
\[
W_{\ell} \approx \eps^{2/3-\alpha}\langle \sigma \rangle^{2/3}
\]
and
\[
W_s \approx \eps^{1-\alpha}\langle \eps \sigma\rangle^{-2/3}
\]
\end{itemize} 
\subsubsection{equation for ansatz in region I}

\subsection{Region II}

Region II is defined by $\{ \sigma : \sigma > 0 \}$. Which corresponds to the original time $t$ as $\{ t : t > \eps^{-1}\delta\}$.
\subsubsection{Important times}
\begin{itemize}
\item $T_\infty= \eps^{-1}\delta + \eps^{-1/3}\Omega_0$, is the blow up time to the riccati solution, which corresponds to $\sigma = \infty$.

\item $T = T_\infty - \delta^{-1}$, is the right boundary point

\item $t_*  = T_\infty - \eps^{-1/6}\delta^{-1}$, is the (right) gluing time.
\end{itemize}
\subsubsection{ansatz in region II}
The ansatz in region II takes the form
\[
u_{II}(t) = \chi_r(t)u_r(t) + \chi_b(t)u_b(t) + W_r(t)+W_b(t)
\]

Where 

\begin{itemize}
\item $u_r$ has the same formula as $u_l$, except it is restricted on $t>\eps^{-1}\delta$.
\begin{equation}\label{urdef}
u_r(t) = \eps^{1/3}u_R(\eps^{1/3}(t-\eps^{-1}\delta)),
\end{equation}

it satisfies
\begin{equation}\label{ureq}
\frac{d}{dt}u_r(t) = \mu(t) + u_r^2(t).
\end{equation}

\item $u_b$ is a ``blow up'' layer that is defined as follows:
\begin{equation}\label{ubdef}
u_b(t) = (u_r(t_*)^{-1}+t_* - t)^{-1},
\end{equation}

it satisfies
\begin{equation}\label{ubeq}
\frac{d}{dt} u_b = u_b^2.
\end{equation}

\item The cutoff $\chi_b$ and $\chi_r$ are functions of $t$ and it is true that $1=\chi_b+\chi_r$, and they satisfy

\begin{equation}\label{cutoffIIr}
\chi_r(t) =\begin{cases}
1,  \hspace{0.1in} t \le t_* -1\\
0 , \hspace{0.1in} t \ge t_* +1.
\end{cases}
\end{equation}
and
\begin{equation}\label{cutoffIIb}
\chi_b(t) =\begin{cases}
0,  \hspace{0.1in} t \le t_* -1\\
1 , \hspace{0.1in} t \ge t_* +1.
\end{cases}
\end{equation}

\pagebreak
\item We introduce the time $\xi$ with the scaling
\[
e^{-\xi} = u_b(t)^{-1}
\] Asymptotics \begin{equation*}
u_b(t) =e^\xi
\end{equation*}
As for $\mu(t) = \eps t -\delta$, we have
\begin{align*}
\mu &= \eps(t_*+\delta\eps^{-1/6}-u_+^{-1}) -\delta = \eps^{2/3}\Omega_0 +(\delta-\delta^{-1})\eps^{5/6} -\eps u_+^{-1}\\
&=\Omega_0\eps^{2/3} +(\delta-\delta^{-1})\eps^{5/6}-\eps e^{-\xi}
\end{align*}


\item Gluing time


the gluing time $\sigma_*$ is set to equal to $ \log(\eps^{-1/6}\delta )$, notice in terms of the original time $t$, this is at
\[
s(\sigma_*) = \Omega_0-\delta^{-1}\eps^{1/6} = \tau -\tau_0 = \eps^{1/3}t - \eps^{-2/3}\delta \implies   t=t_*:= \eps^{-1/3}[\Omega_0+\eps^{-2/3}\delta -\delta^{-1} \eps^{1/6}]
\]

We note then 
\[
u_*:= u_-(t_*) = \eps^{1/3}[(\Omega_0-(\Omega_0-\delta^{-1}\eps^{1/6}))^{-1} + \rmO(\eps^{1/6})]  = \eps^{1/6}\delta+\rmO(\eps^{1/2})
\]

\item since $u_b(t)$ solves 
\[
\frac{d}{dt} u_b(t) = u_b(t)^2,
\]
Convert the equation in $\xi$ time via 
\[
e^{-\xi}\frac{d}{dt}  = \frac{d}{d\xi},
\]
\item norms
see subsection on linear equation.

\subsubsection{Distribute equations for ansatz in region II}

Substituting $u_{II}$ into \eqref{model} gives

\begin{align*}
\chi_r'u_r+ u_r'\chi_r +\chi_b'u_b+ u_b'\chi_b + W_r'+W_b' &=\mu+ (\chi_ru_r+W_r+\chi_bu_b+W_b)^2\\
&+  (\chi_ru_r+W_r+\chi_bu_b+W_b)^3
\end{align*}

to get the appropriate distribution of terms, we first simplify:

use $\chi_r+\chi_b = 1$, we have
\[
\chi_r'u_r+\chi_b'u_b = \chi_r'(u_r-u_b)
\]
by \eqref{ureq} and \eqref{ubeq}, we have
\[
u_r'\chi_r = \chi_r( \mu + u_r^2),\hspace{0.1in}
u_b'\chi_b = \chi_b u_b^2 
\]

we expand the quardratic terms first, this gives
\[
(\chi_ru_r+W_r)^2+(\chi_bu_b+W_b)^2+2(\chi_ru_r+W_r)(\chi_bu_b+W_b)
\]

We move $\chi_r'(u_r-u_b)$, $\chi_ru_r'$ and $\chi_bu_b'$ to the right handside, without the cubic terms, the right hand side becomes

\begin{align*}
&\chi_r'(u_r - u_b) + (\chi_r^2-\chi_r)u_r^2 + (\chi_b^2-\chi_b)u_b^2 + 2\chi_r\chi_bu_ru_b \\
&+ 2\chi_ru_rW_r + 2\chi_bu_bW_b + 2\chi_ru_rW_b+ 2\chi_bu_bW_r \\
&+ W_r^2 + W_b^2+ 2W_rW_b.
\end{align*}

Since $\chi_r^2 - \chi_r = \chi_r(\chi_r - 1)=-\chi_r\chi_b$, $\chi_b^2-\chi_b = \chi_b(\chi_b-1) = -\chi_b\chi_r$, we have
\[
(\chi_r^2-\chi_r)u_r^2 + (\chi_b^2-\chi_b)u_b^2 + 2\chi_r\chi_bu_ru_b = -\chi_b\chi_r(u_r-u_b)^2.
\]

For $2\chi_r u_rW_r+2\chi_b u_bW_b+2\chi_ru_rW_b + 2\chi_bu_bW_r$, it can be simplified as
\[
2(\chi_ru_r+\chi_bu_b)W_r = 2(u_r + \chi_b(u_b-u_r))W_r, \hspace{0.1in}2(\chi_ru_r+\chi_bu_b)W_b = 2(u_b + \chi_r(u_r-u_b))W_b.
\]

Lastly, we note
\[
2W_rW_b = 2\chi_rW_rW_b + 2\chi_bW_rW_b
\]

Hence, up to quadratic terms, the original equation becomes
\begin{align*}
W_r' + W_b' &= \chi_b\mu + \chi_r'(u_r-u_b) - \chi_b\chi_r(u_r-u_b)^2 +\\
& + 2u_rW_r + 2\chi_b(u_b-u_r)W_r +\\
& + 2u_bW_b + 2\chi_r(u_r-u_b)W_b +\\
& + W_r^2 + W_b^2 + 2\chi_rW_bW_r + 2\chi_bW_bW_r
\end{align*}

To simplify the cubic term so that it become natural to distribute the terms, first we have
\begin{align*}
&(\chi_ru_r+W_r+\chi_bu_b+W_b)^3\\ &=(\chi_ru_r+\chi_bu_b)^3 + (W_r+W_b)^3+ \\&+ 3(\chi_ru_r+\chi_bu_b)(W_r+W_b)^2+3(\chi_ru_r+\chi_bu_b)^2(W_r+W_b)
\end{align*}

For $(W_r+W_b)^3$ and $3(\chi_ru_r+\chi_bu_b)(W_r+W_b)^2$, we already distributed the quadratic term $(W_r+W_b)^2$ as $W_r^2+W_b^2+2\chi_rW_bW_r+2\chi_bW_bW_r$. The linear term $3(\chi_ru_r+\chi_bu_b)^2(W_r+W_b)$ can be ditrubuted as follows:
\begin{align*}
3(\chi_ru_r+\chi_bu_b)^2(W_r+W_b) &= 3[(\chi_ru_r)^2+2(\chi_r\chi_b)(u_ru_b)] W_r +\\&+ 3(\chi_ru_r)^2W_b + \\
&+3[(\chi_bu_b)^2+2(\chi_r\chi_b)(u_ru_b)] W_b  +\\
&+ 3(\chi_bu_b)^2W_r 
\end{align*}

The pure residual term $(\chi_ru_r + \chi_bu_b)^3$ equals
\begin{align*}
(\chi_r u_r)^3+(\chi_b u_b)^3 + 3(\chi_r^2\chi_b)u_r^2u_b +  3(\chi_b^2\chi_r)u_b^2u_r
\end{align*}
As we shall see, the distribution of the mixed terms is flexible. Hence we have
\subsubsection{Equation for \texorpdfstring{$W_r$}{Wr}}
\begin{align}\label{eqn_wr}
\begin{split}
W_r' - 2u_rW_r &= \frac{\chi_r'}{2}(u_r-u_b)-\frac{\chi_b\chi_r}{2}(u_r-u_b)^2 \\
&+3[(\chi_ru_r)^2+2(\chi_r\chi_b)(u_ru_b)] W_r \\&+[3(\chi_ru_r)^2+\chi_r(u_r-u_b)]W_b 
\\&+(\chi_ru_r)^3+3(\chi_r^2\chi_b)u_r^2u_b
\\&+ [1+(W_r+W_b)+3(\chi_ru_r+\chi_bu_b)](W_r^2 + 2\chi_rW_bW_r)
\end{split}
\end{align}
\subsubsection{Equation for \texorpdfstring{$W_b$}{Wb}}
\begin{align}\label{eqn_wb}
\begin{split}
W_b' - 2u_bW_b &= \chi_b\mu+ \frac{\chi_r'}{2}(u_r-u_b)-\frac{\chi_b\chi_r}{2}(u_r-u_b)^2 
\\&+3[(\chi_bu_b)^2+2(\chi_r\chi_b)(u_ru_b)] W_b \\&+[3(\chi_bu_b)^2+\chi_b(u_b-u_r)]W_r 
\\&+(\chi_bu_b)^3+3(\chi_b^2\chi_r)u_b^2u_r
\\&+ [1+(W_r+W_b)+3(\chi_ru_r+\chi_bu_b)](W_b^2 + 2\chi_bW_bW_r)
\end{split}
\end{align}

\subsubsection{Linear Equation and Norms}
Denote the right hand side of \eqref{eqn_wr} as $R_r$, use the time-scaling map between $t$ and $\sigma$, the entire equation becomes
\begin{equation}\label{rescl_wr}
\left(\frac{d}{d\sigma} - 2\varphi u_R\right) \tilde{W}_r =\eps^{-1/3}\varphi \tilde{R}_r
\end{equation}

Here $\tilde{W}_r(\sigma) = W_r( \rho^{-1}(\sigma))=W_r(t)$, and $\tilde{R}_r$ is similarly defined. We will abuse notation and drop the title below.

Similarly, to rescale equation \eqref{eqn_wb}, recall the time variable $\xi$ is defined by $u_b(t) = e^\xi$, hence we obtain
\begin{equation}\label{rescl_wb}
\left(\frac{d}{d\xi} - 2\right) \tilde{W}_b =e^{-\xi} \tilde{R}_b
\end{equation}


By the asmpyotic properties of $\varphi$ and $u_R$, it is true that
\[
2\varphi(\sigma)u_R(\sigma) \to  2 \text{ as }\sigma \to \infty.
\]

Our goal now is to solve \eqref{rescl_wr} on the interval $\sigma \in (0, \rho(T))$ and solve \eqref{rescl_wb} on $\xi \in (\ln(u_b(\eps^{-1}\delta)), \ln(u_b(T)) )$ using a fixed point argument. 

To do so, we introduce the function spaces below:

\begin{align*}
\mathcal{C}_{Wr} &= \left\{ u(\sigma) : \sup_{\sigma\ge 0} \left|\eps^{(\alpha-2)/3} e^{(\alpha-2)\sigma}u(\sigma)\right| < \infty \right\} \\
%--------------------------------------
&= \left\{ u(t) :\sup_{t \ge \eps^{-1}\delta} \left| (T_\infty(\eps)-t)^{2-\alpha}u(t)\right|<\infty
\right\} \\
%--------------------------------------
\mathcal{C}_{Wb} &= \left\{ u(\xi) : \sup_{e^\xi\ge u_b(\eps^{-1}\delta)} \left| e^{(\alpha-2)\xi}u(\xi)\right| < \infty \right\} \\
%--------------------------------------
 &= \left\{ u(t) : \sup_{t\ge \eps^{-1}\delta} \left| (T_\infty(\eps)-t)^{2-\alpha}u(t)\right| < \infty \right\}
\end{align*}

\begin{theorem}
For $W_r \in \mathcal{C}_{Wr}, W_b \in \mathcal{C}_{Wb}$, it is true that $\eps^{-1/3}\varphi R_r \in \mathcal{C}_{Wr}$ and $e^{-\xi}R_b \in \mathcal{C}_{Wb}$. Specifically
\begin{align}
\|\eps^{-1/3}\varphi R_r \| = \rmO(\eps^{5													\alpha/6})\\
\|e^{-\xi}R_b \| = \rmO(\eps^{?})
\end{align}
\end{theorem}
\begin{proof}
We collect the estimates needed to prove this theorem:

For $\eps^{-1}\delta \le t \le t_*$:
\begin{align*}
|u_r| \lar |u_b| &\lar (T_\infty - t)^{-1} \\
W_b &\lar (T_\infty-t_*)^{\alpha-2}
\end{align*}
For $t_*-1\le t \le T$:
\begin{align*}
W_r &\lar (T_\infty-T)^{\alpha-2}\\
|u_b(t)-u_r(t)| &\lar \eps^{1/3}
\end{align*}
For $|t-t_*|\le 1$:
\begin{align*}
|u_b(t)-u_r(t)| &\lar \eps^{2/3}
\end{align*}
To be consistent, we convert these norms back to the $t$ variable and establish the corresponding estimate.
\end{proof}
\iffalse
%solving directly for $\mu$ we have $\mu \sim e^{-\xi}$ so the nonhomogeneous term in the $w$ equation have
\[ 
e^{-\xi}u_+^3 \sim e^{2\xi}
\]

%also
\[
\mu(t)=\Omega_0\eps^{2/3} +(\delta-\delta^{-1})\eps^{5/6}-\eps e^{-\xi}
\]
%Note $|e^{-\xi} | \le e^{-\xi_*} \le \delta \eps^{-1/6}$, so that
\[
\eps e^{-2\xi} \le \eps\delta^2\eps^{-1/3} \le \delta^2\eps^{2/3}  \implies |\eps e^{-\xi}\mu| =\rmO(\eps^{2/3})
\]
%which suggests the nonhomogeneous term is dominated by $e^{-\xi}u_+^3$ and hence a $e^{-2\xi}$ weight in the norm.

%In fact, due to the resonance of $e^{2\xi}$ with the linear part, we need to choose a slightly weaker norm, let $\eta \in (0,1)$, and our weight will be $e^{-(2-\eta)\xi}$.

%We check the nonhomogeneous term\begin{align*}
e^{-(2-\eta)\xi} e^{-\xi}\mu &=e^{-(3-\eta)\xi} (\Omega_0 \eps^{2/3} + (\delta-\delta^{-1})\eps^{5/6}-\eps e^{-\xi} )\\
& \le e^{-(3-\eta)\xi} (\eps^{2/3}+\eps^{5/6}+\eps e^{-\xi}) \le  \\
&\sim \delta^{3-\eta}\eps^{\frac{\eta+1}{6}}
\end{align*}


%We also check breifly the norm should work with the nonlinearity

%quadratic
\[
\sup_{0\ge \xi\ge \xi_*} e^{-(2-\eta)\xi}|e^{-\xi}w^2| \le \|w\|\sup |e^{-\xi}w| \le \|w\| e^{-\xi}e^{(2-\eta)\xi} =\|w\|e^{(1-\eta)\xi}
\]
%quadratic again
\[
\sup_{0\ge \xi\ge \xi_*} e^{-(2-\eta)\xi}|e^{-\xi}u_+w^2|\le \|w\| \sup |e^{-\xi}w u_+| \le \|w\|e^{(2-\eta)\xi}
\]
%cubic
\[
\sup_{0\ge \xi\ge \xi_*} |e^{-(2-\eta)\xi}e^{-\xi}w^3| \le \|w\|\sup_{\xi \ge \xi_*} |e^{-\xi} w^2| \le \|w\| e^{-\xi}e^{(4-2\eta)\xi}
\]

%Linear
\[
\sup_{0\ge \xi\ge \xi_*} e^{-(2-\eta)\xi}|e^{-\xi}u_+^2w| \le \|w\| \sup |e^{\xi}u_+^2| \le \|w\|e^{(1-\eta)\xi}
\]

%The Lipschitz constant will be of order $e^{-\xi}w \sim e^{(1-\eta)\xi}$, which is small on the relevant interval $\xi_* \le \xi \le 0$. 
\fi

\end{itemize}
\pagebreak


\section{Gluing}
Ansatz
\[
U(t) = \chi_-(t)u_-(t) + \chi_+(t)u_+(t) + W_-(t)+W_+(t;\beta),
\]
where $W_+(t;\beta) = W_+(t) + \beta w_{+}^k$, $w_+^k = \chi_{\{t<\eps^{-1}\delta\}}u_+^2$. Also recall $\mu(t) = \eps t- \delta$.

"insert picture of $\chi_-,\chi_+$."

The support of $\chi_+$ is $(t_*-1,\infty)$ and  the support of $\chi_-$ is $(-\infty, t_*+1)$.

\begin{itemize}
\item 
Plug in the anstaz
\begin{align*}
&\chi_-' u_- + \chi_- u_-' + \chi_+' u_+ + \chi_+ u_+' +W_-'(t)+W_+'(t) = \\
&=\mu +(\chi_-u_-+W_-+\chi_+u_+ +W_+)^2 +(\chi_-u_-+W_-+\chi_+u_+ +W_+)^3.
\end{align*}
where $' = \frac{d}{dt}$.

Useful identities 
\[
\chi_-+\chi_+ = 1, \quad \frac{d}{dt}(\chi_-+\chi_+)(t) = 0,
\]
and
\[ 
 \frac{d}{dt} u_- = \mu+ u_-^2, \quad \frac{d}{dt} u_+ = u_+^2.
 \]
Equation after cancellation:
\begin{align*}
-\chi_+'(u_- -u_+) +W_-'+W_+' &= \chi_+\mu +\chi_-\chi_+(u_+ - u_-)u_- +2\chi_- u_- W_- +W_-^2 \\
&+\chi_-\chi_+(u_- -u_+)u_+ + 2\chi_+ u_+W_+ + W_+^2 \\
&+ 2\chi_+ u_+ W_- + 2\chi_- u_- W_+ + 2W_-W_+\\
&+ (\chi_-u_-+W_- + \chi_+u_+ +W_+)^3
\end{align*}

\item Distribute terms in the $-$ side 
\begin{align*}
W_-' -2u_-W_- &=  2\chi_-(u_- -u_+)W_+ + (\chi_-\chi_+) u_-(u_- -u_+)+\frac{1}{2}\chi_-'(u_+ -u_-) +2\chi_-W_-W_+\\
&+W_-^2+ (\chi_-u_- + W_-)^3\\
&+ 3\chi_-^2\chi_+ u_-^2 u_+  + 6\chi_-\chi_+ u_+ u_-W_- + 6\chi_- u_- W_- W_+ + 3\chi_-(W_-+W_+)W_-W_+ \\&+3(\chi_+u_+)W_-^2 + 3(\chi_-u_-)^2W_+
\end{align*}

\item Distribute terms in the $+$ side 
\begin{align*}
W_+'-2u_+W_+  &=  \chi_+\mu + 2\chi_+(u_+ -u_-)W_-+ (\chi_-\chi_+) u_+(u_- -u_+)+\frac{1}{2}\chi_+'(u_- -u_+) +2\chi_+W_-W_+\\
&+W_+^2+ (\chi_+u_+ + W_+)^3 \\
&+ 3\chi_+^2\chi_-u_+^2u_- + 6\chi_+\chi_- u_+u_- W_+ +6 \chi_+u_+W_+W_- +3\chi_+(W_-+W_+)W_-W_+\\
&+3(\chi_-u_-)W_+^2 + 3(\chi_+u_+)^2W_-
\end{align*}
\end{itemize}


\pagebreak
\subsection{Linear equation.} 

Now the equation in $W_-$ and $W_+$ can be written in the following form
\begin{align*}
W_-' - 2u_- W_- &= \mathcal{R}_- ,\\
W_+' - 2u_+ W_+ &= \mathcal{R}_+
\end{align*}
with $\mathcal{R}_{\pm}$ defined as in the distribution of terms.

First fix
\[
\eta \in (1,2), \nu = 2-\eta\in (0,1).
\]

To be able to solve the linear equation, we first introduce the following weighted spaces, for the $-$ side we have
\[
\mathcal{C}_v = \{u(t) \in \mathcal{C}(0,T) \mid \sup |v(t) u(t)| < \infty\}
\]
where the weight $v(t)$ is defined as follows:
\[
v(t)=
\begin{cases}
\delta^{-\frac{1}{4}}\eps^{\frac{1}{3}(\nu-1)} (T_\infty-t)^\nu, \text{ for }t> \eps^{-1}\delta\\
[\delta^{\frac{1}{4}}\eps^{1/3}+\delta^{-\frac{1}{4}}(\delta-\eps t)]^{-1}, \text{ for }t < \eps^{-1}\delta\\
\end{cases}
\]

We can similarly define $\mathcal{C}_V$, with the other weight $V(t)$ defined as
\[
V(t)=
\begin{cases}
\delta^{-\frac{1}{4}}\eps^{\frac{1}{3}(\nu-1)} (T_\infty-t)^{\nu+1}, \text{ for }t> \eps^{-1}\delta\\
[\delta^{\frac{1}{4}}\eps^{\frac{2}{3}}+\delta^{-\frac{1}{4}}(\delta - \eps t)^{\frac{3}{2}}]^{-1}, \text{ for }t < \eps^{-1}\delta\\
\end{cases}
\]
and for the $+$ side we have:
\[
\mathcal{C}_\eta(0,T) = \{ u(t) \in \mathcal{C}(0,T) \mid \sup_{t\in (0,T)}|(T_\infty - t)^{\eta} u(t)| < \infty  \}
\]

\begin{enumerate}

\item Time scale for $W_-$

The scaling of time is as follows:
\[
s = \eps^{\frac{1}{3}}(t-\eps^{-1}\delta),
\]

\[
s=\psi(\sigma) =\begin{cases}
-(-\frac{3}{2} \sigma)^{2/3} , \text{ for }\sigma \le -M\\
\Omega_0 -e^{-\sigma}, \text{ for }\sigma \ge M,
\end{cases}
\]
At $t = -\infty$, we have
\begin{align*}
s_{-\infty} = -\infty\\
\sigma_{-\infty} = -\infty
\end{align*}

At $t = 0$, we have
\begin{align*}
s_0 &:= \eps^{\frac{1}{3}}(0-\eps^{-1}\delta) = -\eps^{-\frac{2}{3}}\delta \\
\sigma_0 &:= -\frac{2}{3} \delta^{\frac{3}{2}} \eps^{-1}
\end{align*}

At $t = t_*$, we have 
\begin{align*}
s_* &: = \eps^{\frac{1}{3}}(t_* - \eps^{-1}\delta) = \Omega_0 - \delta^{-1}\eps^{\frac{1}{6}} \\
\sigma_* &:= -\log(\delta^{-1}\eps^{\frac{1}{6}})  
\end{align*}
At $t = T$, we have

\begin{align*}
s_T &: = \eps^{\frac{1}{3}}(T-\eps^{-1}\delta) = \Omega_0 - \delta^{-1}\eps^{\frac{1}{3}} \\
\sigma_T &:= -\log(\delta^{-1}\eps^{\frac{1}{3}})
\end{align*}

At $t = T_\infty$, we have
\begin{align*}
s_\infty = \Omega_0\\
\sigma_\infty = \infty
\end{align*}

Hence, as $\eps \to 0$, we see that $\sigma_0 \to -\infty$ and $\sigma_T \to \infty$.

Therefore, rescale 
\[
\frac{d}{dt}W_- -2u_-W_-=\mathcal{R}_- \text{ for }t \in (0,T)
\] into
\[
\frac{d}{d\sigma} W_- -a(\sigma)W_- = \eps^{-\frac{1}{3}}\varphi \mathcal{R}_- \text{ for }\sigma \in \left( -\frac{2}{3}\delta^{\frac{3}{2}}\eps^{-1}  , -\log(\delta^{-1}\eps^{\frac{1}{3}}) \right)
\]

Recall $a(\sigma) \to \pm 2$ as $\sigma \to \pm \infty$.
\item Time scale for $W_+$.

It is scaled as thus
\[
\xi = \log(u_+(t))
\]
At $t = -\infty$, note $u_+ \to 0$ as $t \to -\infty$, then
\[
\xi_{-\infty} := \log(0) = -\infty
\]

At $t= 0 $, we have
\[
\xi_0 := \log(u_+(0)) \sim - \log T_\infty=-\log\left(\eps^{-1}\delta+\eps^{-\frac{1}{3}}\Omega_0\right) = -\log\left(\eps^{-1}(\delta+\eps^{\frac{2}{3}}\Omega_0)\right) \sim \log(\eps)
\]

At $t = T$, we have
\[
\xi_T : = \log(u_+(T)) \sim \log \left(\delta + \rmO(\eps^{\frac{1}{2}} ) \right) \sim \log \delta
\]

At $t = T_\infty$, we have 
\[
\xi_\infty := \log(u_+(T_\infty)) \sim -\log\left(\rmO(\eps^{\frac{1}{2}})\right)
\]
\end{enumerate}
Therefore, rescale 
\[
\frac{d}{dt}W_+ -2u_-W_+=\mathcal{R}_+ \text{ for }t \in (0,T)
\] 
into
\[
\frac{d}{d\xi} W_+ - 2W_+ = e^{-\xi} \mathcal{R}_+ \text{ for }\xi \in \left( \log(\eps) , \log(\delta) \right)
\]

then we can show the Fredholm properties of the linear operators as follows: 
\begin{theorem}
For $t \in (0,T)$, the linear operator on the $-$ side
\[
\frac{d}{dt} - 2u_-(t) : \mathcal{C}_v (0,T) \to \mathcal{C}_V (0,T)
\]

and the linear operator on the $+$ side
\[
\frac{d}{dt}  - 2u_+(t) : \mathcal{C}_\eta (0,T) \to \mathcal{C}_{\eta+1}(0,T)
\]
are Fredholm, and their indices are $-1$, $1$, respectively..
\end{theorem}

\begin{proof}
For the $W_-$ equation, recall we had the  scalings $s = \eps^{\frac{1}{3}}(t-\eps^{-1}\delta) $, $\psi(\sigma) =s $, $\varphi = \partial_\sigma \psi$. Hence the equation in the $\sigma$-variable takes the form
\begin{align*}
\frac{d}{d\sigma} \tilde{W}_- -a(\sigma)\tilde{W}_- = \eps^{-1/3}\varphi \tilde{\mathcal{R}}_-.
\end{align*}
Where $\tilde{W}_-(\sigma) = W_-(\eps^{-\frac{1}{3}} \psi(\sigma)+\eps^{-1}\delta)=W_-(t)$, and similarly for $\tilde{\mathcal{R}}_-$. Now recall $a(\sigma) \to \pm 2 $ as $\sigma \to \pm \infty$. In these variables, the weight satisfies 
\[
v(\sigma) \sim \begin{cases}
 \eps^{-\frac{1}{3}} e^{-\nu \sigma},  \text{ for }\sigma >0 \\
\eps^{-\frac{2}{3}}[(-\sigma)^{\frac{2}{3}}+1]^{-1}, \text{ for }\sigma < 0.
\end{cases}
\]

and 
\[
V(\sigma) \sim \begin{cases}
 \eps^{-\frac{2}{3}} e^{-(\nu+1)\sigma},  \text{ for }\sigma >0 \\
[\eps |\sigma|+\eps^{\frac{2}{3}}]^{-1} \text{ for }\sigma < 0.
\end{cases}
\]

Then for $\nu \neq 2$, the linear operators $\frac{d}{d\sigma} -a(\sigma)$ is Fredholm on the weighted spaces.  Since $0 < \nu < 1$ and $w$ has algebraic decay for $\sigma<0$, we conclude that the Fredholm index is...

For the $W_+$ equation, we used the rescaling $u_+(t)  = e^\xi$, and in the $\xi$ equation, the $+$ side equation becomes
\[
\frac{d}{d\xi}\tilde{W}_+ -2\tilde{W}_+ = e^{-\xi}\mathcal{R}_+
\]
the weight for $W_+$ is just $u_+(t)^{\eta}=e^{\eta \xi}$ and because of $1<\eta<2$, we see the linear opertator $\frac{d}{d\xi}-2$ is Fredholm on this weighted function space.

To find the Fredholm index of this operator.
\end{proof}
\pagebreak

\subsection{Fixed point arguments-set up for \texorpdfstring{$W_{r}$}{Wright}}

We use the ansatz $U = W_r + u_r$ for $t \in  (\eps^{-1}\delta, T)$. Then $W_r$ satisfies
\begin{equation}\label{Wreqn}
\left(\frac{d}{dt} - 2u_r \right)W_r = W_r^2+(u_r+W_r)^3 := R_r,
\end{equation}

When we change variable from $t$ to $\sigma$, we obtain, for $\sigma \in (0,\sigma_T)$,
\[
\left(\frac{d}{d\sigma} -a(\sigma) \right)\tilde{W}_r(\sigma)  = \eps^{-1/3}\varphi \tilde{R}_r(\sigma).
\]

Where $a(\sigma) = 2u_R(\psi(\sigma))\varphi(\sigma)$ satisfy
\[
|a(\sigma)-2| \le Ce^{-2\sigma},
\]
for some constant $C$, as $\sigma \to \infty$.

\textbf{Invertibility of the linear operator} 
The operator $\frac{d}{d\sigma}-a(\sigma) : C_{Wr} \to C_{Wr}$ will be invertible if we can find bounded solution to the equation
\[
\left(\frac{d}{d\sigma}-a(\sigma) \right) u = f.
\]

Variation of constants gives the formula
\begin{equation}\label{solution}
u(\sigma) = \exp\left(\int_\tau^\sigma a(\rho)d\rho\right)u(\tau) + \int_\tau^\sigma \exp\left(\int_s^\sigma a(\rho)d\rho\right)f(s)ds
\end{equation}


If we are looking for bounded solution $u$ on 
$C_{Wr}(0,\infty)$. Which implies $u(\tau) \le 
\eps^{(2-\alpha)/3}\exp\left((2-\alpha)\tau
\right)$, so letting $\tau \to \infty$ gives the formula
\[
u(\sigma) = \int_\infty^\sigma \exp\left(\int_s^\sigma a(\rho)d\rho \right)f(s)ds.
\]
Using the convergence $|a(\sigma)-2|\le e^{-2\sigma}$ we discover that
\[
\|u\|_{Wr} \le C \|f\|_{Wr}
\]
for some constant $C$. Moreover the homogeneous solution DOES NOT belong to the space $C_{Wr}(0,\infty)$, we get Fredholm $-1$!

However, we are solving on the finite interval $\sigma \in (0,\sigma_T)$, to get bounded inverse on this space, we use the solution formula, putting $\tau = \sigma_T$ in \eqref{solution}:
\[
u(\sigma) = \exp\left(\int_{\sigma_T}^\sigma a(\rho)d\rho\right)u(\sigma_T) + \int_{\sigma_T}^\sigma \exp\left(\int_s^\sigma a(\rho)d\rho\right)f(s)ds
\]


We see that 
\begin{align*}
\| e^{\int_{\sigma_T}^\sigma a(\rho
)d\rho }u(\sigma_T)\| &\lar \left|\eps^{\frac{\alpha-2}{3}}e^{(\alpha-2)\sigma}e^{\int (a-2)} e^{2(\sigma-\sigma_T)}u(\sigma_T)\right| \\
&\lar |u(\sigma_T)|
\end{align*}

and
\begin{align*}
\left\|\int_{\sigma_T}^\sigma e^{\int_s^\sigma a(\rho)d\rho}f(s)ds\right\| &\lar \left|\eps^{\frac{\alpha-2}{3}}e^{(\alpha-2)\sigma}\left(\int_{\sigma_T}^\sigma e^{\int_s^\sigma a(\rho)d\rho}f(s)ds\right)\right|_\infty \\
& \le \left| e^{\alpha\sigma}\int_{\sigma_T}^\sigma e^{\int_s^\sigma (a(\rho)-2)d\rho} e^{-\alpha s}ds \right|_\infty \|f\|\\
& \lar \left|\frac{1}{\alpha} \left(e^{\alpha(\sigma-\sigma_T)}-1 \right) \right|_\infty \|f\|\lar \|f\|.
\end{align*}
For some constant $C$ independent of $\eps$, not including the homogeneous part.

The parameter $u(\sigma_T)$ is choosen so that $|u(\sigma_T)|\le \delta$.

We define the operator
\[
\mathcal{L} u = \frac{d}{d\sigma}u-a(\sigma)u.
\]

\textbf{Nonlinear estimates}
Recall that 
\[
R_r(W_r) = W_r^2 + (u_r+W_r)^3
\]


It is easier to estimate the nonlinear terms in the original $t-$variables, we estimate
\[
\|R_r(W_r)(t)\|=\|\eps^{-1/3}\varphi R_r(\sigma)\|_{C_{Wr}}= \sup_{\eps^{-1}\delta \le t \le T}|(T_\infty -t)^{3-\alpha}(W_r^2+(u_r+W_r)^3) |
\]
for $W_r \in C_{Wr}$. But the latter implies that $|W_r|_\infty \le (T_\infty-t)^{\alpha-2}\|W_r\|$, also we have that $|u_r| \lar (T_\infty- t)^{-1}$
\begin{align*}
\|R_r(W_r)\|  &\lar  \left| (T_\infty-t)^{3-\alpha}(W_r^2+u_r^3+W_r^3)\right|_\infty \\
&\lar (T_\infty-T)^{-\alpha}+(T_\infty-T)^{-(1-\alpha)}\|W_r\|^2 +(T_\infty-T)^{2\alpha-3}\|W_r\|^3\\
& \lar \delta^\alpha+\delta^{1-\alpha}\|W_r\|^2 + \delta^{3-2\alpha}\|W_r\|^3
\end{align*}
This implies that $R_r$ maps a ball of radius $\delta^{\alpha/2}$ in $C_{Wr}$ into itself, provided that $\delta$ is small enough. Indeed, we see if $\|W_r\| \le \delta^{\alpha/2}$, then
\[
\|R_r(W_r)\| \lar \delta^{\alpha} + \delta^{1-\alpha}\delta^{\alpha} + \delta^{3-\alpha/2} \lar \delta^{\alpha} \le \delta^{\alpha/2}.
\]

Denote $h(W_1,W_2) = (W_1+W_2+(u_r+W_1)^2+(u_r+W_2)^2+(u_r+W_1)(u_r+W_2))$, then
\begin{align*}
\|R_r(W_1)-R_r(W_2)\| &= \sup|(T_\infty - t)^{3-\alpha} (W_1-W_2)h(W_1,W_2)|\\
& \lar |(T_\infty-t)h(W_1,W_2)|_\infty\|W_1-W_2\| ,
\end{align*}

since $|W_{1,2}| \lar (T_\infty-t)^{\alpha-2}$ and $|u_r| \le (T_\infty-t)^{-1}$, we have $|h(W_1,W_2)| \le |W_1+W_2|+O((T_\infty-t)^{-2})$, hence 
\[
|(T_\infty-t)h(W_1,W_2)|_\infty \lar (T_\infty-t)^{\alpha-1} \le \delta^{1-\alpha}
\]
Which shows $R_r(W)$ is Lipshitz in $W$ with small ($\delta^{1-\alpha}$) Lipschitz constants.

Equation \eqref{Wreqn} is re-written in the form 
\[
W_r(\sigma) = \exp\left(\int_{\sigma_T}^\sigma a(\rho)d\rho\right)W_T+\int_{\sigma_T}^\sigma \exp\left(\int_s^\sigma a(\rho)d\rho\right)\eps^{-1/3}\varphi R_r(W_r(s),\eps)ds := \mathcal{T}(W_r, W_T,\eps),
\]
Previous estimates show that
$\|\mathcal{T}(0,0,\eps) \| \le C\|u_r^3\| \le C\delta^{1-\alpha}$, and $\mathcal{T}(W_r,W_T,\eps) $ is Lipschitz in $W_r, W_T$ with order $\delta^{1-\alpha}$, and for $\|W_r\| \lar \delta^{\alpha/2}$, $\|\mathcal{T}(W_r,W_T,\eps)\| \lar \delta^{\alpha/2}$. Provided that $W_T$ is chosen small enough ($\rmO(\delta)$), as indicated by the following estimate.

\[
\left\|\exp\left(\int^{\sigma}_{\sigma_T} a \right)W_T\right\| =\sup_{\sigma \le \sigma_T} |\eps^{(\alpha-2)/3} e^{(\alpha-2)\sigma}e^{2(\sigma-\sigma_T)}e^{\int_{\sigma_T}^{\sigma}(a-2)} W_T| \lar |\eps^{\alpha/3} e^{\alpha\sigma}W_T| \le |W_T|
\]
\pagebreak
\subsection{Estimates for \texorpdfstring{$W_r(0)$}{Wzero} }
Using the fixed point formula, we can write down equation for $W_r(0)$:
\[
W_r(0) = \exp\left(\int_{\sigma_T}^0 a(\rho)d\rho \right) W_T + \int_{\sigma_T}^0 \exp\left(\int_{s}^0 a(\rho)d\rho \right)\eps^{-1/3}\varphi R_r(W_r(s),\eps)ds
\]

then using the fact that 
\begin{align*}
|\eps^{-1/3}\varphi R_r(W_r(s))| &\le \eps^{(2-\alpha)/3}e^{(2-\alpha)s} \| \eps^{-1/3}\varphi R_r(W_r(s))\| \\
&\lar \eps^{(2-\alpha)/3}e^{(2-\alpha)s}\left( \delta^{1-\alpha}\|W_r^2\|+\delta^{\alpha} \right)
\end{align*}
we estimate
\begin{align*}
\left|e^{2\sigma_T}\left( W_r(0) - \exp\left(\int_{\sigma_T}^0 a(\rho)d\rho \right) W_T \right)\right| &\le \int_0^{\sigma_T} e^{\int_s^0(a-2)}e^{-2(s-\sigma_T)} |\eps^{-1/3}\varphi R_r(W_r(s,\eps))|ds \\
& \lar \int_0^{\sigma_T} e^{-2(s-\sigma_T)}\eps^{(2-\alpha)/3}e^{(2-\alpha)s}(\delta^{\alpha}+\delta^{1-\alpha}\|W_r^2\|)ds \\
&\lar  \eps^{(2-\alpha)/3}\int_0^{\sigma_T} e^{-\alpha s}(\delta^{\alpha}+\delta^{1-\alpha}\|W_r^2\|)ds\\
&\lar \eps^{-\alpha/3}
\end{align*}



\subsection{Fixed point argument-set up for \texorpdfstring{$W_{l}$}{Wplus}}



\end{document}
