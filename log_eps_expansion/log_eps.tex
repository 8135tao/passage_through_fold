\documentclass[letterpaper,11pt]{article}

\usepackage{ucs}
\usepackage[utf8x]{inputenc}
\usepackage{graphicx}
\usepackage{amsfonts}
\usepackage{dsfont}
\usepackage{amssymb}
\usepackage{amsmath}
\usepackage{amsthm}
\usepackage[titletoc]{appendix}

\usepackage{enumerate}
\usepackage{stmaryrd}
\usepackage{fullpage}
\usepackage{ifthen}
\usepackage{subfigure}
\usepackage{epic}
\usepackage{authblk}
\usepackage{textcomp}
\usepackage[small]{caption}
\SetSymbolFont{stmry}{bold}{U}{stmry}{m}{n}

\usepackage{enumitem}

\usepackage[hypertexnames=false,colorlinks=true,linkcolor=blue,citecolor=blue]{hyperref}
\usepackage[numbers,comma,square,sort&compress]{natbib}
\usepackage[letterpaper,text={7in,9in},centering]{geometry}

\usepackage{bm}
\usepackage{color}
\usepackage{titlesec}
\setlength{\parindent}{0.0in}
\setlength{\parskip}{1.0ex plus0.2ex minus0.2ex}
\renewcommand{\baselinestretch}{1.1}
\graphicspath{{eps/}{pdf/}}
\setcaptionmargin{0.25in}
\def\captionfont{\itshape\small}
\def\captionlabelfont{\upshape\small}

\renewcommand{\labelenumi}{(\roman{enumi})}

\newcommand{\bqq}{\begin{equation}}
\newcommand{\eqq}{\end{equation}}
\newcommand{\bqs}{\begin{equation*}}
\newcommand{\eqs}{\end{equation*}}

\newcommand{\Ral}{\mathcal{R}}


\newcommand{\C}{\mathbb{C}}
\newcommand{\D}{\mathbb{D}}
\newcommand{\N}{\mathbb{N}}
\newcommand{\R}{\mathbb{R}} 
\newcommand{\Z}{\mathbb{Z}}

\newcommand{\rme}{\mathrm{e}}
\newcommand{\rmi}{\mathrm{i}}
\newcommand{\rmd}{\mathrm{d}}
\newcommand{\rmo}{{\scriptstyle\mathcal{O}}}
\newcommand{\rmO}{\mathcal{O}}
\newcommand{\eps}{\varepsilon}
\newcommand{\lar}{ \lesssim }


\newcommand{\Rho}{\bm{\rho}}
\newcommand{\bigma}{\bm{\sigma}}
\newcommand{\diag}{\operatorname{diag}}
\newcommand{\supp}{\operatorname{supp}}

\renewcommand{\qedsymbol}{$\blacksquare$}


\numberwithin{equation}{section}

\newenvironment{Hypothesis}[1]%
  {\begin{trivlist}\item[]{\bf Hypothesis #1 }\em}{\end{trivlist}}

\renewcommand{\arraystretch}{1.25}


% Define Theorem Styles ----------------------------------
\theoremstyle{plain}
\newtheorem{theorem}{Theorem}[section]
\newtheorem{proposition}[theorem]{Proposition}
\newtheorem{lemma}[theorem]{Lemma}
\newtheorem{corollary}[theorem]{Corollary}
\newtheorem{conjecture}[theorem]{Conjecture}
\newtheorem{main}[theorem]{Main Result}
\newtheorem{rmk}[theorem]{rmk}


\newcommand{\etal}{\textit{et al.}\ }

\newcommand{\greg}[1]{%
  {\color{blue}\textbf{Greg:} #1}%
 }
 
\newcommand{\arnd}[1]{%
  {\color{red}\textbf{Arnd:} #1}%
 }

\newenvironment{Proof}[1][.]%
 {\begin{trivlist}\item[]\textbf{Proof#1 }}%
 {\hspace*{\fill}$\rule{0.3\baselineskip}{0.35\baselineskip}$\end{trivlist}}

\renewcommand\labelitemi{$\bullet$}

\title{Passage through a fold without a phase space}
\author{author}
\date{2016}
\begin{document}
Here is what I was doing with the exist time:



The Ansatz chosen was 
\[
u_*(t; u_0)  = \eps^{1/3}u_R(\eps^{1/3}(t-\eps^{-1}\delta); u_0)
\]

where $u_R(\cdot ; u_0)$ was the solution to the Ricatti equation with initial condition $u_R(0; u_0) = u_0$.

Which has the following asymptotics:
\[
u_R( s ; u_0) = (\Omega_\infty(u_0) - s)^{-1} + (\Omega_\infty(u_0) - s) r(\Omega_\infty(u_0) - s; u_0) 
\]
with 
\[
r(\Omega_\infty(u_0) - s; u_0) = -\frac{\Omega_\infty}{3} + \rmO( |\Omega_\infty(u_0) - s |) 
\]

Then we take a solution of the form
$u_A(t) = u_*(t) + w_r(t)$, plug it into the equation
\[
\frac{d}{dt} u(t) = \mu + u^2 + f(u, \mu; \eps)
\]
to get the solution satisfied by $w_r$, which is
\begin{equation}
\frac{d}{dt} w_r -2u_* w_r = w_r^2 + f(u_*+w_r, \mu; \eps)
\end{equation}


After the rescaling this becomes
\begin{equation}\label{W_r}
\left( \frac{d}{d\sigma} - a(\sigma) \right) W_r = \eps^{-1/3} \varphi \left(W_r^2 + f(u_*+W_r, \mu; \eps)\right)
\end{equation}
for $\sigma \in (\sigma_m, \sigma_T)$.

where we have
$|a(\sigma) - 2 | \le Ce^{-2\sigma}$ as $\sigma \to \infty$.

We have proposed the function space
\[
C_r = \{ w(\sigma) : \sup|\eps^{(\alpha-2)/3} e^{(\alpha-2)\sigma} w|<\infty \}
\]

And concluded that we need boundary condition at $\sigma_T$ to complete the fixed point argument to solve equation \eqref{W_r}.



Since we at the start required to have
$u(T) = \delta$ as the definition for the exit time $T$. The simple choice we took was to use $T$ such that
\[
u_*(T) = 0
\]
and hence
\[
w_r(T) = 0 \text{ or }W_r(\sigma_T) = 0
\]
as the needed boundary condition. The fixed point argument can be completed, as shown in section ``Region A'' of the entire writeup. 


But from $u_*(T) = \delta$ we can compute the asymptotics of $T$ to be
\[
T = \eps^{-1}\delta + \eps^{-1/3}\Omega_\infty - \delta^{-1} + \rmO(\eps^{2/3}),
\]
so that in the end we have
\[
\mu(T) = \eps T- \delta = \eps^{2/3}\Omega_\infty -\eps\delta^{-1}+\rmO(\eps^{5/3})
\]
which I think is right, as this agrees with Lemma 2.10 in Krupa Szmolyan.

The problems is this expansion did not use any information about our correction $w_r$ at $t=T$ (it is $0$ by our assumption), so we cannot to expect to get the $\eps\log(\eps)$ just by using the asymptotics of the Riccati solution.
\end{document}